\documentclass{elegantbook}

%%=====编译方式pdflatex============================

% settings.tex
% 储存配置信息

\usepackage{setspace}
\usepackage{bm}
\usepackage{makecell}
\usepackage{lipsum}
\usepackage{texnames}
\usepackage{ntheorem}

%green color
   \definecolor{main1}{RGB}{0,120,2}
   \definecolor{second1}{RGB}{230,90,7}
   \definecolor{third1}{RGB}{0,160,152}
%cyan color
   \definecolor{main2}{RGB}{0,175,152}
   \definecolor{second2}{RGB}{239,126,30}
   \definecolor{third2}{RGB}{120,8,13}
%blue color
   \definecolor{main3}{RGB}{20,50,104}
   \definecolor{second3}{RGB}{180,50,131}
   \definecolor{third3}{RGB}{7,127,128}

%%=====新数学命令================================
\newcommand\dif{\mathrm{d}}
\newcommand\no{\noindent}
\newcommand\dis{\displaystyle}
\newcommand\ls{\leqslant}
\newcommand\gs{\geqslant}

\newcommand\limit{\dis\lim\limits}
\newcommand\limn{\dis\lim\limits_{n\to\infty}}
\newcommand\limxz{\dis\lim\limits_{x\to0}}
\newcommand\limxi{\dis\lim\limits_{x\to\infty}}
\newcommand\limxpi{\dis\lim\limits_{x\to+\infty}}
\newcommand\limxni{\dis\lim\limits_{x\to-\infty}}
\newcommand\limtpi{\dis\lim\limits_{t\to+\infty}}
\newcommand\limtni{\dis\lim\limits_{t\to-\infty}}

\newcommand\sumn{\dis\sum\limits_{n=1}^{\infty}}
\newcommand\sumnz{\dis\sum\limits_{n=0}^{\infty}}

\newcommand\sumi{\dis\sum\limits_{i=1}^{\infty}}
\newcommand\sumiz{\dis\sum\limits_{i=0}^{\infty}}
\newcommand\sumin{\dis\sum\limits_{i=1}^{n}}
\newcommand\sumizn{\dis\sum\limits_{i=0}^{n}}

\newcommand\sumk{\dis\sum\limits_{k=1}^{\infty}}
\newcommand\sumkz{\dis\sum\limits_{k=0}^{\infty}}
\newcommand\sumkn{\dis\sum\limits_{k=0}^n}
\newcommand\sumkfn{\dis\sum\limits_{k=1}^n}

\newcommand\pzx{\dis\frac{\partial z}{\partial x}}
\newcommand\pzy{\dis\frac{\partial z}{\partial y}}

\newcommand\pfx{\dis\frac{\partial f}{\partial x}}
\newcommand\pfy{\dis\frac{\partial f}{\partial x}}

\newcommand\pzxx{\dis\frac{\partial^2 z}{\partial x^2}}
\newcommand\pzxy{\dis\frac{\partial^2 z}{\partial x\partial y}}
\newcommand\pzyx{\dis\frac{\partial^2 z}{\partial y\partial x}}
\newcommand\pzyy{\dis\frac{\partial^2 z}{\partial y^2}}

\newcommand\pfxx{\dis\frac{\partial^2 f}{\partial x^2}}
\newcommand\pfxy{\dis\frac{\partial^2 f}{\partial x\partial y}}
\newcommand\pfyx{\dis\frac{\partial^2 f}{\partial y\partial x}}
\newcommand\pfyy{\dis\frac{\partial^2 f}{\partial y^2}}

\newcommand\intzi{\dis\int_{0}^{+\infty}}
\newcommand\intd{\dis\int}
\newcommand\intab{\dis\int_a^b}

\newcommand\ma{\mathcal{A}}
\newcommand\mb{\mathcal{B}}
\newcommand\mc{\mathcal{C}}
\newcommand\me{\mathcal{E}}
\newcommand\mg{\mathcal{g}}

\newcommand\mr{\mathbb{R}}
\newcommand\mz{\mathbb{Z}}

\newcommand\mx{\mathbf{x}}
\newcommand\mX{\mathbf{X}}
\newcommand\my{\mathbf{y}}
\newcommand\mY{\mathbf{Y}}
%%==============================================

%%=====定义新数学符号=====================
\DeclareMathOperator{\sgn}{sgn}
\DeclareMathOperator{\arccot}{arccot}
\DeclareMathOperator{\arccosh}{arccosh}
\DeclareMathOperator{\arcsinh}{arcsinh}
\DeclareMathOperator{\arctanh}{arctanh}
\DeclareMathOperator{\arccoth}{arccoth}
\DeclareMathOperator{\grad}{\bf{grad}}
\DeclareMathOperator{\argmax}{argmax}
%====================================


\author{杨敬轩}
\date{\today}
\email{yangjingxuan@stu.hit.edu.cn}
\zhtitle{概率论}
%\zhend{模板}
\entitle{Probability}
%\enend{Template}
\version{\LaTeX\ v1.1}
\myquote{Victory won\rq t come to us unless we go to it.}
\logo{logop.jpg}
\cover{cover.pdf}

\begin{document}
\maketitle
\tableofcontents
\mainmatter
\chapter{Axioms of Probability}

\begin{definition}{Sample Space}{Sample}
The sample space $\Omega$ of an experiment is the set of all possible outcomes of the experiment.
\end{definition}

\begin{definition}{Event}{Event}
An event of an experiment is a subset of the sample space $\Omega$ of the experiment. 
We call $\Omega$ the certain event and $\Phi$ the impossible event of the experiment. 
We say that an event $A$ occurs if the outcome of the experiment belongs to $A$.
\end{definition}

\begin{definition}{$\sigma$-algebra}{algebra}
A $\sigma$-algebra $\ma$ of subsets of a sample space $\Omega$ is a collection of subset of $\Omega$ s.t.\\
(1) $\Omega\in\ma$,\\
(2) $\ma$  is closed under complementation, i.e., if $A\in\ma$ , then $\Omega\backslash A\in\ma$,\\
(3) $\ma$  is closed under countable union, i.e., if $A_n\in\ma$ for $n=1,2,\cdots$, then $$\bigcap_{n=1}^\infty A_n \in\ma.$$
\end{definition}

\begin{theorem}{Properties of $\sigma$-algebra}{Properties}
Suppose $A$ is a $\sigma$-algebra of subsets of a sample space $\Omega$.
\\
(1) $\Phi\in\ma,$\\
(2) $A$ is closed under f\/inite union,\\
(3) $A$ is closed under countable and f\/inite intersection.
\end{theorem}

\begin{proof}
\\[4cm]\vspace{0.01cm}
\end{proof}

\begin{theorem}{Intersection of $\sigma$-algebra}{Intersection}
Suppose $\Gamma$ is a nonempty collection of $\sigma$-algebra of subsets of a sample space $\Omega$. Then the intersection $$B=\bigcap_{A\in\Gamma}A$$ of the $\sigma$-algebra in $\Gamma$ is also a $\sigma$-algebra of subsets of $\Omega$.
\end{theorem}

\begin{proof}
\\[4cm]\vspace{0.01cm}
\end{proof}

\begin{corollary}{Existence of Smallest $\sigma$-algebra}{Existence}
Suppose $\mathcal{C}$ is a collection of subsets of a sample space $\Omega$. Then there exists a smallest $\sigma$-algebra of subsets of $\Omega$ including $\mathcal{C}$.
\end{corollary}

\begin{proof}
\\[4cm]\vspace{0.01cm}
\end{proof}

\begin{definition}{Generated $\sigma$-algebra}{Generated}
Let $\mathcal{C}$ be a collection of subsets of a sample space $\Omega$, we def\/ine the $\sigma$-algebra of subsets of $\Omega$ generated by $\mathcal{C}$ as the smallest $\sigma$-algebra of subsets of $\Omega$ including $\mathcal{C}$ and denoted it as $\sigma$($\mathcal{C}$).
\end{definition}

\begin{definition}{Probability Measure}{Probability Measure}
Let $A$ be a $\sigma$-algebra of subsets of a sample space $\Omega$, a probability measure $P: \ma\rightarrow\mr$ on $A$ is a real-valued function on $A$ s.t.\\
(1) Nonnegativity: $P(A)\gs0,\ \forall A\in\ma,$ \\
(2) Normalization: $P(\Omega)=1$, \\
(3) Countable additivity: If $A_1,A_2, \cdots$ are pairwise disjoint events in 
$A$ then $$P\left(\bigcup_{n=1}^\infty A_n \right)=\sumn P(A) .$$
For an event $A\in\ma$ , we call $P(A)$ the probability of the event $A$.
\end{definition}

\begin{definition}{Probability Space}{Probability Space}
A probability space is an ordered triple $(\Omega,\ma, P)$ consisting of a sample space $\Omega$ , a $\sigma$-algebra $\ma$  of subsets of $\Omega$, and a probability measure $P$ on $\ma$ .
\end{definition}

\begin{theorem}{A Kind of Probability Measure }{A Kind}
Suppose $\Omega=\{w_1,w_2,\cdots\}$, $\ma\in\mathcal{P}(\Omega)$ and $$P(A)=\sum\limits_{w_i\in\ma}P_i,\  \text{for all}\ A\in\mathcal{P}(\Omega),$$
where $P_i\gs0,\ \forall i=1,2,\cdots$ and $$\sumi P_i =1,$$ then $P$ is a probability measure on $\mathcal{P}(\Omega)$. \\
A similar result holds if $\Omega=\{w_1,w_2,\ \cdots,w_N\}$, where $N\gs1$.
\end{theorem}

\begin{proof}
\\[4cm]\vspace{0.01cm}
\end{proof}

\begin{corollary}{A Kind of Probability Measure (special)}{special}
Suppose $\Omega=\{w_1,w_2,\ \cdots,w_N\}$, $\ma\in\mathcal{P}(\Omega)$, and $$P(A)=\frac{|A|}{N}$$ for all $A\in\mathcal{P}(\Omega)$, then $P$ is a probability measure on $\mathcal{P}(\Omega)$.
\end{corollary}

\begin{proof}
\\[4cm]\vspace{0.01cm}
\end{proof}

\begin{theorem}{Classical def\/inition of probability}{Classical}
Suppose $\Omega=\{w_1,w_2,\ \cdots,w_N\}$, $\ma\in\mathcal{P}(\Omega)$ and $P$ is a probability measure on $\mathcal{P}(\Omega)$  such that $P({w_1})= P({w_2})=\cdots= P({w_N})$, then $$P(A)=\frac{|A|}{N}$$ for all $A\in\mathcal{P}(\Omega)$.
\end{theorem}

\begin{proof}
\\[3cm]\vspace{0.01cm}
\end{proof}

\begin{theorem}{Properties of Probability Measure}{Properties}
Suppose $(\Omega,\ma, P)$ is a probability space.\\
(1) $P(\Phi)=0,$\\
(2) $P(A)+P(A^c)=1$. Therefore, $0\ls P(A)\ls1$, for all $A\in\ma$.\\
(3) Finite additivity: If $A_1,A_2,\ \cdots,A_N$ are pairwise disjoint events in $A$,              then $$P\left(\bigcup_{n=1}^NA_n \right)=\sum_{n=1}^NP(A). $$
\end{theorem}

\begin{proof}
\\[3cm]\vspace{0.01cm}
\end{proof}

\begin{theorem}{Properties of Probability Measure}{Properties of P}
Suppose $(\Omega,\ma, P)$ is a probability space, and suppose $A,B\in\ma$.
\\
(1) If $A_1,A_2,\cdots$ are pairwise disjoint events on $A$ and 
$$ \bigcup_{n=1}^\infty A_n =\Omega,$$ then 
$$P(A)= \sumi P\left(A \bigcap A_n \right).$$
(2) If $B\subseteq A$, then $P(A)=P(A\cap B)+P(A\cap A^c )$ for all $A,B\in\ma$.\\
(3) $P(A\cap B)  \ls \min \{P(A),P(B)\}  \ls \max\{ P(A),P(B)\} \ls  P(A\cup B)$.
\end{theorem}

\begin{proof}
\\[4cm]\vspace{0.01cm}
\end{proof}

\begin{corollary}{Finite Additivity under Union}{Finite Additivity}
Suppose $(\Omega,\ma, P)$ is a probability space, $A\in\ma$ , $A_1,A_2,\cdots$ are pairwise disjoint events in $\ma$, and $$P\left(\bigcup_{n=1}^\infty A_n \right)=1,$$ then 
$$P(A)=\sum_{n=1}^\infty P\left(A\bigcap A_n \right).$$ 
\end{corollary}

\begin{proof}
\\[4cm]\vspace{0.01cm}
\end{proof}

\begin{theorem}{Inclusion-exclusion Identity}{Inclusion-exclusion identity}
Suppose $(\Omega,\ma, P)$ is a probability space, and suppose $A_1,A_2,\ \cdots,A_n\in\ma $, where $n \gs2$, then $$P\left(\bigcup_{i=1}^n A_i \right)=\sumkfn(-1)^{k+1}\cdot  \sum_{1\ls i_1<i_2<\cdots<i_k\ls n}P\left(A_{i_1} \bigcap A_{i_2 } \bigcap\cdots\bigcap A_{i_k } \right) .$$
\end{theorem}

\begin{proof}
\\[4cm]\vspace{0.01cm}
\end{proof}

\begin{lemma}{Generated Pairwise Disjoint}{Generated Pairwise Disjoint}
Suppose $A$ is a $\sigma$-algebra of subsets of a sample space $\Omega$, suppose $A_1,A_2,\cdots\in\ma$, $B_1=A_1$, and $$B_n=A_n\setminus\bigcup_{i=1}^{n-1}A_i$$  for all $n\gs2$, then $B_1,B_2,\cdots$ are pairwise disjoint events in $\ma$, $$\bigcup_{i=1}^nA_i =\bigcup_{i=1}^nB_i$$  for all $n\gs1$, and $$\bigcup_{n=1}^\infty A_n =\bigcup_{n=1}^\infty B_n.$$
\end{lemma}

\begin{proof}
\\[4cm]\vspace{0.01cm}
\end{proof}

\begin{theorem}{Inclusion-exclusion Inequality}{Inclusion-exclusion inequality}
Suppose $(\Omega,\ma, P)$ is a probability space, and suppose $A_1,A_2,\ \cdots,A_n\in\ma$, where $n\gs2$, then
$$P\left(\bigcup_{i=1}^n A_i \right)
\begin{cases}
\ls\sum\limits_{k=1}^m(-1)^{k+1}\cdot  \sum\limits_{1\ls i_1<i_2<\cdots<i_k\ls n}P\left(A_{i_1} \bigcap A_{i_2 } \bigcap\cdots\bigcap A_{i_k } \right),& \text{if $m$ is odd}\\
\gs\sum\limits_{k=1}^m(-1)^{k+1}\cdot  \sum\limits_{1\ls i_1<i_2<\cdots<i_k\ls n}P\left(A_{i_1} \bigcap A_{i_2 } \bigcap\cdots\bigcap A_{i_k } \right),& \text{if $m$ is even}
\end{cases}
$$
where $1\ls m\ls n$.\\
In particular,
$$P\left(\bigcup_{i=1}^n A_i \right)\ls\sumin P(A_i),$$
$$P\left(\bigcup_{i=1}^n A_i \right)\gs\sumin P(A_i)-
\sum\limits_{1\ls i<j\ls n}P\left(A_i\bigcap A_j\right).$$
\end{theorem}

\begin{proof}
\\[4cm]\vspace{0.01cm}
\end{proof}

\begin{theorem}{Boole's Inequality}{Boole's inequality}
Suppose $(\Omega,\ma, P)$ is a probability space, and suppose $A_1,A_2,\cdots\in\ma$ , then 
$$P\left(\bigcup_{i=1}^\infty A_i \right)\ls\sum_{i=1}^\infty P(A_i ) .$$
\end{theorem}

\begin{proof}
\\[4cm]\vspace{0.01cm}
\end{proof}

\begin{definition}{Monotonicity}{Monotonicity}
Let $(\Omega,\ma, P)$ be a probability space.\\
A sequence $\{A_1,A_2,\cdots\}$ of events in $A$ is increasing if $A_1\subseteq A_2\subseteq\cdots$\\
A sequence $\{A_1,A_2,\cdots\}$ of events in $A$ is decreasing if $A_1\supseteq A_2\supseteq\cdots$
\end{definition}

\begin{definition}{Limit of Events}{Limit of Events}
Let $(\Omega,\ma, P)$ be a probability space.\\
(1) The limit $\limn A_n$ of an increasing sequence $\{A_1,A_2,\cdots\}$ of events in $A$ is the event that at least one of the events occurs, i.e., $$\limn A_n=\bigcup_{n=1}^\infty A_n.$$
(2) The limit $\limn A_n$ of a decreasing sequence $\{A_1,A_2,\cdots\}$ of events in $A$ is the event that all the events occur, i.e., $$\limn A_n=\bigcap_{n=1}^\infty A_n.$$
\end{definition}

\begin{theorem}{Continuity of Probability Measure}{Continuity}
Let $(\Omega,\ma, P)$ be a probability space.\\
(1) Suppose that $\{A_1,A_2,\cdots\}$ is an increasing sequence of events in $A$. Then $$P\left(\limn A_n\right)=\limn P(A_n).$$
(2) Suppose that $\{A_1,A_2,\cdots\}$ is a decreasing sequence of events in $A$. Then $$P\left(\limn A_n\right)=\limn P(A_n).$$
\end{theorem}

\begin{proof}
\\[4cm]\vspace{0.01cm}
\end{proof}

\begin{remark}{Not Necessary}{not necessary}
If $P(A)=0$, then it is not necessary that $A=\Phi$, e.g., $\Omega=(0,1)$ and $A=A_\alpha,\ \alpha\in(0,1)$.
If $P(A)=1$, then it is not necessary that $A=\Omega$, e.g., $\Omega=(0,1)$ and $A=A_\alpha^c,\ \alpha\in(0,1)$.
\end{remark}

\begin{definition}{Length}{Length}
The length of the intervals $(a,b),\ [a,b),\ (a,b],\ [a,b]$ are def\/ined to be $(b-a)$.
\end{definition}

\begin{definition}{Random}{Random}
A point is said to be randomly selected from an interval $(a,b)$ if any subintervals of $(a,b)$ with the same length are equally likely to contain the randomly selected point.
\end{definition}

\begin{theorem}{Probability of Randomness}{Probability of Randomness}
The probability that a randomly selected point from $(a,b)$ falls in the subinterval $(\alpha,\beta)$ of $(a,b)$ is $$P=\frac{\beta-\alpha}{b-a}.$$
\end{theorem}

\begin{proof}
\\[5cm]\vspace{0.01cm}
\end{proof}

\begin{definition}{Borel Algebra}{Borel Algebra}
The $\sigma$-algebra of subsets of $(a,b)$ generated by the set of all subintervals of $(a,b)$ is called Borel algebra associated with $(a,b)$ and is denoted $\mathcal{B}_{(a,b)}$.
\end{definition}

\begin{theorem}{Existence of Probability Measure}{Existence}
For any interval $(a,b)$, there exists a unique probability measure $P$ on $\mathcal{B}_{(a,b)}$ s.t., $$P\left[(\alpha,\beta)\right]=\frac{\beta-\alpha}{b-a},$$
for all $(\alpha,\beta)\subseteq(a,b)$.
\end{theorem}

\begin{proof}
\\[4cm]\vspace{0.01cm}
\end{proof}

\chapter{Combinational Methods}

\begin{theorem}{Counting Principle}{Counting Principle}
There are $n_1\times n_2\times\cdots\times n_k$ dif\/ferent ways in which we can f\/irst choose an element from a set of $n_1$  elements, then an element from a set of $n_2$ elements,..., and f\/inally an element from a set of $n_k$ elements.
\end{theorem}

\begin{proof}
\\[2cm]\vspace{0.01cm}
\end{proof}

\begin{definition}{Permutation}{Permutation}
An ordered arrangement of $r$ objects from a set $A$ containing $n$ objects is called an $r$-arrangement permutation of $A$, where $0\ls r\ls n$. 

An $n$-element permutation of $A$ is called a permutation of $A$. The number of dif\/ferent $r$-permutation permutations of $A$ is given by $$_nP_r =n\times(n-1)\times(n-2)\times\cdots\times(n-r+1)=\frac{n!}{(n-r)!}.$$
\end{definition}

\begin{theorem}{Permutation with Types}{Permutation with Types}
The number of dif\/ferent (w.r.t. types) permutations of $n$ objects of $k$ dif\/ferent types is $$\frac{n!}{n_1 !\times n_2 !\times\cdots\times n_k !},$$
where $n_1$  are alike, $n_2$  are alike,..., $n_k$  are alike, and $n=n_1+n_2+\cdots+n_k $.
\end{theorem}

\begin{proof}
\\[2cm]\vspace{0.01cm}
\end{proof}

\begin{definition}{Combination}{Combination}
An unordered arrangement of $r$ objects from a set $A$ containing $n$ objects is called an $r$-element combination of $A$. The number of dif\/ferent $r$-element combinations of $A$ is given by 
$$_n C_r =\binom n r=\frac{_n P_r}{r!}=\frac{n!}{(n-r)!r!}.$$
\end{definition}

\begin{theorem}{Property of Combination}{Property of Combination}
$$\sum\limits_{i=0}^k\binom {n+i} i=\sum\limits_{i=0}^k \binom {n+i} n
=\binom {n+k+1} k$$
\end{theorem}

\begin{proof}
\\[4cm]\vspace{0.01cm}
\end{proof}

\begin{theorem}{Multinomial Expansion}{Multinomial Expansion}
$$(x_1+x_2+\cdots+x_k )^n=\sum\limits_
{\mbox{\tiny $\begin{aligned}
&n_1+n_2+\cdots+n_k=n \\
&n_1,n_2,\ \cdots,n_k\gs0
\end{aligned}$}}
 \frac{n!}{n_1 !\times n_2 !\times\cdots\times n_k !}\cdot x_1^{n_1 } x_2^{n_2 }\cdots x_k^{n_k},\forall n\gs0.$$
\end{theorem}

\begin{proof}
\\[4cm]\vspace{0.01cm}
\end{proof}

\begin{corollary}{Binomial Expansion}{Binomial Expansion}
$$(x+y)^n=\sumin\binom n i  x^i y^{n-i},\forall n\gs 0.$$
\end{corollary}

\begin{theorem}{Stirling's Formula}{Stirling's Formula}
$$\sqrt{2\pi n} 
\left(\frac{n}{e}\right)^n
\cdot
\exp\left(\frac{1}{12n}-\frac{1}{360 n^2 }\right) 
<n!
<\sqrt{2\pi n} 
\left(\frac{n}{e}\right)^n
\cdot\exp\left(\frac{1}{12n}\right),\ 
\forall n\gs1.$$
Therefore, 
$$n!\sim\sqrt{2\pi n} \left(\frac{n}{e}\right)^n,\ \text{i.e.},\ \limn \frac{n!}{\sqrt{2\pi n} \left(\frac{n}{e}\right)^n}=1.$$
\end{theorem}

\begin{proof}
\\[4cm]\vspace{0.01cm}
\end{proof}

\chapter{Conditional Probability and Independence}

\begin{definition}{Conditional Probability}{Conditional Probability}
Let $(\Omega,\ma, P)$ be a probability space, and $A,B\in\ma$. The conditional probability of $A$ given $B$, denoted $P(A|B)$, is given by
$$P(A|B)=
\begin{cases}
\dfrac{P(A\cap B)}{P(B)},\ \text{if}\  P(B)>0,  \\
     \qquad 0          ,\qquad\ \text{if}\  P(B)=0.
\end{cases}
$$
\end{definition}

\begin{remark}{Property of Conditional Probability}{Property of Conditional Probability}
$$P\left(A\cap B\right)=P(B)\cdot P(A|B),\forall A,B\in\ma.$$ 
\end{remark}

\begin{proof}
\\[2cm]\vspace{0.01cm}
\end{proof}

\begin{theorem}{Conditional Probability Space}{Conditional Probability Space}
Suppose $(\Omega,\ma, P)$ is a probability space, and suppose $P(B)>0$, for some $B\in\ma$. 
Then the conditional probability function $P(\cdot|B): \ma\rightarrow\mr$ is a probability measure on $\ma$, and hence $(\Omega,\ma, P(\cdot|B))$ is a probability space.
\end{theorem}

\begin{proof}
\\[2cm]\vspace{0.01cm}
\end{proof}

\begin{theorem}{Reduction of Probability Space}{Reduction of Probability Space}
Suppose $(\Omega,\ma, P)$ is a probability space, and suppose $P(B)>0$, for some $B\in\ma$. Let $\ma_B:\{A\in\ma: A\subseteq B\}$ and $P_B (A)=P(A|B)$  for all $A\in\ma_B$.  Then $\ma_B$ is a $\sigma$-algebra of subsets of $B$ and $P_B$ is a probability measure on $\ma_B$, and hence $(B,\ma_B,P_B )$ is a probability space.
\end{theorem}

\begin{proof}
\\[4cm]\vspace{0.01cm}
\end{proof}

\begin{remark}{Conversion of Reduced and Conditional Probability Space}{1}
Note that $P(A|B)=P(A\cap B|B)=P_B (A\cap B),\forall A\in\ma.$ 
And $P(A|B)=P_B (A)$, if $A\in\ma$ and $A\subseteq B$.
\end{remark}

\begin{proof}
\\[4cm]\vspace{0.01cm}
\end{proof}

\begin{theorem}{Law of Multiplication}{Law of Multiplication}
Suppose $(\Omega,\ma, P)$ is a probability space, and suppose $A_1,A_2,\ \cdots,A_n\in\ma$. Then $$P\left(A_1\cap A_2\cap\cdots\cap A_n\right)=P(A_1 )P(A_2|A_1 )\cdots P\left(A_n|A_1 \cap A_2\cap\cdots\cap A_{n-1}\right).$$
\vspace{0.05cm}
\end{theorem}

\begin{proof}
\\[4cm]\vspace{0.01cm}
\end{proof}

\begin{theorem}{Law of Total Probability (inf\/inite)}{Law of Total Probability (infinite)}
Suppose $(\Omega,\ma, P)$ is a probability space, and suppose $B_1,B_2,\cdots\in\ma$ are pairwise disjoint and $\bigcup_{n=1}^\infty B_n=\Omega$. Then,\\
(1) $P(A)=\sumn P(B_n )\cdot P(A|B_n ),\forall A\in\ma$.\\
(2) $P(A|B)=\sumn P(B_n |B)\cdot P(A|B\cap B_n ),\forall A,B\in\ma.$
\end{theorem}

\begin{proof}
\\[4cm]\vspace{0.01cm}
\end{proof}

\begin{corollary}{Law of Total Probability (f\/inite)}{Law of Total Probability (finite)}
Suppose $(\Omega,\ma, P)$ is a probability space, and suppose $B_1,B_2,\cdots B_n\in\ma$ are pairwise disjoint and $\bigcup_{i=1}^n B_i=\Omega$. Then,\\
(1) $P(A)=\sumin P(B_i )\cdot P(A|B_i ),\forall A\in\ma$.\\
(2) $P(A|B)=\sumin P(B_i |B)\cdot P(A|B\cap B_i ),\forall A,B\in\ma.$
\end{corollary}

\begin{proof}
\\[4cm]\vspace{0.01cm}
\end{proof}

\begin{theorem}{Bayes' Theorem (inf\/inite)}{Bayes' Theorem (infinite)}
Suppose $(\Omega,\ma, P)$ is a probability space, and suppose $B_1,B_2,\cdots\in\ma$ are pairwise disjoint and $\bigcup_{n=1}^\infty B_n=\Omega$.  Then $$P(B_k|A)=\frac{P(B_k)\cdot P(A|B_k)}{\sumn P(B_n )\cdot P(A|B_n )},\forall A\in\ma, P(A)>0,k=1,2,\cdots$$
\end{theorem}

\begin{proof}
\\[4cm]\vspace{0.01cm}
\end{proof}

\begin{corollary}{Bayes' Theorem (f\/inite)}{Bayes' Theorem (finite)}
Suppose $(\Omega,\ma, P)$ is a probability space, and suppose $B_1,B_2,\cdots B_n\in\ma$ are pairwise disjoint and $\bigcup_{i=1}^n B_i=\Omega$.  Then $$P(B_k|A)=\frac{P(B_k)\cdot P(A|B_k)}{\sumin P(B_i )\cdot P(A|B_i )},\forall A\in\ma, P(A)>0,k=1,2,\ \cdots,n$$
\end{corollary}

\begin{proof}
\\[4cm]\vspace{0.01cm}
\end{proof}

\begin{theorem}{Properties of Conditional Probability}{Properties of Conditional Probability}
Suppose $(\Omega,\ma, P)$ is a probability space, and suppose $A,B\in\ma$.

(1) $P(A|B)>P(A)\Leftrightarrow P(A\cap B)>P(A)\cdot P(B)\Leftrightarrow P(B|A)>P(B)$

 $\begin{aligned}
\text{(2) }P(A|B)<P(A),P(B)>0&\Leftrightarrow P\left(A\cap B\right)<P(A)\cdot P(B)\\
                              &\Leftrightarrow P(B|A)<P(B),P(A)>0
                                     \end{aligned}$
                                                          
(3) $P(A|B)=P(A)\Rightarrow P(A\cap B)=P(A)\cdot P(B)$

    $\quad\ P(A\cap B)=P(A)\cap P(B),\ P(A)=0$ or $P(B)>0\Rightarrow P(A|B)=P(A)$
    
If $P(A)=0$ or $P(B)>0$, then $P(A|B)=P(A)\Leftrightarrow P(A\cap B)=P(A)\cdot P(B)  $

\end{theorem}

\begin{proof}
\\[4cm]\vspace{0.01cm}
\end{proof}

\begin{definition}{Independence}{Independence}
Let $(\Omega,\ma, P)$ be a probability space, and $A,B\in\ma$. 
If $P(A\cap B)=P(A)\cdot P(B)$, then $A$ and $B$ are said to be independent, denoted $A\perp B$. If $A$ and $B$ are not independent, they are said to be dependent. 
Furthermore, if $P(A|B)>P(A)$, then $A$ and $B$ are said to be positively correlated, and if $P(A|B)<P(A)$, then $A$ and $B$ are said to be negatively correlated.
\end{definition}

\begin{theorem}{Properties of Independence}{Properties of Independence}
Suppose $(\Omega,\ma, P)$ is a probability space, and suppose  $A,B\in\ma$.

(1) If $P(A)=0$ or $P(A)=1$, then $A\perp B,\ \forall B\in\ma$.

(2) If $A\subseteq B$ and $A\perp B$, then either $P(A)=0$ or $P(B)=1$.

(3) If $A$ and $B$ are disjoint and $P(A)>0$, $P(B)>0$, then they are dependent.

\end{theorem}

\begin{proof}
\\[4cm]\vspace{0.01cm}
\end{proof}

\begin{theorem}{Independence of Two Events}{Independence of Two Events}
Suppose $(\Omega,\ma, P)$ is a probability space, and suppose $A,B\in\ma$, and $A\perp B$.

Then $A^*\perp B^*$, i.e., $P(A^*\cap B^* )=P(A^* )\cdot P(B^* ),\forall  A^*=A,\ A^c;\ B^*=B,\ B^c$. 
\end{theorem}

\begin{proof}
\\[4cm]\vspace{0.01cm}
\end{proof}

\begin{corollary}{Conditional Probability with Independence }{Conditional Probability with Independence }
Suppose $(\Omega,\ma, P)$ is a probability space, and suppose $A,B\in\ma$, and $A\perp B$.

If $P(B)>0$, then $P(A^*|B)=P(A^* ),\forall A^*=A,\ A^c$.\\
If $P(B)<1$, then $P(A^*|B^c )=P(A^* ),\forall A^*=A,\ A^c$.

\end{corollary}

\begin{proof}
\\[4cm]\vspace{0.01cm}
\end{proof}

\begin{remark}{Conditional Probability with Independence}{Conditional Probability with Independence}
If $A\perp B$ and $P(B)>0$, then knowledge about the occurrence of $B$ does not change the probability of the occurrence of $A^*$.

If $A\perp B$ and $P(B)<1$, then knowledge about the occurrence of $B^c$  does not change the probability of the occurrence of $A^*$.

\end{remark}

\begin{proof}
\\[4cm]\vspace{0.01cm}
\end{proof}

\begin{definition}{Independent Set}{Independent Set}
Let $(\Omega,\ma, P)$ be a probability space, and $A_1,A_2,\ \cdots,A_n\in\ma$, where $n\gs2$.\\
If $P\left(A_{i_1}\bigcap A_{i_2}\bigcap\cdots\bigcap A_{i_k}\right)=P(A_{i_1} )P(A_{i_2})\cdots P(A_{i_k } ),\forall 2\ls k\ls n$,
$$ \#=\sum\limits_{k=2}^n \binom n k=2^n-n-1,1\ls i_1<i_2<\cdots<i_k\ls n,\#\triangleq \text{number}.$$
Then $A_1,A_2,\ \cdots,A_{n }$ are said to be independent; otherwise, they are said to be dependent.
\vspace{0.05cm}
\end{definition}

\begin{remark}{Sub Independent Set}{Sub Independent Set}
If $A_1,A_2,\ \cdots,A_n\in\ma$ are independent, then $A_{i_2},A_{i_2},\ \cdots,A_{i_k}$ are independent, $\forall 2\ls k\ls n,\ 1\ls i_1<i_2<\cdots <i_k\ls n$. 
\end{remark}

\begin{proof}
\\[4cm]\vspace{0.01cm}
\end{proof}

\begin{theorem}{Equivalent Statements of Independence}{Equivalent Statements of Independence}
Suppose $(\Omega,\ma, P)$ is a probability space, $A_1,A_2,\ \cdots,A_n\in\ma$, where $n\gs2$. The following statements are equivalent:

(1) $A_1,A_2,\ \cdots,A_n$ are independent.

(2) $P\left(A_{i_1}^*\bigcap A_{i_2}^*\bigcap\cdots\bigcap A_{i_k}^*\right)=P\left(A_{i_1}^*\right) P\left(A_{i_2}^* \right)\cdots P\left(A_{i_k}^* \right),\ \forall 2\ls k\ls n,$

$\quad\ \ 1\ls i_1<i_2<\cdots <i_k\ls n,\ A_{i_r}^*=A_{i_r}$ or $A_{i_r}^c$ .
       
(3) $P\left(A_{i_1}^*\bigcap A_{i_2}^*\bigcap\cdots\bigcap A_{i_n}^*\right)=P\left(A_{i_1}^* \right)P\left(A_{i_2}^* \right)\cdots P\left(A_{i_n}^* \right),\ \forall A_i^*=A_i,\ A_i^c,$

$\quad\ \  i=1,2,\ \cdots,n$.

\end{theorem}

\begin{proof}
\\[4cm]\vspace{0.01cm}
\end{proof}

\begin{definition}{Independent Set}{Independent Set}
Let $(\Omega,\ma, P)$ be a probability space, and $A_i\in\ma,\ \forall i\in I$, where $I$ is an index set, then $\{A_i: i\in I\}$ is said to be independent if any finite subset of $\{A_i: i\in I\}$ is independent; otherwise, it is said to be dependent.
\end{definition}

\begin{corollary}{Independence under Finite Union}{Independence under Finite Union}
Suppose $(\Omega,\ma, P)$ is a probability space, and suppose $A_1,A_2,\ \cdots,A_n\in\ma$ are independent. Then
$$\begin{aligned}
&P\left[\left(A_{i_1}^* \bigcap A_{i_2}^*\bigcap\cdots\bigcap A_{i_k}^*\right)\bigcap\left(A_{j_1}^* \bigcap A_{j_2}^*\bigcap\cdots\bigcap A_{j_l}^* \right)\right] \\
=&P\left(A_{i_1}^*\bigcap A_{i_2}^*\bigcap\cdots\bigcap A_{i_k}^*\right)\cdot P\left(A_{j_1}^*\bigcap A_{j_2}^*\bigcap\cdots\bigcap A_{j_l}^*\right)
\end{aligned}$$
 $\forall k,l\gs1,\ k+l\ls n,\ 1\ls i_1,i_2,\ \cdots,i_k,j_1,j_2,\ \cdots,j_l\ls n$ distinct, and $A_{i_r}^*=A_{i_r}$ or $A_{i_r}^c,\ r=1,2,\ \cdots,k,\ A_{j_r}^*=A_{j_r}$ or $A_{j_r}^c,\ r=1,2,\ \cdots,l.$

In particular, if $P\left(A_{j_1}^* \bigcap A_{j_2}^* \bigcap\cdots\bigcap A_{j_l}^*\right)>0$, for some $1\ls l\ls n-1,\ 1\ls j_1,\ \cdots,j_l\ls n$ distinct, and $A_{j_r}^*=A_{j_r}$ or $A_{j_r}^c,\ r=1,2,\ \cdots,l$. Then 
$$\begin{aligned}
&P\left[\left(A_{i_1}^*\bigcap A_{i_2}^*\bigcap\cdots\bigcap A_{i_k}^*\right)\bigg|\left(A_{j_1}^*\bigcap A_{j_2}^*\bigcap\cdots\bigcap A_{j_l}^*\right)\right]\\
=&P\left(A_{i_1}^*\bigcap A_{i_2}^*\bigcap\cdots\bigcap A_{i_k}^*\right)\end{aligned}$$
 for all $1\ls k\ls n-l.\ i_1,i_2,\ \cdots,i_k\in\{1,2,\ \cdots,n\}\backslash\{j_1,j_2,\ \cdots,j_l \}$ distinct, and $A_{i_r}^*=A_{i_r }$ or $A_{i_r}^c, \ r=1,2,\ \cdots,k$.
\end{corollary}

\begin{proof}
\\[4cm]\vspace{0.01cm}
\end{proof}

% chapter 4
\chapter{Distribution Functions and Discrete Random Variables}

\section{Random Variables}

\begin{definition}{Measurable Space}{Measurable Space}
A measurable space is an ordered pair $(\Omega,\ma)$ consisting of a sample space $\Omega$ and a $\sigma$-algebra $\ma$ of subsets of $\Omega$.
\end{definition}

\begin{definition}{Measurable Function}{Measurable Function}
Let $(\Omega_1,\ma_1)$, $(\Omega_2,\ma_2)$ be measurable spaces. A function from $\Omega_1$ to $\Omega_2$ is called a measurable function from $(\Omega_1,\ma_1)$ to  $(\Omega_2,\ma_2)$ if $f^{-1}(B)\in \ma_1,\forall B\in\ma_2$ , where $f^{-1} (B)=\{x\in\Omega:f(x)\in B\}$ is the 
pre-image of $B$ under $f$.
\end{definition}

\begin{lemma}{$\sigma$-algebra under Function}{sigma-algebra under Function}
Suppose $f$ is a function from $\Omega_1$ to $\Omega_2$.\\
(1) If $\ma_2$ is a $\sigma$-algebra of subsets of  $\Omega_2$, then $\ma_1=\{f^{-1} (B):B\in\ma _2\}$ is a $\sigma$-algebra of subsets of $\Omega_1$.\\
(2) If $\ma_1$ is a $\sigma$-algebra of subsets of $\Omega_1$, then $\ma_2=\{B\in\Omega_2:f^{-1} (B)\in\ma_1\}$ is a $\sigma$-algebra of subsets of $\Omega_2$.
\end{lemma}

\begin{proof}
\\[4cm]\vspace{0.01cm}
\end{proof}

\begin{theorem}{$\sigma$-algebra Including Subset}{sigma-algebra Including Subset}
Suppose $(\Omega_1,\ma_1)$ is a measurable space and $f$ is a function from $\Omega_1$ to $\Omega_2$. If $\mc\subseteq\{B\subseteq\Omega_2: f^{-1} (B)\in\ma_1\}$, then $\sigma(\mc)\subseteq\{B\subseteq\Omega_2: f^{-1} (B)\in\ma_1\}$.
\end{theorem}

\begin{proof}
\\[4cm]\vspace{0.01cm}
\end{proof}

\begin{corollary}{A Kind of Measurable Function}{A Kind of Measurable Function}
Suppose $(\Omega_1,\ma_1)$, $(\Omega_2,\ma_2)$ are measurable spaces, and $f$ is a function from $\Omega_1$ to $\Omega_2$. Suppose $\mc\subseteq\{B\subseteq\Omega_2: f^{-1} (B)\in\ma_1\}$ and $\sigma(\mc)\supseteq\ma_2$. Then $f$ is a measurable function from $(\Omega_1,\ma_1)$ to $(\Omega_2,\ma_2)$.
\end{corollary}

\begin{proof}
\\[4cm]\vspace{0.01cm}
\end{proof}

\begin{theorem}{Composite Measurable Function}{Composite Measurable Function}
Suppose $(\Omega_1,\ma_1)$, $(\Omega_2,\ma_2)$, $(\Omega_3,\ma_3)$ are measurable spaces, $f$ is a measurable function from $(\Omega_1,\ma_1)$ to $(\Omega_2,\ma_2)$, and $g$ is a measurable function from $(\Omega_2,\ma_2)$ to $(\Omega_3,\ma_3)$. Then $g\circ f$ is a measurable function from  $(\Omega_1,\ma_1)$ to $(\Omega_3,\ma_3)$.
\end{theorem}

\begin{proof}
\\[4cm]\vspace{0.01cm}
\end{proof}

\begin{definition}{Open Set}{Open Set}
A set $A$ in $\mr^n$ is called an open set in $\mr^n$ if for all $\mx\in A, \exists r>0\ \Rightarrow\mb_{\mx}(r)\subseteq A$, where $\mb_{\mx}(r)=\left\{\my\in\mr^n:\Arrowvert\my-\mx\Arrowvert<r\right\}$.
\end{definition}

\begin{definition}{Borel $\sigma$-algebra}{Borel algebra}
The $\sigma$-algebra generated by the set of all open sets in $\mr^n$ is called the Borel $\sigma$-algebra of subsets of $\mr^n$ and is denoted by $\mb_{\mr^n}$. We call a set in $\mb_{\mr^n}$ a Borel set in $\mr^n$.
\end{definition}

\begin{theorem}{Measurable Function from Continuity}{Measurable Function from Continuity}
Suppose $f$ is a continuous function from $\mr^n$ to $\mr^m$. Then $f$ is a measurable function from $(\mr^n,\mb_{\mr^n})$ to $(\mr^m,\mb_{\mr^m})$.
\end{theorem}

\begin{proof}
\\[4cm]\vspace{0.01cm}
\end{proof}

\begin{definition}{Cell}{Cell}
A cell in $\mr$ is a f\/inite interval of the form $(a,b),[a,b),(a,b]$, or $[a,b]$ for some $a\ls b$. A cell $I$ in $\mr^n$, where $n\gs1$, is a Cartesian product of $n$ cells $I_1,I_2,\ \cdots,I_n$ in $\mr$, i.e., $I=I_1\times I_2\times\cdots\times I_n$.
\end{definition}

\begin{definition}{Open Cube}{Open Cube}
Let $\mx\in\mr^n$, $l>0$, and $I_i=\left(x_i-\frac{l}{2},x_i+\frac{l}{2}\right)$,$\forall1\ls i\ls n$. The open cube $C_\mx(l)$ in $\mr^n$ with center $\mx$ and side length $l$ is def\/ined as the open cell $I_1\times I_2\times\cdots\times I_n$  in $\mr^n$.
\end{definition}

\begin{theorem}{Set from Cells}{Set from Cells}
Every open set in $\mr^n$ is a countable union of open cells in $\mr^n$.
\end{theorem}

\begin{proof}
\\[4cm]\vspace{0.01cm}
\end{proof}

\begin{theorem}{Measurable Function on Open Cells}{Measurable Function on Open Cells}
Suppose $(\Omega,\ma)$ is a measurable space and $f$ is a function from $\Omega$ to $\mr^n$. Suppose that $f^{-1}(B)\in\ma$ for all open cells in $\mr^n$. Then $f$ is a measurable function from $(\Omega,\ma)$ to $(\mr^n,\mb_{\mr^n})$.
\end{theorem}

\begin{proof}
\\[4cm]\vspace{0.01cm}
\end{proof}

\begin{theorem}{Components of Measurable Function}{Components of Measurable Function}
Suppose $(\Omega,\ma)$ is a measurable space,  $f=(f_1,f_2,\ \cdots,f_n )$  is a function from $\Omega$ to $\mr^n$. Then $f$ is a measurable function from  $(\Omega,\ma)$ to $(\mr^n,\mb_{\mr^n})\Leftrightarrow f_1,f_2,\ \cdots,f_n$ are measurable functions from $(\Omega,\ma)$ to $(\mr^n,\mb_{\mr^n})$.
\end{theorem}

\begin{proof}
\\[4cm]\vspace{0.01cm}
\end{proof}

\begin{theorem}{Elementary Operation of Measurable Function}{Elementary Operation of Measurable Function}
Suppose $f$ and $g$ are measurable functions from $(\Omega,\ma)$ to $(\mr,\mb_{\mr})$, and $c\in\mr$. Then $cf,\ f^n,\ |f|,\ f+g,\ f\circ g$ are measurable functions from $(\Omega,\ma)$ to $(\mr,\mb_{\mr})$.
\end{theorem}

\begin{proof}
\\[4cm]\vspace{0.01cm}
\end{proof}

\begin{theorem}{Limit of Measurable Functions}{Limit of Measurable Functions}
Suppose that $f_1,f_2,\cdots$ are measurable functions from $(\Omega,\ma)$ to $(\mr,\mb_{\mr})$ and $f_n\to f$ as $n\to\infty$, where $f$ is a function from $\Omega$ to $\mr$. Then $f$ is also a measurable function from $(\Omega,\ma)$ to $(\mr,\mb_{\mr})$.
\end{theorem}

\begin{proof}
\\[4cm]\vspace{0.01cm}
\end{proof}

\begin{theorem}{Equivalence of Nine Types of Set}{Equivalence of Nine Types of Set}
Suppose $(\Omega,\ma)$ is a measurable space and $f$ is a function from $\Omega$ to $\mr$. Let $\mc_1$ be the set of all open sets in $\mr$, 
$$
\begin{array}{ll}
\mc_2=\{(a,b),a,b\in\mr,a\ls b\},    &\mc_3=\{(a,b],a,b\in\mr,a\ls b\},\\
\mc_4=\{[a,b],a,b\in\mr,a\ls b\},     &\mc_5=\{[a,b),a,b\in\mr,a\ls b\},\\
\mc_6=\{[a,+\infty),a\in\mr\},        &\mc_7=\{(a,+\infty),a\in\mr\},\\
\mc_8=\{(-\infty,a],a\in\mr\},        &\mc_9=\{(-\infty,a),a\in\mr\}.\\
\end{array}
$$
Then $f$ is a measurable function from $(\Omega,\ma)$ to $(\mr,\mb_{\mr})$
if $f^{-1} (B)\in\ma,\ \forall B\subseteq\mc_i$ for any $i=1,2,\ \cdots,9$.
\end{theorem}

\begin{proof}
\\[4cm]\vspace{0.01cm}
\end{proof}

\begin{theorem}{Induced Probability Space under Function}{Induced Probability Space under Function}
Suppose $f$ is a measurable function from $(\Omega_1,\ma_1)$ to $(\Omega_2,\ma_2)$. Suppose $P$ is a probability measure on $\ma_1$.
Then the function $P_f$ on $\ma_2$ given by $$P_f (B)=P[f^{-1} (B)],\  \forall B\in\ma_2$$ is a probability measure.\\
We call $(\Omega_2,\ma_2,P_f)$ the probability space induced from $(\Omega_1,\ma_1,P)$ under $f$.
\end{theorem}

\begin{proof}
\\[4cm]\vspace{0.01cm}
\end{proof}

\begin{remark}{Conventional Denotation}{Conventional Denotation}
(1) The set $f^{-1} (B)$ is conventionally denoted as $f\in B$. Therefore $P_f (B)=P[f^{-1} (B)]=P(f\in B),\ \forall B\in\ma_2$.\\
(2) If $B\in\ma_2$, then $f^{-1} (B)=f^{-1}[B\cap f(\Omega_1)]$, and hence
$P_f (B)=P(f\in B)=P[f^{-1} (B)]=P[f^{-1} (B\cap f(\Omega_1)]=P[ f\in (B\cap f(\Omega_1))]=P_f (B\cap f(\Omega_1)).$
\end{remark}

\begin{proof}
\\[4cm]\vspace{0.01cm}
\end{proof}

\begin{definition}{Random Variable}{Random Variable}
Let $(\Omega,\ma, P)$ be a probability space. A measurable function $X$ from $(\Omega,\ma)$ to $(\mr,\mb_{\mr})$ is called a random variable (r.v.) of the probability space $(\Omega,\ma, P)$.\\
A measurable function $\mX=(X_1,X_2,\ \cdots,X_n)$ from $(\Omega,\ma)$ to $(\mr^n,\mb_{\mr^n})$ is called a random vector (r.vect.) of the probability space $(\Omega,\ma, P)$.
\end{definition}

\begin{remark}{Conventional Denotation of Random Variable}{Conventional Denotation of Random Variable}
If $X$ is a r.v. of the probability space $(\Omega,\ma, P)$, then $P_X (B)=P[X^{-1} (B)]=P(X\in B)=P[\{w\in\Omega:X(w)\in B\}],\ \forall B\in\mb_{\mr}$.
\end{remark}

\begin{proof}
\\[4cm]\vspace{0.01cm}
\end{proof}

\begin{theorem}{Additivity of Countable Points }{Additivity of Countable Points }
Suppose $\mX$ is a r.vect. of a probability space $(\Omega,\ma, P)$, and $B$ is a ``countable'' subset of $\mr^n$, then $B\in\mb_{\mr}$, and
$$P_\mX (B)=P(\mX\in B)=\sum\limits_{\mx\in B}P(\mX=\mx)=\sum\limits_{\mx\in B}P_\mX(\{\mx\}).$$
\end{theorem}

\begin{proof}
\\[4cm]\vspace{0.01cm}
\end{proof}

\section{Distribution Functions}

\begin{definition}{Cumulative Distribution Function}{Cumulative Distribution Function}
Let $X$ be a r.v. of a probability space $(\Omega,\ma, P)$. The cumulative distribution function (c.d.f) $F_X$ of the r.v. $X$ is a function from $\mr$ to $[0,1]$, given by
$$F_X (t)=P_X ((-\infty,t])=P(X\in(-\infty,t])=P(X\ls t),\ \forall t\in\mr.$$
\vspace{0.01cm}
\end{definition}

\begin{theorem}{Properties of C.D.F}{Properties of C.D.F}
Suppose $X$ is a r.v. of a probability space $(\Omega,\ma, P)$.\\
(1) $F_X$ is increasing.\\
(2) $F_X (+\infty)\triangleq \limtpi F_X (t)=1$.\\
(3) $F_X (-\infty)\triangleq \limtni F_X (t)=0$.\\
(4) $F_X (t+)=P(X\ls t)=F_X (t)$. $F_X (t)$ is right continuous.\\
(5) $F_X (t-)=P(X<t)$.
\end{theorem}

\begin{proof}
\\[4cm]\vspace{0.01cm}
\end{proof}

\begin{corollary}{More Properties of C.D.F}{More Properties of C.D.F}
Suppose $X$ is a r.v. of a probability space $(\Omega,\ma, P)$.\\
(1) $P(X\ls a)=F_X (a),\ P(X>a)=1-F_X (a)$.\\
(2) $P(X<a)=F_X (a-),\ P(X\gs a)=1-F_X (a-)$.\\
(3) $P(X=a)=F_X (a)-F_X (a-)$.\\
(4) $P(a<X\ls b)=F_X (b)-F_X (a),\quad P(a\ls X\ls b)=F_X (b)-F_X (a-),$\\
\quad$P(a<X<b)=F_X (b-)-F_X (a),\quad P(a\ls X<b)=F_X (b-)-F_X (a-).$
\end{corollary}

\begin{proof}
\\[4cm]\vspace{0.01cm}
\end{proof}

\begin{theorem}{Existence of C.D.F }{Existence of C.D.F }
Suppose $F: \mr\to[0,1]$ is a function s.t. $F$ is increasing and right continuous, 
$$\limtpi F_X (t)=1,\qquad\limtni F_X (t)=0.$$
Then there exists a r.v. $X$ of some probability space $(\Omega,\ma, P)$,
s.t. the c.d.f. $F_X$ of $X$ is equal to $F$. We call such function a c.d.f.
\end{theorem}

\begin{proof}
\\[4cm]\vspace{0.01cm}
\end{proof}

\section{Discrete Random Variables}

\begin{definition}{Discrete R.V.}{Discrete R.V.}
A r.v. of a probability space $(\Omega,\ma, P)$ is called a discrete r.v. if $X(\Omega)=\{X(w):w\in\Omega\}$ is countable.
\end{definition}

\begin{definition}{Probability Mass Function}{Probability Mass Function}
Let $X$ be a discrete r.v. of a probability space $(\Omega,\ma, P)$ s.t. $X(\Omega)=\{x_1,x_2,\cdots\}$. The probability mass function (p.m.f) $p_X: \mr\to[0,1]$ of $X$ is a function from $\mr$ to $[0,1]$ given by $p_X (x)=P_X (\{X=x\})=P(X=x),\ \forall x\in\mr$.
\end{definition}

\begin{theorem}{Properties of P.M.F}{Properties of P.M.F}
Suppose $X$ is a discrete r.v. of a probability space $(\Omega,\ma, P)$. Then,\\
(1) $p_X (x)\gs0,\ \forall x\in X(\Omega)$.\\
(2) $p_X (x)=0,\ \forall x\in\mr\setminus X(\Omega)$.\\
(3) $\dis\sum\limits_{x\in X(\Omega)}p_X (x) =1$.\\
Therefore if $X(\Omega)=\{x_1,x_2,\cdots\}$, then,\\
(1) $p_X (x_i )\gs0,\ \forall i=1,2,\cdots$.\\
(2) $p_X (x)=0,\forall x\in\mr\setminus\{x_1,x_2,\cdots\}$.\\
(3) $\dis\sum\limits_{i=1}^\infty p_X (x_i ) =1$.\\
\end{theorem}

\begin{proof}
\\[4cm]\vspace{0.01cm}
\end{proof}

\begin{theorem}{Existence of P.M.F}{Existence of P.M.F}
Suppose $p: \mr\to[0,1]$ is a function s.t.\\
(1) $p(x_i )\gs0\ \forall i=1,2,\cdots$\\
(2) $p(x)=0,\ \forall x\in\mr\setminus\{x_1,x_2,\cdots\}$.\\
(3) $\dis\sum\limits_{i=1}^\infty p_X (x_i ) =1$.\\
for some distinct $x_1,x_2,\cdots\in\mr$.\\
Then there exists a discrete r.v. $X$ of some probability space $(\Omega,\ma, P)$ s.t. the p.m.f. $p_X$ of $X$ is equal to $p$. We call such a function a p.m.f.
\end{theorem}

\begin{proof}
\\[4cm]\vspace{0.01cm}
\end{proof}

\begin{theorem}{Step Distribution Function for Discrete R.V.}{Step Distribution Function for Discrete R.V.}
Suppose $X$ is a discrete r.v. of a probability space $(\Omega,\ma, P)$ s.t. $X(\Omega)=\{x_1,x_2\cdots\}$, where $x_1<x_2<\cdots$. Then the distribution function of $X$ is a step function given by
$$
F_X (t)=
\begin{cases}
\qquad0,\qquad \text{if} \ t<x_1\\
\sumin p_X (x_i ),\ \text{if} \ x_n\ls t\ls x_{n+1}, \ n=1,2,\cdots
\end{cases}
=\sumin p_X (x_i )U(t-x_i),
$$
where
$$
U(t)=\begin{cases}
1,\ \text{if}\ t\gs0\\
0,\ \text{o.w.}
\end{cases}
$$
\end{theorem}

\begin{proof}
\\[4cm]\vspace{0.01cm}
\end{proof}

\section{Expectations of Discrete Random Variables}

\begin{definition}{Expectation}{Expectation}
Let $X$ be a discrete r.v. of a probability space $(\Omega,\ma, P)$. The expectation (or expected value, or mean) of $X$ is given by 
$$
E[X]=\sum\limits_{x\in X(\Omega)}x\cdot P(X=x) =\sum\limits_{x\in X(\Omega)}x\cdot p_X (x)
$$
if the sum converges absolutely. And if the sum diverges to $\pm\infty$, $E[X]=\pm\infty$.
\end{definition}

\begin{remark}{Explanations of Expectation}{Explanations of Expectation}
(1) The expectation $\dis E[X]=\sum\limits_{x\in X(\Omega)}x\cdot p_X (x)$ is the weighted average of $\{x:x\in X(\Omega)\}$ with weights $\{P(X=x): x\in X(\Omega)\}$.\\
(2) The expectation $\dis E[X]=\sum\limits_{x\in X(\Omega)}x\cdot p_X (x)$ is the center of gravity of $\{P(X=x): x\in X(\Omega)\}$.
\end{remark}

\begin{proof}
\\[4cm]\vspace{0.01cm}
\end{proof}

\begin{theorem}{Expectation of Constant}{Expectation of Constant}
Suppose $X$ is a discrete r.v. of a probability space $(\Omega,\ma, P)$ s.t. $X$ is a constant with probability 1, i.e., $P(X=c)=1$ for some $c\in\mr$. Then $c\in X(\Omega),\ P(X=x)=0,\ \forall x\in X(\Omega)\setminus\{c\}$, and $E[X]=c$. In particular, if $X$ is a constant r.v. of $(\Omega,\ma, P)$, i.e., $X(w)=c,\ \forall w\in\Omega$, for some $c\in\mr$, then $E[X]=c$.
\end{theorem}

\begin{proof}
\\[4cm]\vspace{0.01cm}
\end{proof}

\begin{theorem}{Composition of Function and R.V.}{Composition of Function and R.V.}
Suppose $X$ is a discrete r.v. of a probability space $(\Omega,\ma, P)$ and $g$ be a measurable function from $(\mr,\mb_{\mr})$ to $(\mr,\mb_{\mr})$. Then $g(X)\triangleq g\circ X$ is a discrete r.v. of $(\Omega,\ma, P)$ and
$$E[g(X)]=\sum\limits_{x\in X(\Omega)}g(x)P(X=x).$$
\end{theorem}

\begin{proof}
\\[4cm]\vspace{0.01cm}
\end{proof}

\begin{corollary}{Linearity of Expectation}{Linearity of Expectation}
Suppose $X$ is a discrete r.v. of a probability space $(\Omega,\ma, P)$,
$g_1,g_2,\ \cdots,g_n$ are measurable functions from $(\mr,\mb_{\mr})$ to $(\mr,\mb_{\mr})$, and $\alpha_1,\alpha_2,\ \cdots,\alpha_n\in\mr$, Then
$$
\sumin\alpha_i  g_i (X)
$$
is a discrete r.v. of $(\Omega,\ma, P)$ and 
$$
E\left[\sumin\alpha_i  g_i (X)\right]=\sumin\alpha_i  E[g_i (X)].
$$
\end{corollary}

\begin{proof}
\\[4cm]\vspace{0.01cm}
\end{proof}

\section{Variances and Moments of Discrete Random Variables}

\begin{definition}{Variance and Standard Deviation}{Variance and Standard Deviation}
Let $X$ be a discrete r.v. of a probability space $(\Omega,\ma, P)$ and suppose $E[X]$ exists. The variance of $X$ is given by $$Var(X)=E[(X-E[X])^2],$$ and the standard deviation of $X$ is given by $\sigma_X=\sqrt{Var(X)}$.
\end{definition}

\begin{remark}{Explanation about Variance}{Explanation about Variance}
The variance of a discrete r.v. measures the dispersion (or spread) of its probability masses about its expectation (center of gravity of its probability masses).
\end{remark}

\begin{proof}
\\[4cm]\vspace{0.01cm}
\end{proof}

\begin{theorem}{Calculating Variance}{Calculating Variance}
Suppose $X$ is a discrete r.v. of a probability space $(\Omega,\ma, P)$ and suppose $E[X]$ exists. Then $Var(X)=E[X^2 ]-(E[X])^2$.
\end{theorem}

\begin{proof}
\\[4cm]\vspace{0.01cm}
\end{proof}

\begin{theorem}{Minimum Distance with Expectation}{Minimum Distance with Expectation}
Suppose $X$ is a discrete r.v. of a probability space $(\Omega,\ma, P)$ and suppose $E[X]$ exists. If $E[X^2 ]<+\infty$, then $Var(X)=\min\limits_{a\in\mr}E[(X-a)^2]$.
\end{theorem}

\begin{proof}
\\[4cm]\vspace{0.01cm}
\end{proof}

\begin{theorem}{With Probability 1}{With Probability 1}
Suppose $X$ is a discrete r.v. of a probability space $(\Omega,\ma, P)$.\\
(1) $E[X^2 ]\gs0$, ``='' holds $\Leftrightarrow X=0$ with probability 1, i.e., $P(X=0)=1$.\\
(2) If $E[X]$ exists, then $Var(X)\gs0$, ``='' holds $\Leftrightarrow X=E[X]$ with probability 1, i.e., $P(X=E[X])=1$.
\end{theorem}

\begin{proof}
\\[4cm]\vspace{0.01cm}
\end{proof}

\begin{theorem}{Calculating Linear Combination}{Calculating Linear Combination}
Suppose $X$ is a discrete r.v. of a probability space $(\Omega,\ma, P)$ and suppose $E[X]$ exists. Then $Var(aX+b)=a^2 Var(X)$ and $\sigma_{aX+b}=|a| \sigma_X,\ \forall a,b\in\mr$.
\end{theorem}

\begin{proof}
\\[4cm]\vspace{0.01cm}
\end{proof}

\begin{definition}{Moment and Absolute Moment}{Moment and Absolute Moment}
Let $X$ be a discrete r.v. of a probability space $(\Omega,\ma, P)$, and $r,c\in\mr$.
$$\left\{
\begin{array}{l}
\text{The }r^{th}\ \text{moment of } X\ \text{is given by } E[X^r ]   \\                              
\text{The }r^{th}\ \text{central moment of } X\ \text{is given by } E[(X-E[X])^r]\\
\text{The }r^{th}\ \text{moment of } c\ \text{is given by } E[(X-c)^r ]      \\     
\text{The }r^{th}\ \text{absolute moment of } X\ \text{is given by } E[|X|^r ]\\
\text{The }r^{th}\ \text{absolute central moment of } X \text{is given by } E[|X-E[X] |^r]  \\
\text{The } r^{th}\ \text{absolute moment of } c\ \text{is given by } E[|X-c|^r ] 
\end{array}
\right.$$
If the respective sum converges absolutely.
\end{definition}

\begin{theorem}{Existence of Lower Order Moment}{Existence of Lower Order Moment}
Suppose $X$ is a discrete r.v. of a probability space $(\Omega,\ma, P)$ and suppose $0<r<s$. If $E[|X|^s]$ exists, then $E[|X|^r]$ exists. That is, the existence of a higher order moment of $X$ guarantees the existence of a lower order moment of $X$.
\end{theorem}

\begin{proof}
\\[4cm]\vspace{0.01cm}
\end{proof}

\section{Standardized Random Variables}

\begin{definition}{Standardized R.V.}{Standardized R.V.}
Let $X$ be a discrete r.v. of a probability space $(\Omega,\ma, P)$. If $Var(X)$ exists and $Var(X)\neq0$, then the standardized r.v. of $X$ is given by 
$$X^*=\frac{X-E[X]}{\sigma_X}$$
i.e., $X^*$ is the number of standard deviation units by which $X$ dif\/fers from $E[X]$.
\end{definition}

\begin{theorem}{Expectation and Variance of Standardized R.V.}{Expectation and Variance of Standardized R.V.}
Suppose $X$ is a discrete r.v. of a probability space $(\Omega,\ma, P)$ and $Var(X)$ exists, $Var(X)\neq0$. Then $E[X^* ]=0$ and $Var(X^* )=1$.
\end{theorem}

\begin{proof}
\\[4cm]\vspace{0.01cm}
\end{proof}

\begin{theorem}{Independence of Units}{Independence of Units}
Suppose $X$ is a discrete r.v. of a probability space $(\Omega,\ma, P)$ and $Var(X)$ exists, $Var(X)\neq0$. Then the standardized r.v. of $X$ is independent of the units in which $X$ is measured.
\end{theorem}

\begin{proof}
\\[4cm]\vspace{0.01cm}
\end{proof}

\begin{remark}{Standardization for Comparison}{Standardization for Comparison}
Standardization can be useful when comparing r.v.'s with dif\/ferent distributions.
\end{remark}

\begin{proof}
\\[4cm]\vspace{0.01cm}
\end{proof}

\chapter{Special Discrete Distributions}
\section{Bernoulli R.V.'s and Binomial R.V.'s}

\begin{definition}{Bernoulli Trial}{Bernoulli Trial}
A Bernoulli trial is an experiment that has only two outcomes, say success and failure, so that its sample space is given by $\Omega=\{s,f\}$.
\end{definition}

Let $X$ be the number of success in a Bernoulli trial.
$$
p_X (i)=
\begin{cases}
1-p,\quad\text{if } i=0\\
p,\quad\text{if }i=1\\
0,\quad\text{o.w.}\\
\end{cases}
$$
where $p=P(X=1)=P(\{s\})$ is the probability of success.

\begin{definition}{Bernoulli R.V. }{Bernoulli R.V. }
A discrete r.v. $X$ of a probability space $(\Omega,\ma, P)$ is called a Bernoulli r.v. with parameter $p$ where $0<p<1$, denoted $X\sim$ Bernoulli$(p)$, if its p.m.f is given by
$$
p_X (i)=
\begin{cases}
1-p,\quad\text{if } i=0\\
p,\quad\text{if }i=1\\
0,\quad\text{o.w.}\\
\end{cases}
$$
Such a p.m.f is called a Bernoulli p.m.f with parameter $p$.
\end{definition}

\begin{theorem}{Expectation and Variance of Bernoulli R.V.}{Expectation and Variance of Bernoulli R.V.}
Suppose $X\sim$ Bernoulli$(p)$, where $0<p<1$. Then $$E[X]=p,\qquad Var(X)=p(1-p).$$
\vspace{0.01cm}
\end{theorem}

\begin{proof}
\\[4cm]\vspace{0.01cm}
\end{proof}

Consider an experiment in which $n$ independent Bernoulli trials with the same probability of success, say $p$, are performed. The sample space of the experiment is $\Omega=\{(w_1,w_2,\ \cdots,w_n ): w_i=s \text{ or } f,i=1,2,\ \cdots,n\} $ and $P(\{(w_1,w_2,\ \cdots,w_n )\})=p^i (1-p)^{n-i}$, where $i=|\{1\ls j\ls n: w_j=s\}|.$

Let $X$ be the number of successes in the $n$ Bernoulli trials.
$$
p_X (i)=
\left\{\begin{aligned}
&\binom n i p^i(1-p)^{n-i},\ \text{if}\ i=0,1,2,\ \cdots,n\\
&\qquad0,\qquad\ \text{o.w.}    
\end{aligned}
\right.
$$

\begin{definition}{Binomial R.V.}{Binomial R.V.}
A discrete r.v. $X$ of a probability space $(\Omega,\ma, P)$ is called a binomial r.v. with parameter $n$ and $p$ where $n\gs 1$ and $0<p<1$, denoted $X\sim$ binomial$(n,p)$, if its p.m.f is given by
$$
p_X (i)=
\left\{\begin{aligned}
&\binom n i p^i(1-p)^{n-i},\ \text{if}\ i=0,1,2,\ \cdots,n\\
&\qquad0,\qquad\ \text{o.w.}    
\end{aligned}
\right.
$$
Such a p.m.f is called a binomial p.m.f with parameter $n$ and $p$.
\end{definition}

\begin{remark}{Bernoulli R.V. from Binomial R.V.}{Bernoulli R.V. from Binomial R.V.}
(1) A Bernoulli r.v. with parameter $p$ is a binomial r.v. with parameter 1 and $p$.\\
(2)
$$
\sumin p_X(i)=\sumin\binom n i p^i(1-p)^{n-i}=[p+(1-p)]^n=1
$$
Thus $p_X (\cdot)$ is a p.m.f.
\end{remark}

\begin{theorem}{Expectation and Variance of Binomial R.V.}{Expectation and Variance of Binomial R.V.}
Suppose $X\sim$ binomial$(n,p)$, where $n\gs1$ and $0<p<1$. Then 
$$E[X]=np,\qquad Var(X)=np(1-p).$$
\vspace{0.01cm}
\end{theorem}

\begin{proof}
\\[4cm]\vspace{0.01cm}
\end{proof}

\begin{theorem}{Maximum Point of Binomial Probability}{Maximum Point of Binomial Probability}
Suppose $X\sim$ binomial$(n,p)$, where $n\gs1$ and $0<p<1$. Then
$$
arg\max\limits_{0\ls i\ls n} p_X (i)=
\begin{cases}
(n+1)p-1 \text{  or  } (n+1)p,\ \text{if }  (n+1)p\in\mz\\
\qquad\lfloor(n+1)p\rfloor\qquad, \ \text{if }  (n+1)p\notin\mz
\end{cases}
$$
\end{theorem}

\begin{proof}
\\[4cm]\vspace{0.01cm}
\end{proof}

\section{Poisson R.V.'s}

If $X\sim$ binomial$(n,p)$, then $p_X (i)=\binom n i p^i (1-p)^{n-i}$ is dif\/f\/icult to calculate if $n$ is large. A recursive relation: 
$$p_X (0)=(1-p)^n, \ p_X (i)=\frac{n-i+1}{i(1-p)} \cdot p_X (i-1),\ \forall i\gs1.$$
An approximation for large $n$, small $p$, and moderate $np$, say $np=\lambda$ for some constant $\lambda$:
$$
\begin{aligned}
p_X (i)&=\binom n i p^i (1-p)^{n-i}=\frac{n(n-1)\cdots(n-i+1)}{i!}\left(\frac{\lambda}{n}\right)^i \left(1-\frac{\lambda}{n} \right)^{n-i}\\
&=\frac{n(n-1)\cdots(n-i+1)}{n^i}\cdot\frac{1}{\left(1-\frac{\lambda}{n}\right)^i }\cdot\frac{\lambda^i}{i!}\cdot\left(1-\frac{\lambda}{n}\right)^n  \xrightarrow{n\to\infty}e^{-\lambda} \frac{\lambda^i}{i!}.
\end{aligned}
$$

\begin{definition}{Poisson R.V.}{Poisson R.V.}
A discrete r.v. of a probability space $(\Omega,\ma, P)$ is called a Poisson r.v. with parameter $\lambda$ where $0<\lambda<1$, denoted $X\sim$ Poisson$(\lambda)$, if its p.m.f is given by
$$
p_X (i)=
\left\{\begin{aligned}
&e^{-\lambda}\cdot \frac{\lambda^i}{i!}, \ i=0,1,2,\cdots\\
&0,\quad\text{o.w.}
\end{aligned}           
\right.
$$
Such a p.m.f is called a Poisson p.m.f with parameter $\lambda$.
\end{definition}

\begin{remark}{Poisson R.V. from Binomial R.V.}{Poisson R.V. from Binomial R.V.}
(1) A Poisson r.v. with parameter $\lambda$ is an approximation of a binomial p.m.f. with parameters $n$ and $p$ such that $n$ is large and $p$ is small, and $np=\lambda$.\\
(2)
$$
\sumiz p_X (i)=\sumiz e^{-\lambda} \cdot\frac{\lambda^i}{i!}=e^{-\lambda}\cdot e^{\lambda}=1 
$$
thus $p_X (\cdot)$ is a p.m.f.
\end{remark}

\begin{theorem}{Expectation and Variance of Poisson R.V.}{Expectation and Variance of Poisson R.V.}
Suppose $X\sim$ Poisson$(\lambda)$, where $\lambda>0$. Then $E[X]=\lambda$ and $Var(X)=\lambda.$
\end{theorem}

\begin{proof}
\\[4cm]\vspace{0.01cm}
\end{proof}

\begin{theorem}{Maximum Point of Poisson Probability}{Maximum Point of Poisson Probability}
Suppose $X\sim$ Poisson$(\lambda)$, where $\lambda>0$. Then
$$
arg\max\limits_{i\gs0}p_X(i)=
\begin{cases}
\lambda-1 \text{ or } \lambda,\ \text{if}\ \lambda\in\mz\\
\lfloor\lambda\rfloor,\quad \text{if}\ \lambda\notin\mz
\end{cases}
$$
\end{theorem}

\begin{proof}
\\[4cm]\vspace{0.01cm}
\end{proof}

\section{Geometric R.V.'s, Negative Binomial R.V.'s and Hypergeometric R.V.'s}

Consider an experiment in which independent Bernoulli trials with the same probability of success, say $p$, are performed until the f\/irst success occurs. The sample space of the experiment is $\Omega=\{s,fs,ffs,\cdots\}$.

Let $X$ be the number of Bernoulli trials until the f\/irst success occurs,
$$
p_X (i)=
\begin{cases}
(1-p)^{i-1}\cdot p ,\   i=0,1,2\cdots\\
0,\quad\text{o.w.}
\end{cases}
$$

\begin{definition}{Geometric R.V.}{Geometric R.V.}
A discrete r.v. of a probability space $(\Omega,\ma, P)$ is called a geometric r.v. with parameter $p$ where $0<p<1$, denoted $X\sim$ geometric$(p)$, if its p.m.f is given by
$$
p_X (i)=
\begin{cases}
(1-p)^{i-1}\cdot p ,\   i=0,1,2\cdots\\
0,\quad\text{o.w.}
\end{cases}           
$$
Such a p.m.f is called a geometric p.m.f with parameter $p$.
\end{definition}

\begin{remark}{Justification of P.M.F.}{Justification of P.M.F.}
$$
\sumi p_X (i)=\sumi(1-p)^{i-1}\cdot p=p\cdot\frac{1}{1-(1-p)}=1 
$$
thus $p_X (\cdot)$ is a p.m.f. 
\end{remark}

\begin{theorem}{Expectation and Variance of Geometric R.V.}{Expectation and Variance of Geometric R.V.}
Suppose $X\sim$ geometric$(p)$, where $0<p<1$. 
Then
$$
E[X]=\frac{1}{p},\qquad
Var(X)=\frac{1-p}{p^2}.
$$
\end{theorem}

\begin{proof}
\\[4cm]\vspace{0.01cm}
\end{proof}

\begin{theorem}{Memoryless Property}{Memoryless Property}
Suppose $X$ is a discrete r.v. of a probability space $(\Omega,\ma, P)$ with $X(\Omega)=\{1,2\cdots\}$. Then $P[(X>m+n)|(X>m)]=P(X>n),\ \forall m,n>0 \Leftrightarrow X$ is a geometric r.v.
\end{theorem}

\begin{proof}
\\[4cm]\vspace{0.01cm}
\end{proof}

Consider an experiment in which independent Bernoulli trials with the same
probability of success, say $p$, are performed until the $r^{th}$ success occurs, where $r\gs1$.

Let $X$ be the number of Bernoulli trials until the  $r^{th}$ success occurs,
$$
p_X (i)=
\left\{\begin{aligned}
&\binom {i-1} {r-1} p^r(1-p)^{i-r},\   i=r,r+1,\cdots\\
&0,\quad\text{o.w.}
\end{aligned}        
\right.   
$$

\begin{definition}{Negative Binomial R.V.}{Negative Binomial R.V.}
A discrete r.v. of a probability space $(\Omega,\ma, P)$ is called a negative binomial r.v. with parameters $r$ and $p$ where $r\gs1$ and $0<p<1$, denoted $X\sim$ neg.-binomial$(r,p)$, if its p.m.f is given by
$$
p_X (i)=
\left\{\begin{aligned}
&\binom {i-1} {r-1} p^r(1-p)^{i-r},\   i=r,r+1,\cdots\\
&0,\quad\text{o.w.}
\end{aligned}        
\right.   
$$
Such a p.m.f is called a negative binomial p.m.f with parameters $r$ and $p$.
\end{definition}

\begin{remark}{Geometric R.V. from Negative Binomial R.V.}{Geometric R.V. from Negative Binomial R.V.}
(1) A geometric r.v. with parameter $p$ is a negative binomial r.v. with parameters 1 and $p$.\\
(2)
$$
\begin{aligned}
\sum_{i=r}^{\infty}\left(i-1\right)\left(i-2\right)\cdots\left(i-r+1\right)x^{i-r}&=\frac{\dif^{r-1}}{\dif x^{r-1}}\left(\sum_{i=1}^{\infty}x^{i-1}\right)\\
&=\frac{\dif^{r-1}}{\dif x^{r-1}}\left(\frac{1}{1-x}\right)=\frac{\left(r-1\right)!}{(1-x)^r}\\
\end{aligned}
$$
$$
\begin{aligned}
&\Rightarrow\sum_{i=r}^{\infty}{p_X\left(i\right)}=\sum_{i=r}^{\infty}{\binom{i-1}{r-1}p^r(1-p)^{i-r}}=\frac{p^r}{\left(r-1\right)!}\cdot\frac{\left(r-1\right)!}{(1-(1-p))^r}=1\\
   &\Rightarrow\ p_X\left(\cdot\right) \text{is a p.m.f.}
   \end{aligned}
$$
\end{remark}

\begin{theorem}{Expectation and Variance of Negative Geometric R.V.}{Expectation and Variance of Negative Geometric R.V.}
Suppose $X\sim$ neg.-binomial$(r,p)$, where $r\gs1$ and $0<p<1$. Then
$$
E\left[X\right]=\frac{r}{p},\qquad
Var(x)=\frac{r(1-p)}{p^2}.
$$
\end{theorem}

\begin{proof}
\\[4cm]\vspace{0.01cm}
\end{proof}

\begin{theorem}{Maximum Point of Negative Geometric Probability}{Maximum Point of Negative Geometric Probability}
Suppose $X\sim$ neg.-binomial$(r,p)$, where $r\gs1$ and $0<p<1$. Then
$$
arg\max\limits_{i\gs r}p_X\left(i\right)=
\left\{\begin{array}{cl}                
1,                             &\text{if } r=1\\
\dfrac{r-1}{p}\ \text{or } \dfrac{r-1}{p+1},\  &\text{if } \dfrac{r-1}{p}\in\mz^+\vspace{0.3cm}\\
\left\lfloor\dfrac{r-1}{p+1}\right\rfloor,               &\text{if } \dfrac{r-1}{p}\notin\mz     
\end{array}
\right.
$$
\end{theorem}

\begin{proof}
\\[4cm]\vspace{0.01cm}
\end{proof}

A box contains $N_1$ red balls and $N_2$ blue balls. Suppose that $n$ balls are randomly drawn from the box, one by one and without replacement.

Let $X$ be the number of ``red'' balls drawn
$$
p_X\left(i\right)=
\left\{\begin{aligned}
&\dfrac{\binom {N_1} i \binom {N_2} {n-i}}{\binom {N_1+N_2} n},i=a,a+1,\ \cdots, b. \ a=\max\{n-N_1,0\},b=\min\{n,N_1\}     \\
&\quad0, \qquad\qquad\text{o.w.}              
\end{aligned}\right.                                                    
$$

\begin{definition}{Hypergeometric R.V.}{Hypergeometric R.V.}
A discrete r.v. of a probability space $(\Omega,\ma, P)$ is called a hypergeometric r.v. with parameter $N_1,\ N_2$ and $n$ where $N_1,N_2\gs1$ and $n\gs1$, denoted $X\sim$ hypergeometric$(N_1,N_2,n)$, if its p.m.f is given by
$$
p_X\left(i\right)=
\left\{\begin{aligned}
&\dfrac{\binom {N_1} i \binom {N_2} {n-i}}{\binom {N_1+N_2} n},i=a,a+1,\ \cdots, b. \ a=\max\{n-N_1,0\},b=\min\{n,N_1\}     \\
&\quad0, \qquad\qquad\text{o.w.}              
\end{aligned}\right.                                                    
$$
Such a p.m.f is called a hypergeometric r.v. with parameter $N_1,\ N_2$ and $n$.
\end{definition}

\begin{remark}{Justif\/ication of P.M.F.}{Justification of c P.M.F.}
(1) If $n\ls\min\{N_1,N_2\}\Rightarrow a=\max\{n-N_1,0\}=0,b=\min\{n,N_1\}=n$.\\
(2) 
$$
\begin{aligned}
&(1+x)^{N_1+N_2}=(1+x)^{N_1}(1+x)^{N_2}\\
\Rightarrow\ \ &\text{the coef\/f\/icient of } x^n \text{ is }
\binom{N_1+N_2}{n}=\sum_{i=a}^{b}\binom{N_1}{i}\binom{N_2}{n-i},\\
&\text{where} \ a=\max\{n-N_1,0\},b=\min\{n,N_1\} \\
\Rightarrow\ &\sum_{i=a}^{b}{p_X\left(i\right)}=\sum_{i=a}^{b}\frac{\binom{N_1}{i}\binom{N_2}{n-i}}{\binom{N_1+N_2}{n}}=1.\\
\Rightarrow\ &p_X\left(\cdot\right)\ \text{is a p.m.f.}
\end{aligned}
$$
\end{remark}

\begin{theorem}{Expectation and Variance of Hypergeometric R.V.}{Expectation and Variance of Hypergeometric R.V.}
Suppose $X\sim$ hypergeometric$(N_1,N_2,n)$, where $N_1,N_2\gs1$ and $1\ls n\ls\min\{N_1,N_2\}$. Then
$$
E\left[X\right]=\frac{nN_1}{N_1+N_2},\ 
Var(x)=n\cdot\frac{N_1}{N_1+N_2}\cdot\frac{N_2}{N_1+N_2}\cdot\left(1-\frac{n-1}{N_1+N_2-1}\right).
$$
\end{theorem}

\begin{proof}
\\[4cm]\vspace{0.01cm}
\end{proof}

\begin{theorem}{Binomial Approximation for Hypergeometric}{Binomial Approximation for Hypergeometric}
$n$ balls are drawn with replacement
$$\begin{aligned}
  &\Rightarrow\ X\sim\ \mathrm{binomial}\left(n,\frac{N_1}{N_1+N_2}\right)\\
&\Rightarrow\ E\left[X\right]=n\cdot\frac{N_1}{N_1+N_2},\ \ Var(x)=n\cdot\frac{N_1}{N_1+N_2}\cdot\frac{N_2}{N_1+N_2}.
\end{aligned}
$$
Therefore, if $n\ll N_1+N_2$, then drawing with replacement is a good approximation of drawing without replacement.
\end{theorem}

\begin{proof}
\\[4cm]\vspace{0.01cm}
\end{proof}

\begin{theorem}{Maximum Point of Hypergeometric Probability}{Maximum Point of Hypergeometric Probability}
Suppose $X\sim\ \mathrm{hypergeometric}(N_1,N_2,n)$, where $N_1,N_2\gs1$ and $1\ls n\ls\min\{N_1,N_2\}$. Then
$$
\begin{aligned}
&arg\max\limits_{0\ls i\ls n}{p_X\left(i\right)}\\
&=\left\{\begin{aligned}
&\frac{(n+1)(N_1+1)}{N_1+N_2+2}-1 \text{ or } \frac{(n+1)(N_1+1)}{N_1+N_2+2},     \ \text{if }  \frac{(n+1)(N_1+1)}{N_1+N_2+2}\in\mz   \\
&\left\lfloor \frac{(n+1)(N_1+1)}{N_1+N_2+2}\right\rfloor, \ \text{if }  \frac{(n+1)(N_1+1)}{N_1+N_2+2}\notin\mz  
\end{aligned}\right.   
\end{aligned}
$$
\end{theorem}

\begin{proof}
\\[4cm]\vspace{0.01cm}
\end{proof}

\begin{remark}{Binomial and Poisson Approximation for Hypergeometric}{Binomial and Poisson Approximation for Hypergeometric}
$$
\begin{aligned}
p_X\left(i\right)&=\dfrac{\binom{N_1}{i}\binom{N_2}{n-i}}{\binom{N_1+N_2}{n}}\\
&=\frac{n!}{i!\left(n-i\right)!}\cdot\frac{N_1\left(N_1-1\right)\cdots\left(N_1-i+1\right)N_2\left(N_2-1\right)\cdots(N_2-n+i+1)}{\left(N_1+N_2\right)\left(N_1+N_2-1\right)\cdots(N_1+N_2+n-1)}
\end{aligned}
$$
(1) If $N_1\rightarrow\infty,\ N_2\rightarrow\infty,\ \dfrac{N_1}{N_1+N_2}\rightarrow p$, then
$$\begin{aligned}
p_X\left(i\right)&=\binom{n}{i}\cdot\frac{1}{1\cdot\left(1-\dfrac{1}{N_1+N_2}\right)\cdots\left(1-\dfrac{n-1}{N_1+N_2}\right)}\\
&\cdot\frac{N_1}{N_1+N_2}\left(\frac{N_1}{N_1+N_2}-\frac{1}{N_1+N_2}\right)\cdots\left(\frac{N_1}{N_1+N_2}-\frac{i-1}{N_1+N_2}\right)\left(\frac{N_2}{N_1+N_2}\right)\\
&\cdot\left(\frac{N_2}{N_1+N_2}-\frac{1}{N_1+N_2}\right)\cdots\left(\frac{N_2}{N_1+N_2}-\frac{n-i-1}{N_1+N_2}\right)\\
&\xrightarrow{N_1,\ N_2\rightarrow\infty}\binom{n}{i}p^i(1-p)^{n-i}\leftarrow\mathrm{binomial}(n,p)\\
\end{aligned}
$$
(2) If $n\rightarrow\infty,\ N_1\rightarrow\infty,\ N_2\rightarrow\infty,\ \dfrac{n}{N_1+N_2}\rightarrow0,\ \dfrac{N_1}{N_1+N_2}\rightarrow\dfrac{\lambda}{n}$, then
$$\begin{aligned}
p_X\left(i\right)&=\frac{1}{i!}\cdot\frac{1}{\dfrac{\left(N_1+N_2\right)!}{\left(N_1+N_2-n\right)!}}\cdot nN_1\cdot\left(n-1\right)\left(N_1-1\right)\cdots\left(n-i+1\right)\left(N_1-i+1\right)\\ 
&\cdot(N_1+N_2-N_1)(N_1+N_2-N_1-1)\cdots(N_1+N_2-N_1-n+i+1)
\\
&=\frac{1}{i!}\cdot\frac{\prod\limits_{j=0}^{i-1}{\dfrac{nN_1-j\left(n+N_1\right)+j^2}{N_1+N_2}\cdot\prod\limits_{j=0}^{n-i-1}\left(1-\dfrac{N_1+j}{N_1+N_2}\right)}}{\dfrac{1}{(N_1+N_2)^n}\cdot\dfrac{\sqrt{2\pi\left(N_1+N_2\right)}\left(\dfrac{N_1+N_2}{e}\right)^{N_1+N_2}e^{a_{N_1+N_2}}}{\sqrt{2\pi\left(N_1+N_2-n\right)}\left(\dfrac{N_1+N_2-n}{e}\right)^{N_1+N_2-n}e^{a_{N_1+N_2-n}}}}\\
\end{aligned}
$$
where $a_n=\ln{\dfrac{n!}{\sqrt{2\pi n}\left(\dfrac{n}{e}\right)^n}}\xrightarrow{n\rightarrow\infty}0.\ $
$$\begin{aligned}
p_X\left(i\right)&\xrightarrow{\dis n,N_1,N_2\rightarrow\infty,\dfrac{n}{N_1+N_2}\rightarrow0,\ \dfrac{N_1}{N_1+N_2}\rightarrow\dfrac{\lambda}{n}}\\
&\frac{1}{i!}\cdot\limn{\dfrac{\lambda^i\left(1-\dfrac{\lambda}{n}\right)^{n-i}}{\dfrac{1}{e^n\cdot\lim\limits_{N_1,N_2\rightarrow\infty}{\left(1-\dfrac{n}{N_1+N_2}\right)^{N_1+N_2-n}}}}}\\
&=\lim\limits_{n\rightarrow\infty}{\frac{\lambda^i}{i!}}\ \left(1-\frac{\lambda}{n}\right)^{n-i}=e^{-\lambda}\cdot\frac{\lambda^i}{i!}\ \leftarrow\mathrm{Poisson}(\lambda)\\
\end{aligned}
$$
\end{remark}

\chapter{Continuous Random Variables}
\section{Probability Density Function}

\begin{definition}{Probability Density Function}{Probability Density Function}
Let $X$ be a r.v. of a probability space $(\Omega,\ma, P)$. $X$ is called an absolutely continuous (or a continuous) r.v. if there exists a nonnegative real-valued function $f_X:\ \mathbb{R}\rightarrow[0,\infty)$ s.t.
$$
P(x\in B)=\int_{B}{f_X(x)\dif x},\ \forall B\in \mb_\mr.
$$
The function $f_X$ is called the probability density function (p.d.f.) of $X$.
\end{definition}

\begin{remark}{Approximation of Probability}{Approximation of Probability}
$$
P(a\ls X\ls a+\delta)=\int_{a}^{a+\delta}{f_X(x)\dif x}=f_X\left(a_\delta\right)\cdot\delta,
$$
for some $a_\delta\in[a,\ a+\delta]$.\\
If $f_X$ is continuous at $a$
$$
\Rightarrow\lim\limits_{\delta\rightarrow0}{\dfrac{P(a\ls X\ls a+\delta)}{\delta}}=\lim\limits_{\delta\rightarrow0}{f_X(a_\delta)}=f_X\left(a\right).
$$
So $P(a\ls X\ls a+\delta)\approx f_X(a_\delta)\cdot\delta$, if $f_X$ is continuous at $a$ and $\delta$ is very small.
\end{remark}

\begin{theorem}{C.D.F and Probability from P.D.F.}{C.D.F and Probability from P.D.F.}
Suppose $X$ is a continuous r.v. of a probability space $(\Omega,\ma, P)$.\\
(1)
$$
F_X(x)=\int_{-\infty}^{x}{f_X(t)\dif t}.
$$
   Therefore, $F_X(x)$ is a continuous function.\\
(2)
$$
\int_{-\infty}^{\infty}{f_X(x)\dif x}=1
$$
(3) If $f_X$ is continuous at $a$, then $F_X^\prime(a)=f_X(a)$. Therefore, if $f_X$ is a continuous function, then $F_X^\prime(x)=f_X(x),\ \forall x\in\mathbb{R}$.\\
(4) $P(X=a)=0,\forall a\in\mathbb{R}$. Therefore,
$$\begin{aligned}
&P(a\ls X\ls b)=P(a\ls X<b)\\
=&P(a<X\ls b)=P(a<X<b)\\
=&\int_{a}^{b}{f_X(x)\dif x}.
\end{aligned}
$$
\end{theorem}

\begin{proof}
\\[4cm]\vspace{0.01cm}
\end{proof}

\begin{theorem}{Existence of P.D.F.}{Existence of P.D.F.}
Suppose $f:\ \mathbb{R}\rightarrow[0,\infty)$ is a nonnegative real-valued function s.t.
$$
\int_{-\infty}^{\infty}{f(x)\dif x}=1.
$$
Then there exists a continuous r.v. $X$ of some probability space $(\Omega,\ma, P)$ s.t. the p.d.f. is equal to $f$.
\end{theorem}

\begin{proof}
\\[4cm]\vspace{0.01cm}
\end{proof}

\begin{definition}{Suf\/f\/icient Conditions of P.D.F.}{Sufficient Conditions of P.D.F.}
A nonnegative real-valued function $f:\ \mathbb{R}\rightarrow[0,\infty)$ s.t.
$$
\int_{-\infty}^{\infty}{f(x)\dif x}=1
$$
is called a p.d.f. \\
The c.d.f. $F:\ \mathbb{R}\rightarrow[0,1]$ associated with $f$ is given by
$$
F(t)=\int_{-\infty}^{t}{f(x)\dif x},\forall t\in\mathbb{R}.
$$
\end{definition}

\begin{remark}{Neither Discrete Nor Continuous R.V.}{Neither Discrete Nor Continuous R.V.}
There are r.v.'s that are neither discrete nor continuous, e.g., $$F_X(x)=\alpha F_d(x)+(1-\alpha)F_c(x),$$
where $0<\alpha<1$.
\end{remark}

\section{The Probability Density Function of A Function of A R.V.}

\begin{theorem}{Method of Distribution Functions}{Method of Distribution Functions}
Suppose $X$ is a continuous r.v. of a probability space$(\Omega,\ma, P)$.\\
If $Y=g(X)$, then
$$\begin{aligned}
f_Y(y)&=\frac{\dif}{\dif y}\left[F_Y(y)\right]=\frac{\dif}{\dif y}\left[P(Y\ls y)\right]=\frac{\dif}{\dif y}\left[P[g(x)\ls y]\right]\\
&\rightarrow\frac{\dif}{\dif y}\left[X\sim g^{-1}(y)\right]\rightarrow\frac{\dif}{\dif y}\left[F_X\left(g^{-1}(y)\right)\right]\rightarrow\frac{\dif}{\dif y}\left[g^{-1}(y)\right]\cdot f_X\left[g^{-1}(y)\right].
\end{aligned}
$$
\end{theorem}

\begin{proof}
\\[4cm]\vspace{0.01cm}
\end{proof}

\begin{theorem}{Method of Transformations}{Method of Transformations}
Suppose $X$ is a continuous r.v. of a probability space $(\Omega,\ma, P)$ such that its p.d.f. is continuous. Suppose $Y=g(X)$, where $g$ is a measurable function from $(\mathbb{R},\mb_\mathbb{R})$ to $(\mathbb{R},\mb_\mathbb{R})$.\\
(1) If $g(X)$ is a discrete r.v., then
$$
P_Y(y)=\int_{x:g(x)=y}{f_X(x)\dif x},\ \forall y\in g\left[X(\Omega)\right].
$$
(2) If $g(X)$ is a continuous r.v., $g^\prime(x)$ exists, and $g^\prime(x)\neq0,\ \forall x\in g^{-1}\left(\left\{y\right\}\right):\{x:g(x)=y\}$, where $y\in g\left[X(\Omega)\right]$. Then,
$$
f_Y(y)=\sum_{x:g(x)=y}\frac{f_X(x)}{|g^\prime(x)|}.
$$
\end{theorem}

\begin{proof}
\\[4cm]\vspace{0.01cm}
\end{proof}

\section{Expectations and Variances}

\begin{definition}{Expectation}{Expectation}
Let $X$ be a continuous r.v. of a probability space $(\Omega,\ma, P)$ s.t. its p.d.f. is continuous. The expectation (or mean) of $X$ is given by
$$
E[X]=\int_{-\infty}^{\infty}xf_X(x)\dif x
$$
if $xf_X(x)$ is absolutely integrable, i.e.,
$$
\int_{-\infty}^{\infty}{|xf_X(x)|\dif x}<+\infty,
$$
and is given by $E[X]=\pm\infty$, if the integration diverges to $\pm\infty$.
\end{definition}

\begin{remark}{Necessary and Suf\/f\/icient Condition of Absolutely Integrable}{Necessary and Sufficient Condition of Absolutely Integrable}
$$\begin{aligned}
&E[X]=\int_{-\infty}^{\infty}xf_X(x)\dif x=\int_{0}^{\infty}xf_X(x)\dif x-\int_{-\infty}^{0}(-x)f_X(x)\dif x\\
\Rightarrow &E[|X|]=\int_{0}^{\infty}xf_X(x)\dif x+\int_{-\infty}^{0}(-x)f_X(x)\dif x\\
\therefore\ &E[|X|]<\infty\Leftrightarrow \int_{0}^{\infty}xf_X(x)\dif x<\infty  \text{ and } \int_{-\infty}^{0}(-x)f_X(x)\dif x<\infty.
\end{aligned}$$
\end{remark}

\begin{theorem}{Calculation of Expectation}{Calculation of Expectation}
Suppose $X$ is a continuous r.v. of a probability space $(\Omega,\ma, P)$.
Then
$$\begin{aligned}
E[X]&=\int_{0}^{\infty}P(x>t)\dif t-\int_{0}^{\infty}P(x\ls-t)\dif t\\
&=\int_{0}^{\infty}[1-F_X(t)]\dif t-\int_{0}^{\infty}[F_X(-t)]\dif t.
\end{aligned}$$
\end{theorem}

\begin{proof}
\\[4cm]\vspace{0.01cm}
\end{proof}

\begin{corollary}{Calculation of $r^{th}$ Moment}{Calculation Moment}
Suppose $X$ is a nonnegative continuous r.v. of a probability space $(\Omega,\ma, P)$, and $r>0.$ Then
$$
E\lbrack X^{r}\rbrack = \int_{0}^{\infty}{rt^{r - 1}}P(x > t)\dif t = \int_{0}^{\infty}{rt^{r - 1}}\left[ 1 - F_{X}( t ) \right]\dif t.
$$
In particular,
$$
E\lbrack X\rbrack = \int_{0}^{\infty}{P(x > t)}\dif t = \int_{0}^{\infty}\left[ 1 - F_{X}(t) \right]\dif t.
$$
\end{corollary}

\begin{proof}
\\[4cm]\vspace{0.01cm}
\end{proof}

\begin{theorem}{Approximation of Expectation}{Approximation of Expectation}
Suppose $X$ is a continuous r.v. of a probability space $(\Omega,\ma, P)$. Then
$$
\sum_{n = 1}^{\infty}{P(|X| \gs n)} \ls E\lbrack|X|\rbrack \ls 1 + \sum_{n = 1}^{\infty}{P( \left| X \right| \gs n )}.
$$
Therefore,
$$
E\lbrack|X|\rbrack < \infty\  \Leftrightarrow \sum_{n = 1}^{\infty}{P(|X| \gs n)} \ls \infty.
$$
\end{theorem}

\begin{proof}
\\[4cm]\vspace{0.01cm}
\end{proof}

\begin{theorem}{Inf\/inite Zero}{Infinite Zero}
Suppose $X$ is a continuous r.v. of a probability space $(\Omega,\ma, P)$. Then,
$$
E\lbrack X\rbrack < \infty \Rightarrow \lim_{x \rightarrow \infty}{x \cdot P(X > x)} = \lim_{x \rightarrow - \infty}{x \cdot P(X \ls x)} = 0.
$$
\end{theorem}

\begin{proof}
\\[4cm]\vspace{0.01cm}
\end{proof}

\begin{theorem}{Expectation of Measurable Function}{Expectation of Measurable Function}
Suppose $X$ is a continuous r.v. of a probability space $(\Omega,\ma, P)$, and suppose $g$ is a measurable function from $(\mathbb{R},\mb_\mathbb{R})$ to $(\mathbb{R},\mb_\mathbb{R})$.
Then
\[E\lbrack g(X)\rbrack = \int_{- \infty}^{\infty}{g(X)} \cdot f_{X}(x)\dif x\]
\end{theorem}

\begin{proof}
\\[4cm]\vspace{0.01cm}
\end{proof}

\begin{corollary}{Expectation of Linear Combination of Measurable Functions}{Expectation of Linear Combination of Measurable Functions}
Suppose $X$ is a continuous r.v. of a probability space $(\Omega,\ma, P)$. $g_1,g_2,\cdots g_n$ are measurable functions from $(\mathbb{R},\mb_\mathbb{R})$ to $(\mathbb{R},\mb_\mathbb{R})$, and $\alpha_1,\alpha_2,\cdots\alpha_n\in\mathbb{R}$. Then
\[E\left\lbrack \sum_{i = 1}^{n}\alpha_{i}g_{i}(x) \right\rbrack = \sum_{i = 1}^{n}\alpha_{i}E\lbrack g_{i}(X)\rbrack\]
\end{corollary}

\begin{proof}
\\[4cm]\vspace{0.01cm}
\end{proof}

\begin{definition}{Variance and Standard Deviation}{Variance and Standard Deviation}
Let $X$ be a continuous r.v. of a probability space $(\Omega,\ma, P)$ and suppose $E\lbrack X\rbrack$ exists. The \textbf{variance} of $X$ is given by $Var(x) = E\lbrack(X - E\lbrack X\rbrack)^{2}\rbrack$. And the \textbf{standard deviation} of $X$ is given by
$\sigma_{X} = \sqrt{Var(x)} = \sqrt{E\lbrack(X - E\lbrack X\rbrack)^{2}\rbrack}$.
\end{definition}

\begin{theorem}{Minimum Distance with Expectation}{Minimum Distance with Expectation}
Suppose $X$ is a continuous r.v. of a probability space $(\Omega,\ma, P)$, and suppose $E[X]$ exists. If $E\left[X^2\right]<+\infty$, then $Var(x)=\min\limits_{a\in\mr}E[(X-a)^2]$.
\end{theorem}

\begin{proof}
\\[4cm]\vspace{0.01cm}
\end{proof}

\begin{theorem}{Calculation of Linear Combination}{Calculation of Linear Combination}
Suppose $X$ is a continuous r.v. of a probability space $(\Omega,\ma, P)$, and suppose $E[X]$ exists. Then\\
(1)
$$
Var(x)=E\left[X^2\right]-(E[X])^2
$$
(2)
$$
Var(aX+b)=a^2Var(x),\quad
\sigma_{aX+b}=\left|a\right|\sigma_X,\ \forall a,b\in\mathbb{R}.
$$
\end{theorem}

\begin{proof}
\\[4cm]\vspace{0.01cm}
\end{proof}

\begin{definition}{Moment and Absolute Moment}{Moment and Absolute Moment}
Let $X$ be a continuous r.v. of a probability space $(\Omega,\ma, P)$, and $r,c\in\mr$.
$$\left\{
\begin{array}{l}
\text{The }r^{th}\ \text{moment of } X\ \text{is given by } E[X^r ]   \\                              
\text{The }r^{th}\ \text{central moment of } X\ \text{is given by } E[(X-E[X])^r]\\
\text{The }r^{th}\ \text{moment of } c\ \text{is given by } E[(X-c)^r ]      \\     
\text{The }r^{th}\ \text{absolute moment of } X\ \text{is given by } E[|X|^r ]\\
\text{The }r^{th}\ \text{absolute central moment of } X \text{is given by } E[|X-E[X] |^r]  \\
\text{The } r^{th}\ \text{absolute moment of } c\ \text{is given by } E[|X-c|^r ] 
\end{array}
\right.$$
If the respective sum converges absolutely.
\end{definition}

\begin{theorem}{Existence of Lower Order Moment}{Existence of Lower Order Moment}
Suppose $X$ is a continuous r.v. of a probability space $(\Omega,\ma, P)$ and suppose $0<r<s$. If $E[|X|^s]$ exists, then $E[|X|^r]$ exists. That is, the existence of a higher order moment of $X$ guarantees the existence of a lower order moment of $X$.
\end{theorem}

\begin{proof}
\\[4cm]\vspace{0.01cm}
\end{proof}

\begin{theorem}{Positive Variance}{Positive Variance}
Suppose $X$ is a continuous r.v. of a probability space $(\Omega,\ma, P)$. Then
$$
E\left[\left(X-a\right)^2\right]>0,\ \forall a\in\mathbb{R}.
$$
Therefore
$$
E[X] \text{ exists }\Rightarrow   Var(X)>0.
$$
\end{theorem}

\begin{proof}
\\[4cm]\vspace{0.01cm}
\end{proof}

\chapter{Special Continuous Distributions}
\section{Uniform R.V.'s}

\begin{definition}{Uniform R.V.}{Uniform R.V.}
A continuous r.v. of a probability space $(\Omega,\ma, P)$ is called a uniform r.v. over $(\alpha,\beta)$, where $\alpha,\beta\in\mathbb{R}$ and $\alpha<\beta$, denoted $X\sim U(\alpha,\beta)$, if its p.d.f. is given by
$$
f_{X}(x) = \left\{ \begin{aligned}
&\frac{1}{\beta - \alpha},\quad\mathrm{\text{if}}\ \alpha < x < \beta \\
&\quad0,\qquad\mathrm{\text{o.w.}}\\
\end{aligned} \right.
$$
\end{definition}

\begin{remark}{P.D.F. and C.D.F.}{P.D.F. and C.D.F.}
(1) $f_X(x)\gs0,\ \forall x\in\mathbb{R}$, and
$$
\int_{-\infty}^{\infty}{f_X(x)}\dif x=\int_{\alpha}^{\beta}\frac{1}{\beta-\alpha}\dif x=1
$$
$\Rightarrow f_X(x)$ is a p.d.f.\\
(2)
$$
F_{X}(x) = \left\{ \begin{array}{cl}
0,&\mathrm{\text{if}}\ x \ls \alpha \\
\dfrac{x - \alpha}{\beta - \alpha},&\mathrm{\text{if}}\ \alpha < x < \beta \\
1,&\mathrm{\text{if}}\ x \gs \beta \\
\end{array} \right.
$$
\end{remark}

\begin{theorem}{Expectation and Variance of Uniform R.V.}{Expectation and Variance of Uniform R.V.}
Suppose $X\sim U(\alpha,\beta)$, where $\alpha,\beta\in\mathbb{R}$ and $\alpha<\beta$. Then
$$
E\lbrack X^{n}\rbrack = \frac{\sum\limits_{i = 1}^{n}{\alpha^{n - i}\beta^{i}}}{n + 1}.
$$
Therefore
$$
E\lbrack X\rbrack = \frac{\alpha + \beta}{2},\qquad
Var(x)=\frac{(\beta-\alpha)^2}{12}.
$$
\end{theorem}

\begin{proof}
\\[4cm]\vspace{0.01cm}
\end{proof}

\begin{remark}{Expectation and Variance of Discrete ``Uniform R.V.''}{Expectation and Variance of Discrete ``Uniform R.V.''}
Suppose $X\sim\ \mathrm{Uniform}(1,2,\ \cdots,n)$, where $n\gs1$. Then
$$
E\lbrack X\rbrack = \frac{n + 1}{2},\quad E\left\lbrack X^{2}\right\rbrack = \frac{(n + 1)(2n + 1)}{6}
$$
and
$$
Var(x)=\frac{n^2-1}{12}.
$$
\end{remark}

\begin{theorem}{Linear Generated R.V.}{Linear Generated R.V.}
Suppose $X\sim U(\alpha,\beta)$, where $\alpha,\beta\in\mathbb{R}$ and $\alpha<\beta$. Suppose $Y=aX+b$, where $\alpha,\beta\in\mathbb{R}$ and $a\neq0$. Then
\[Y\sim\left\{ \begin{aligned}
&U\left( a\alpha + b,a\beta + b \right),\quad\mathrm{\text{if}}\ a > 0 \\
&U\left( a\beta + b,a\alpha + b \right),\quad\mathrm{\text{if}}\ a < 0 \\
\end{aligned} \right.\]
\end{theorem}

\begin{proof}
\\[4cm]\vspace{0.01cm}
\end{proof}

\section{Normal (Gaussian) R.V.'s}

\begin{definition}{Normal (Gaussian) R.V.}{Normal (Gaussian) R.V.}
A continuous r.v. of a probability space $(\Omega,\ma, P)$ is called a normal (Gaussian) r.v. with parameters $\mu$ and $\sigma^2$, where $\mu,\sigma\in\mathbb{R}$, $\sigma\neq0$, denoted $X\sim N(\mu,\sigma^2)$, if its p.d.f. is given by
$$
f_X(x)=\frac{1}{\sqrt{2\pi\sigma^2}}\cdot\exp\left[{-\frac{(x-\mu)^2}{2\sigma^2}}\right],\ -\infty<x<\infty.
$$
\end{definition}

\begin{remark}{P.D.F. and C.D.F.}{P.D.F. and C.D.F.}
(1) $f_X(x)\gs0,\ \forall x\in\mathbb{R}$, and let $I=\int_{-\infty}^{\infty}e^{-ax^2}\dif x$.
$$\begin{aligned}
I^2&=\int_{-\infty}^{\infty}\int_{-\infty}^{\infty}{e^{-a\left(x^2+y^2\right)}\dif x\dif y}\\
&\xrightarrow{x=r\cos\theta,y=r\sin\theta}\int_{0}^{\infty}\int_{0}^{2\pi}{e^{-ar^2}r\dif r\dif\theta}=\frac{\pi}{a}\\
&\Rightarrow I=\sqrt{\frac{\pi}{a}}\Rightarrow\int_{-\infty}^{\infty}{\sqrt{\frac{a}{\pi}}}\cdot e^{-ax^2}\dif x=1
\end{aligned}
$$
$$
\therefore\int_{-\infty}^{\infty}{f_X(x)}\dif x=\int_{-\infty}^{\infty}{\frac{1}{\sqrt{2\pi\sigma^2}}e^{-\tfrac{(x-\mu)^2}{2\sigma^2}}}\dif x=\int_{-\infty}^{\infty}{\frac{1}{\sqrt{2\pi\sigma^2}}e^{-\tfrac{x^2}{2\sigma^2}}}\dif x=1
$$
$\Rightarrow f_X(x)$ is a p.d.f.\\
(2) If $\mu=0,\ \sigma^2=1$, then $X$ is called a standard normal (Gaussian) r.v.\\
(3)
$$\begin{aligned}
F_X(x)&=\int_{-\infty}^{x}{\frac{1}{\sqrt{2\pi\sigma^2}}e^{-\tfrac{(y-\mu)^2}{2\sigma^2}}}\dif y\\
&\xrightarrow{y=\sigma t+\mu}\int_{-\infty}^{\tfrac{x-\mu}{\sigma}}{\frac{1}{\sqrt{2\pi}}e^{-\tfrac{{t}^2}{2}}}\dif t\\
&={\Phi}\left(\frac{x-\mu}{\sigma}\right)\\
\end{aligned}
$$
where
$$
{\Phi}(x)=\int_{-\infty}^{x}{\frac{1}{\sqrt{2\pi}}e^{-\tfrac{{t}^2}{2}}}\dif t.
$$
\end{remark}

\begin{theorem}{Symmetric about ${\mu}$}{Symmetric about}
Suppose $X\sim N(\mu,\sigma^2)$.\\
(1) $f_X(x)$ is symmetric about $x=\mu$, with maximum at $x=\mu$, and inflection points at $x=\mu\pm\sigma$.\\
(2) $\Phi\left(-y\right)=1-{\Phi}(y),\ \forall y\in\mathbb{R}$ and $\Phi\left(0\right)=1$. Therefore, $$F_X\left(\mu-y\right)=1-F_X\left(\mu+y\right)$$ and $$F_X\left(\mu\right)=\frac{1}{2}.$$
\end{theorem}

\begin{proof}
\\[4cm]\vspace{0.01cm}
\end{proof}

\begin{theorem}{Linear Generated R.V.}{Linear Generated R.V.}
Suppose $X\sim N(\mu,\sigma^2)$, where $\mu,\sigma\in\mathbb{R},\ \sigma\neq0$. Suppose $Y=aX+b$, where $\alpha,\beta\in\mathbb{R}$ and $a\neq0$. Then,
$$
Y\sim N\left(a\mu+b,a^2\sigma^2\right).
$$
In particular, if
$$
Y=\frac{x-\mu}{\sigma},
$$
then
$$
Y\sim N(0,1).
$$
\end{theorem}

\begin{proof}
\\[4cm]\vspace{0.01cm}
\end{proof}

\begin{definition}{Gamma Function}{Gamma Function}
The function $\Gamma:(0,\infty)\rightarrow\mathbb{R}$ given by
$$
{\Gamma}(\alpha)=\int_{0}^{\infty}e^{-t}t^{\alpha-1}\dif t,\ \forall\alpha>0
$$
is called the gamma function.
\end{definition}

\begin{theorem}{Properties of Gamma Function}{Properties of Gamma Function}
(1)
$$
{\Gamma}(\alpha+1)=\alpha\Gamma(\alpha),\ \forall\alpha>0.
$$
(2)
$$
{\Gamma}(n+1)=n!,\ \forall n\gs0.
$$
(3)
$$
{\Gamma}\left(n+\frac{1}{2}\right)=\frac{\left(2n\right)!}{2^{2n}n!}\sqrt\pi,\ \forall n\gs0.
$$
\end{theorem}

\begin{proof}
\\[4cm]\vspace{0.01cm}
\end{proof}

\begin{theorem}{Calculation of Moment and Absolute Moment}{Calculation of Moment and Absolute Moment}
Suppose $X\sim N(\mu,\sigma^2)$, where $\mu,\sigma\in\mathbb{R},\ \ \sigma\neq0$.\\
(1)
\[E\left\lbrack \left| x - \mu \right|^{n} \right\rbrack = \frac{\left( 2\sigma^{2} \right)^{\tfrac{n}{2}}}{\sqrt{\pi}}\Gamma\left( \frac{n + 1}{2} \right) = \left\{ \begin{array}{cll}
\dfrac{2^{k + 1} \cdot k!}{\sqrt{2\pi}}\sigma^{2k + 1},&\mathrm{\text{if}}\ n = 2k + 1,&k \gs 0 \vspace{0.5cm}\\
\dfrac{(2k)!}{2^{k} \cdot k!}\sigma^{2k},&\text{if}\ \ n = 2k,&k \gs 0 \\
\end{array} \right.\]
(2)
\[E\left\lbrack {(x - \mu)}^{n} \right\rbrack = \left\{ \begin{array}{cll}
0,&\mathrm{\text{if}}\ n = 2k + 1,&k \gs 0 \vspace{0.5cm}\\
\dfrac{(2k)!}{2^{k} \cdot k!}\sigma^{2k}\ ,&{\text{if}}\ \ n = 2k,&k \gs 0 \\
\end{array} \right.\]
(3)
\[E[X^{n}] = \sum_{k = 0}^{n}\binom n k E\left\lbrack \left( x - \mu \right)^{k} \right\rbrack\cdot \mu^{n - k}.\]
\end{theorem}

\begin{proof}
\\[4cm]\vspace{0.01cm}
\end{proof}

\begin{theorem}{De Moivre-Laplace Theorem}{De Moivre-Laplace Theorem}
Suppose $X\sim\ \mathrm{binomial}(n,p)$, where $n\gs1$ and $0<p<1$. Then
\[\lim_{n \rightarrow \infty}{P\left( a < \frac{X - np}{\sqrt{np( 1 - p )}} < b \right) = \int_{a}^{b}\frac{1}{\sqrt{2\pi}}}e^{- \tfrac{x^{2}}{2}}\dif x,\ \forall a,b \in \mathbb{R},\ a < b.\]
\end{theorem}

\begin{proof}
\\[4cm]\vspace{0.01cm}
\end{proof}

\begin{theorem}{Approximation of $\Phi(x)$}{Approximation of }
\[\frac{1}{\sqrt{2\pi}x}\left( 1 - \frac{1}{x^{2}} \right)e^{- \tfrac{{x}^{2}}{2}} < 1 - \Phi(x) < \frac{1}{\sqrt{2\pi}x} \cdot e^{- \tfrac{{x}^{2}}{2}},\ \forall x > 0.\]
\end{theorem}

\begin{proof}
\\[4cm]\vspace{0.01cm}
\end{proof}

\begin{theorem}{Expectation of Exponential Function}{Expectation of Exponential Function}
Suppose $X\sim N(\mu,\sigma^2)$, where $\mu,\sigma\in\mathbb{R},\ \sigma\neq0$, and $\alpha\in\mathbb{R}$.
Then 
$$
E\left[e^{\alpha x}\right]=e^{\alpha\mu+\tfrac{1}{2}\alpha^2\sigma^2}.
$$
\end{theorem}

\begin{proof}
\\[4cm]\vspace{0.01cm}
\end{proof}

\section{Gamma R.V.'s, Erlang R.V.'s and Exponential R.V.'s}

\begin{definition}{Gamma R.V., Erlang R.V. and Exponential R.V.}{Gamma R.V., Erlang R.V. and Exponential R.V.}
A continuous r.v. of a probability space $(\Omega,\ma, P)$ is called a gamma r.v. with parameters $\alpha$ and $\lambda$, where $\alpha,\lambda>0$, denoted $X\sim \mg(\alpha,\lambda)$, if its p.d.f. is given by
\[f_{X}(x) = \left\{ \begin{aligned}
&\frac{\lambda e^{- \lambda x}{(\lambda x)}^{\alpha - 1}}{\Gamma( \alpha )},\ \mathrm{\text{if}}\ \alpha > 0 \\
&\quad0,\quad\mathrm{\text{o.w.}} \\
\end{aligned} \right.\ \]
If $\alpha=n,\ n\gs1$, then $X$ is called an Erlang r.v. with parameters $n$ and $\lambda$, denoted $X\sim\me(n,\lambda)$.\\
If $\alpha=1$, then $X$ is called an exponential r.v. with parameters $\lambda$, denoted $X\sim\me(\lambda)$.
\end{definition}

\begin{remark}{Properties of P.D.F.}{Properties of P.D.F.3}
(1)
$$
\int_{-\infty}^{\infty}\frac{\lambda e^{-\lambda x}{(\lambda x)}^{\alpha-1}}{{\Gamma}(\alpha)}\dif x\xrightarrow{t=\lambda x}\int_{0}^{\infty}\frac{e^{-t}t^{\alpha-1}}{{\Gamma}(\alpha)}\dif t=\frac{{\Gamma}(\alpha)}{{\Gamma}(\alpha)}=1
$$
$\Rightarrow f_X(x)$ is a p.d.f.\\
(2) 
$$\begin{aligned}
f_X^\prime(x)&=\frac{\lambda^\alpha}{{\Gamma}(\alpha)}\cdot e^{-\lambda x}\left(-\lambda x^{\alpha-1}+\left(\alpha-1\right)x^{\alpha-2}\right)\\
&=\frac{\lambda^\alpha}{{\Gamma}(\alpha)}\cdot e^{-\lambda x}\cdot x^{\alpha-2}\left[-\lambda x+\left(\alpha-1\right)\right]
\end{aligned}$$
$$\begin{aligned}
&f_X^{\prime\prime}(x)\\
&=\frac{\lambda^\alpha}{{\Gamma}(\alpha)}\cdot e^{-\lambda x}\left[-\lambda^2x^{\alpha-1}-\lambda\left(\alpha-1\right)x^{\alpha-2}-\lambda\left(\alpha-1\right)x^{\alpha-2}+(\alpha-2)\left(\alpha-1\right)x^{\alpha-3}\right]\\
&=\frac{\lambda^\alpha}{{\Gamma}(\alpha)}\cdot e^{-\lambda x}\cdot x^{\alpha-3}\left[\left(\lambda x-\left(\alpha-1\right)\right)^2-\left(\alpha-1\right)\right]
\end{aligned}$$
$$
\therefore0<\alpha\ls1\ \Rightarrow\ f_X^\prime(x)<0,\ f_X^{\prime\prime}(x)>0,\ \forall x>0.
$$
\[\alpha > 1 \Rightarrow f_{X}^{'}(x)\left\{ \begin{aligned}
 &> 0\  \Leftrightarrow x < \frac{\alpha - 1}{\lambda} \\
 &= 0 \Leftrightarrow x = \frac{\alpha - 1}{\lambda} \\
 &< 0 \Leftrightarrow x > \frac{\alpha - 1}{\lambda} \\
\end{aligned} \right.\ \]
and
\[f_{X}^{''}(x)\left\{ \begin{aligned}
& > 0\  \Leftrightarrow x > \frac{\alpha - 1}{\lambda} + \frac{\sqrt{\alpha - 1}}{\lambda}\ \text{or}\ x < \frac{\alpha - 1}{\lambda} - \frac{\sqrt{\alpha - 1}}{\lambda} \\
 &= 0\  \Leftrightarrow x = \frac{\alpha - 1}{\lambda} \pm \frac{\sqrt{\alpha - 1}}{\lambda} \\
& < 0\  \Leftrightarrow\frac{\alpha - 1}{\lambda} - \frac{\sqrt{\alpha - 1}}{\lambda} < x < \frac{\alpha - 1}{\lambda} + \frac{\sqrt{\alpha - 1}}{\lambda}\\
\end{aligned} \right.\ \]
\end{remark}

\begin{theorem}{Calculation of C.D.F.}{Calculation of C.D.F.3}
Suppose $X\sim\mg(\alpha,\lambda)$, where $\alpha,\lambda>0$. Then
$$
F_X(x)=1-\frac{{\Gamma}(\alpha,\lambda x)}{{\Gamma}(\alpha)},
$$
where
$$
{\Gamma}(\alpha,x)=\int_{x}^{\infty}{e^{-t}t^{\alpha-1}\dif t}
$$
is the incomplete gamma function.\\
If $\alpha=n\gs1$, then
$$
F_X(x)=1-\sum_{i=0}^{n-1}\frac{e^{-\lambda x}\left(\lambda x\right)^i}{i!}=P(N\gs n)
$$
where $N\sim\mathrm{Poisson}(n\lambda)$.
\end{theorem}

\begin{proof}
\\[4cm]\vspace{0.01cm}
\end{proof}

\begin{theorem}{Expectation and Variance of Gamma R.V.}{Expectation and Variance of Gamma R.V.}
Suppose $X\sim\mg(\alpha,\lambda)$, where $\alpha,\lambda>0$. Then
$$
E\left[X^n\right]=\frac{{\Gamma}(\alpha+n)}{{\Gamma}(\alpha)\lambda^n}=\frac{\binom{n+\alpha-1}{n}}{\lambda^n}=\frac{{(\alpha)}_n}{\lambda^n}
$$
where
$$
{(\alpha)}_n=\binom{n+\alpha-1}{n}=\left(n+\alpha-1\right)\cdots\left(\alpha-1\right)\cdot\alpha
$$
Therefore,
\[E\lbrack X\rbrack = \frac{\alpha}{\lambda}\text{\ \ }\mathrm{\text{and}}\mathrm{\ }\ Var(x) = \frac{\alpha}{\lambda^{2}}.
\]
\end{theorem}

\begin{proof}
\\[4cm]\vspace{0.01cm}
\end{proof}

\begin{theorem}{Linear Generated Gamma R.V.}{Linear Generated Gamma R.V.}
Suppose $X\sim\mg(\alpha,\lambda)$, where $\alpha,\lambda>0$, and $Y=aX$, where $a>0$. Then
\[Y\sim\mg\left( \alpha,\frac{\lambda}{a} \right).\]
\end{theorem}

\begin{proof}
\\[4cm]\vspace{0.01cm}
\end{proof}

\begin{theorem}{Gamma R.V. from Normal R.V.}{Gamma R.V. from Normal R.V.}
Suppose $X\sim N(\mu,\sigma^2)$, where $\mu,\sigma\in\mathbb{R}$, $\sigma\neq0$ and $Y={(X-\mu)}^2$. Then
\[Y\sim\mg\left( \frac{1}{2},\frac{1}{2\sigma^{2}} \right).\]
\end{theorem}

\begin{proof}
\\[4cm]\vspace{0.01cm}
\end{proof}

\begin{lemma}{Plus to Multiply Property of Exponential Function}{Plus to Multiply Property of Exponential Function}
Suppose $f:[0,+\infty)\rightarrow\mathbb{R}$ is right continuous on $[0,+\infty)$ and $f(x+y)=f(x)\cdot f(y),\ \forall x,y\gs0$. Then there either $f(x)=0,\ \forall x\gs0$ or $\exists\lambda\in\mathbb{R}$ s.t. $f(x)=e^{-\lambda x},\ \forall x\gs0$.
\end{lemma}

\begin{proof}
\\[4cm]\vspace{0.01cm}
\end{proof}

\begin{theorem}{Memoryless Property}{Memoryless Property}
Suppose $X$ is a nonnegative continuous r.v. of a probability space $(\Omega,\ma, P)$. Then $P(x>s+t| x>s)=P(x>t),\ \forall s,t>0\ \Leftrightarrow X\sim\me(\lambda)$, for some $\lambda>0$.
\end{theorem}

\begin{proof}
\\[4cm]\vspace{0.01cm}
\end{proof}

\begin{remark}{Analog of Geometric R.V.}{Analog of Geometric R.V.}
Exponential r.v.'s are the continuous analog of geometric r.v.'s.
\end{remark}

\begin{theorem}{Geometric R.V. from Exponential R.V.}{Geometric R.V. from Exponential R.V.}
Suppose $X\sim\me(\lambda)$ where $\lambda>0$ and $Y=\left\lceil X\right\rceil$. Then $Y\sim\mathrm{geometric}\left(1-e^{-\lambda}\right)$.
\end{theorem}

\begin{proof}
\\[4cm]\vspace{0.01cm}
\end{proof}

\begin{definition}{Independent Set}{Independent Set}
A set of r.v.'s $\{X_i:i\in I\}$ of a probability space $(\Omega,\ma, P)$ is called independent, if for any finite subset $\left\{X_{i_1},X_{i_2},\ \cdots,X_{i_k}\right\},\ k\gs2$ of $\{X_i:i\in I\}$ the events
$$
X_{i_1}\in B_1,\ X_{i_2}\in B_2,\ \cdots,X_{i_k}\in B_k
$$
are independent for all $B_1,B_2,\ \cdots,B_k\in\mb_\mathbb{R}$.\\
Otherwise, $\{X_i:i\in I\}$ is called dependent.
\end{definition}

\begin{definition}{Continuous R.Vect.}{Continuous R.Vect.}
A r.vect. $\mX=(X_1,\ X_2,\ \cdots,X_n)$ of a probability space $(\Omega,\ma, P)$ is called an absolute continuous r.vect. (or continuous r.vect.) if there exists a nonnegative real-valued function $f_{\mX}:\mathbb{R}^n\rightarrow[0,\infty)$ s.t.
$$
P(X_1\in B_1,X_2\in B_2,\ \cdots,X_n\in B_k)=\int_{B_1}\int_{B_2}{\cdots\int_{B_n}{f_{\mX}(\mx)\dif x_n}}\cdots \dif x_2\dif x_1
$$
for all $B_1,B_2,\ \cdots,B_n\in\mb_\mathbb{R}$.\\
Then the function $f_{\mX}$ is called the p.d.f. of the r.vect. $\mX$, or the joint p.d.f. of the r.v.'s $X_1,\ X_2,\ \cdots,X_n$.
\end{definition}

\begin{theorem}{P.D.F. and C.D.F. of Continuous R.Vect.}{P.D.F. and C.D.F. of Continuous R.Vect.}
Suppose $\mX=(X_1,\ X_2,\ \cdots,X_n)$ is a continuous r.vect. and 
$$
F_{\mX}(\mx)=P(X_1\ls x_1,X_2\ls x_2,\ \cdots,X_n\ls x_n).
$$
Then
$$
f_{\mX}(\mx)=\frac{\partial F_{\mX}(\mx)}{\partial x_1\cdots\partial x_n}.
$$
Furthermore, if $X_1,\ X_2,\ \cdots,X_n$ are independent, then
$$
f_{\mX}(\mx)=f_{X_1}(x)f_{X_2}(x)\cdots f_{X_n}(x).
$$
\end{theorem}

\begin{proof}
\\[4cm]\vspace{0.01cm}
\end{proof}

\begin{theorem}{Convolution Theorem}{Convolution Theorem}
If $\mX=(X_1,X_2)$ is a continuous r.vect. and $Y=X_1+X_2$. Then
$$
f_Y(y)=\int_{-\infty}^{\infty}{f_{X_1,X_2}\left(x,y-x\right)}\dif x.
$$
Furthermore, if $X_1\bot X_2$, then
$$
f_Y(y)=\int_{-\infty}^{\infty}{f_{X_1}(x)}f_{X_2}\left(y-x\right)\dif x.
$$
\end{theorem}

\begin{proof}
\\[4cm]\vspace{0.01cm}
\end{proof}

\begin{definition}{Beta Function}{Beta Function}
The function $B:\ \mathbb{R}^+\times\mathbb{R}^+\rightarrow\mathbb{R}$ is given by
$$
B(\alpha,\beta)=\int_{0}^{1}x^{\alpha-1}{(1-x)}^{\beta-1}\dif x,\ \forall\alpha,\beta>0
$$
is called beta function.
\end{definition}

\begin{lemma}{Calculation of Beta Function}{Calculation of Beta Function}
\[B(\alpha,\beta) = \frac{\Gamma( \alpha ) \cdot \Gamma( \beta )}{\Gamma( \alpha + \beta)},\ \forall\alpha,\beta > 0.\]
\end{lemma}

\begin{proof}
\\[4cm]\vspace{0.01cm}
\end{proof}

\begin{theorem}{Independent Additivity of Gamma R.V.}{Independent Additivity of Gamma R.V.}
Suppose $X_i\sim\mg(\alpha_i,\lambda)$ where $\alpha_i,\lambda>0$, $i=1,2,\ \cdots,n,$ ${X}_1,\ X_2,\ \cdots,X_n$ are independent, and $Y=X_1+X_2+\cdots+X_n$. Then
\[Y\sim\mg\left( \sum_{i = 1}^{n}\alpha_{i},\lambda \right).\]
\end{theorem}

\begin{proof}
\\[4cm]\vspace{0.01cm}
\end{proof}

\begin{theorem}{Independent Minimum of Exponential R.V.}{Independent Minimum of Exponential R.V.}
Suppose $X_i\sim\me(\lambda_i)$ where $\lambda_i>0,\ i=1,2,\ \cdots,n$, and ${X}_1,X_2,\ \cdots,X_n$ are independent.\\
(1) If $Y=\min\{X_1,X_2,\ \cdots,X_n\}$, then
\[Y\sim\me\left( \sum_{i = 1}^{n}\lambda_{i} \right).\]
(2)
$$
P(X_1<X_2)=\frac{\lambda_1}{\lambda_1+\lambda_2}.
$$
\end{theorem}

\begin{proof}
\\[4cm]\vspace{0.01cm}
\end{proof}

\begin{definition}{Stochastic Process}{Stochastic Process}
A stochastic process (s.p.) $\{X(t):i\in I\}$ is a collection of r.v.'s of a probability space $(\Omega,\ma, P)$. If $I=\{0,1,2,\cdots\}$ or $\{0,\pm1,\pm2,\cdots\}$, then we call $\{X(t):i\in I\}$ a discrete-time S.P. If $I=[0,\infty)$ or $(-\infty,\infty)$, then we call $\{X(t):i\in I\}$ a continuous-time S.P.
\end{definition}

\begin{definition}{Counting Process and Poisson Process}{Counting Process and Poisson Process}
Let $\{T_1,T_2,\cdots\}$ be a discrete-time S.P. s.t. $T_i,\ i=1,2,\cdots$, is the time of occurrence of the $i^{th}$ event, and $0<T_1<T_2<\cdots$.\\

Let $X_i=T_i-T_{i-1},\ i=1,2,\cdots$, where $T_0=0$ be the inter-occurrence time between the ${(i-1)}^{th}$ and the $i^{th}$ events, and $N(t)=|\{i:0<T_i\ls t\}|$ be the number of events occurring in $(0,t]$, so that $\{N(t):0<t<\infty\}$ is called the counting process of the S.P. $\{T_1,T_2,\cdots\}$.\\

Then we call $\{T_1,T_2,\cdots\}$ a Poisson process with rate $\lambda$, if $X_1,X_2,\cdots$ are independent and identically distributed (i.i.d.) and $N(t)\sim\mathrm{Poisson}(\lambda t)$.
\end{definition}

\begin{theorem}{Necessary and Suf\/f\/icient Condition of Poisson Process}{Necessary and Sufficient Condition of Poisson Process}
Suppose $\{T_1,T_2,\cdots\}$ is a S.P. s.t. $0<T_1<T_2<\cdots$ and its inter-occurrence times $X_i=T_i-T_{i-1},\ i=1,2,\cdots$ are i.i.d., where $T_0=0$. Then $\{T_1,T_2,\cdots\}$ is a Poisson process with rate $\lambda\ \Leftrightarrow X_i\sim\me(\lambda),\ i=1,2,\cdots.$
\end{theorem}

\begin{proof}
\\[4cm]\vspace{0.01cm}
\end{proof}

\begin{remark}{Negative Binomial $\leftrightarrow$ Geometric vs Gamma$ \leftrightarrow$ Exponential}{Negative Binomial Geometric Exponential}
(1) A negative binomial r.v. $T_r=X_1+\ X_2+\cdots+X_r\sim\text{neg.-binomial}(r,p)$ is the number of i.i.d. Bernoulli trials with the same probability of success $p$ until the $r^{th}$ success occurs, where $X_i\sim\mathrm{geometric}(p)$ is the number of Bernoulli trials between the ${(i-1)}^{th}$ and the $i^{th}$ successes, and $X_1,X_2,\cdots$ are independent.\\

(2) A gamma r.v. $T_n=X_1+X_2+\cdots+X_n\sim\mg(n,\lambda)$ is the time of occurrence of the $n^{th}$ event of a Poisson process with rate $\lambda$, where $X_i\sim\me(\lambda)$ is the inter-occurrence time between the ${(i-1)}^{th}$ and the $i^{th}$ events, and $X_1,X_2,\cdots$ are independent.
\end{remark}

\begin{theorem}{Merging and Splitting of Poisson Process}{Merging and Splitting of Poisson Process}
(1) Suppose that $k$ independent Poisson processes with rates $\lambda_1,\lambda_2,\ \cdots,\lambda_k$ are merged into a S.P. $\{T_1,T_2,\cdots\}$. Then $\{T_1,T_2,\cdots\}$ is a Poisson process with rate $\lambda=\lambda_1+\lambda_2+\cdots+\lambda_k$.\\

(2) Suppose that in a Poisson process with rate $\lambda$, an event is a type-$i$ event with probability $P_i,\ i=1,2,\ \cdots,k.$ Then the S.P. $\{T_1,T_2,\cdots\}$ of the times of the occurrences of the type-$i$ events is a
 Poisson process with rate $\lambda\cdot P_i,\ i=1,2,\ \cdots,k.$
\end{theorem}

\begin{proof}
\\[4cm]\vspace{0.01cm}
\end{proof}

\section{Beta R.V.'s}

\begin{definition}{Beta R.V.}{Beta R.V.}
A continuous r.v. $X$ of a probability space $(\Omega,\ma, P)$ is called a beta r.v. with parameter $\alpha$ and $\beta$, where $\alpha,\beta>0$, denoted $X\sim\mb(\alpha,\beta)$, if its p.d.f. is given by
\[f_{X}(x) = \left\{ \begin{aligned}
\frac{1}{B(\alpha,\beta)}&x^{\alpha - 1}{(1 - x)}^{\beta - 1},\ \mathrm{\text{if}}\ \ 0 < x < 1 \\
&0, \qquad\mathrm{\text{o.w.}} \\
\end{aligned} \right.\]
where
\[B(\alpha,\beta) = \int_{0}^{1}x^{\alpha - 1}{(1 - x)}^{\beta - 1}\dif x = \frac{\Gamma( \alpha ) \cdot \Gamma( \beta )}{\Gamma( \alpha + \beta )}.\]
\end{definition}

\begin{remark}{P.D.F. and C.D.F.}{P.D.F. and C.D.F.9}
(1) $\int_{-\infty}^{\infty}{f_X(x)\dif x}=1\ \Rightarrow\ f_X(x)$ is a p.d.f.
\vspace{0.4cm}\\
(2) Beta r.v.'s are good approximations of r.v.'s that vary between two limits.\vspace{0.2cm}\\
(3) If $X_1,X_2,\ \cdots,X_n$ are i.i.d. $\sim U(0,1)$ and ${X}_{(i)}$ is the $i^{th}$ smallest r.v. of $X_1,X_2,\ \cdots,X_n$ so that $X_{(1)}\ls{X}_{(2)}\ls\cdots\ls X_{(n)}$, then
$$
{X}_{(i)}\sim\mb\left(i,n+1-i\right).
$$
(4)
$$\begin{aligned}
f_{X}^{'}(x) &= \frac{\left( \alpha - 1 \right)x^{\alpha - 2}\left( 1 - x \right)^{\beta - 1} - (\beta - 1)x^{\alpha - 1}{(1 - x)}^{\beta - 2}}{B(\alpha,\beta)}\\
&= \frac{x^{\alpha - 2}{(1 - x)}^{\beta - 2}}{B(\alpha,\beta)}\left\lbrack \left( \alpha - 1 \right) - \left( \alpha + \beta - 2 \right)x \right\rbrack\\
&\Rightarrow f_{X}^{'}(x)\left\{ \begin{aligned}
 &> 0,\ \Leftrightarrow \left( \alpha + \beta - 2 \right)x < \alpha - 1 \\
 &= 0,\ \Leftrightarrow \left( \alpha + \beta - 2 \right)x = \alpha - 1 \\
 &< 0,\ \Leftrightarrow \left( \alpha + \beta - 2 \right)x > \alpha - 1 \\
\end{aligned} \right.
\end{aligned}$$
$$\begin{aligned}
&f_{X}^{''}(x)\\
&= \frac{\left( \alpha - 1 \right)\left( \alpha - 2 \right)x^{\alpha - 3}\left( 1 - x \right)^{\beta - 1} - (\beta - 1)(\beta - 2)x^{\alpha - 1}{(1 - x)}^{\beta - 3}}{B(\alpha,\beta)}\\
&= \frac{x^{\alpha - 3}{(1 - x)}^{\beta - 3}}{B(\alpha,\beta)}\cdot h(x,\alpha,\beta)\\
&= \left\{ \begin{aligned}
&\frac{x^{\alpha - 3}\left( 1 - x \right)^{\beta - 3}}{B(\alpha,\beta)}\left( \alpha + \beta - 2 \right)\left( \alpha + \beta - 3 \right)\cdot f(x,\alpha,\beta),\alpha + \beta \neq 2,3 \\
&\frac{x^{\alpha - 3}\left( 1 - x \right)^{\beta - 3}}{B(\alpha,\beta)} \cdot 2 \cdot \left( \alpha - 1 \right) \cdot \left( x - \frac{\alpha - 2}{2} \right),\alpha + \beta = 2\\
&\frac{x^{\alpha - 3}\left( 1 - x \right)^{\beta - 3}}{B(\alpha,\beta)} \cdot \left( \alpha - 1 \right) \cdot \left( \alpha - 2 \right), \alpha + \beta = 3\\
\end{aligned} \right.
\end{aligned}$$
where
$$
h(x,\alpha,\beta)=\left( \alpha + \beta - 2 \right)\left( \alpha + \beta - 3 \right)x^{2} - 2\left( \alpha - 1 \right)\left( \alpha + \beta - 3 \right)x + \left( \alpha - 1 \right)\left( \alpha - 2 \right),
$$
and
$$
f(x,\alpha,\beta)=\left( x - \frac{\alpha - 1}{\alpha + \beta - 2} \right)^{2} - \frac{\left( \alpha - 1 \right)\left( \beta - 1 \right)}{\left( \alpha + \beta - 2 \right)^{2}\left( \alpha + \beta - 3 \right)}.
$$
\end{remark}

\begin{theorem}{Expectation and Variance of Beta R.V.}{Expectation and Variance of Beta R.V.}
Suppose $X\sim\mb(\alpha,\beta)$, then
\[E\lbrack X^{n}\rbrack = \frac{( \alpha )_{n}}{\left( \alpha + \beta \right)_{n}} = \dfrac{ \binom{\alpha + n - 1}{n} }{ \binom{\alpha + \beta + n - 1}{n} }.\]
Therefore,
\[E\lbrack X\rbrack = \frac{\alpha}{\alpha + \beta}\]
and
\[Var(x) = \frac{\alpha\beta}{(\alpha + \beta + 1){(\alpha + \beta)}^{2}}.\]
\end{theorem}

\begin{proof}
\\[4cm]\vspace{0.01cm}
\end{proof}

\begin{theorem}{Beta R.V. vs Binomial R.V.}{Beta R.V. vs Binomial R.V.}
Suppose $X\sim\mb(\alpha,\beta)$, and $Y\sim\mathrm{binomial}(\alpha+\beta-1,p)$, where $\alpha,\beta\in\mathbb{Z}^+,\ 0<p<1$. Then
$$
P(X\ls p)=P(Y\gs\alpha)
$$
and
$$
P(X\gs p)=P(Y\ls\alpha-1).
$$
\end{theorem}

\begin{proof}
\\[4cm]\vspace{0.01cm}
\end{proof}

\chapter{Bivariate and Multivariate Distributions}

\section{Joint Distributions of Two or More R.V.'s}

\begin{definition}{Joint P.M.F. of Multiple R.v.'s}{Joint P.M.F. of Multiple R.v.'s}
Let $X_1,\ X_2,\ \cdots,X_n$ be discrete r.v.'s of a probability space $(\Omega,\ma, P)$. The nonnegative function $P_X:\mathbb{R}^n\rightarrow[0,1]$ given by 
\[p_{\mX}(\mx) = P_{\mX}\left( \left\{ \mx\right\} \right) = P(\mX =\mx ) = \left\{ \begin{aligned}
&P(\mX =\mx ),\ \mx\in\mX(\Omega) \\
&\qquad0,\quad\mx\in \mathbb{R}^{n}\backslash\mX(\Omega) \\
\end{aligned} \right.\]
is called the joint p.m.f. of $X_1,\ X_2,\ \cdots,X_n$.
\end{definition}

\begin{remark}{Properties of Joint P.M.F.}{Properties of Joint P.M.F.}
(1)
\[p_{\mX}(\mx) \gs 0,\ \forall \mx\in \mX( \Omega )\ \mathrm{\text{and}}\mathrm{\ }\ p_{\mX}(\mx) = 0,\ \forall\mx \in \mathbb{R}^{n}\backslash \mX(\Omega).\]
(2)
\[\sum_{\mx\in \mX(\Omega)}{p_{\mX}(\mx)} = \sum_{\mx\in \mX(\Omega)}{P(\mX=\mx)} = P(\mX \in \mX(\Omega) ) = P( \Omega ) = 1\]
(3)
\[\mX(\Omega) \subseteq \prod_{i = 1}^{n}{X_{i}(\Omega)}\]
(4)
\[p_{\mX}(\mx) = \left\{ \begin{aligned}
&P(\mX =\mx),\ \mx\in\prod_{i = 1}^{n}X_{i}(\Omega) \\
&\qquad0,\qquad\mx  \in \mathbb{R}^{n}\backslash\prod_{i = 1}^{n}X_{i}( \Omega) \\
\end{aligned} \right.\]
\end{remark}

\begin{theorem}{Joint Marginal P.M.F.}{Joint Marginal P.M.F.}
Suppose $X_1,X_2,\ \cdots,X_n$ are discrete r.v.'s of a probability space $(\Omega,\ma, P)$. Then 
\[p_{X_{i_{1}},X_{i_{2}},\ \cdots,X_{i_{k}}}\left( x_{i_{1}},x_{i_{2}},\ \cdots,x_{i_{k}} \right) = \left\{ \begin{aligned}
\sum_{\mbox{\tiny$\begin{aligned}
&x_{i} \in X_{i}\left( \Omega \right) \\
i &\neq i_{1},i_{2},\cdots\ ,i_{k} \\
\end{aligned}$}} & {p_{X_i}\left( x_i \right),\ \forall i = i_{1},i_{2},\ \cdots,i_{k}} \\
&0,\qquad\mathrm{\text{o.w.}} \\
\end{aligned} \right.\]
We call $$p_{X_{i_1},X_{i_2},\ \cdots,X_{i_k}}\left(x_{i_1},x_{i_2},\ \cdots,x_{i_k}\right)$$ the joint p.m.f. marginalized over $X_{i_1},X_{i_2},\ \cdots,X_{i_k}$.  If $k=1$, we call $p_{X_i}(x_i)$ the marginal p.m.f. of $X_i$.
\end{theorem}

\begin{proof}
\\[4cm]\vspace{0.01cm}
\end{proof}

\begin{theorem}{Expectation of Measurable Function}{Expectation of Measurable Function}
Suppose $X_1,\ X_2,\ \cdots,X_n$ are discrete r.v.'s of a probability space $(\Omega,\ma, P)$, and $g$ is a measurable function from $(\mathbb{R}^n,\ \ \mb_{\mathbb{R}^n})$ to $(\mathbb{R},\ \mb_\mathbb{R})$. Then
\[E\left\lbrack g(\mx) \right\rbrack = \sum_{\mbox{\tiny$\begin{aligned}
x_{i} \in X_{i}\left( \Omega \right) \\
i = 1,2,\ \cdots,n \\
\end{aligned}$}}^{}{g(\mx) \cdot p_{\mX}(\mx).}\]
\end{theorem}

\begin{proof}
\\[4cm]\vspace{0.01cm}
\end{proof}

\begin{corollary}{Expectation of Linear Combined Measurable Function}{Expectation of Linear Combined Measurable Function}
Suppose $X_1,\ X_2,\ \cdots,X_n$ are discrete r.v.'s of a probability space $(\Omega,\ma, P)$, and $g_1,g_2,\ \cdots,g_m$ are measurable functions from $(\mathbb{R}^n,\mb_{\mathbb{R}^n})$ to $(\mathbb{R},\mb_\mathbb{R})$, and $\alpha_1,\alpha_2,\ \cdots,\alpha_m\in\mathbb{R}$,
Then $$\sum_{k=1}^{m}\alpha_k\cdot g_k(\mx)$$ is a discrete r.v. of $(\Omega,\ma, P)$ and 
\[E\left\lbrack \sum_{k = 1}^{m}\alpha_{k}g_{k}(\mx) \right\rbrack = \sum_{k = 1}^{m}\alpha_{k}E\left\lbrack g_{k}(\mx) \right\rbrack.\]
\end{corollary}

\begin{proof}
\\[4cm]\vspace{0.01cm}
\end{proof}

\begin{definition}{Joint P.D.F.}{Joint P.D.F.}
Let $X_1,\ X_2,\ \cdots,X_n$ be r.v.'s of a probability space $(\Omega,\ma, P)$. We say that $X_1,\ X_2,\ \cdots,X_n$ are jointly continuous r.v.'s if there exists a nonnegative function $f_{\mX}:\mathbb{R}^n\rightarrow[0,1]$ s.t.
$$
P(\mX\in B)=\int\int_{B}\cdots\int f_{\mX}(\mx)\dif \mx,\ \ \forall B\in\mb_{\mathbb{R}^n}.
$$
The function $f_{\mX}$ is called the joint p.d.f. of $X_1,\ X_2,\ \cdots,X_n$.
\end{definition}

\begin{theorem}{Joint Marginal P.D.F.}{Joint Marginal P.D.F.}
Suppose $X_1,\ X_2,\ \cdots,X_n$ are jointly continuous r.v.'s of a probability space $(\Omega,\ma, P)$. Then $X_{i_1},X_{i_2},\ \cdots,X_{i_k}$ are also jointly continuous r.v.'s of a probability space $(\Omega,\ma, P)$ with joint p.d.f.
$$
f_{X_{i_1},X_{i_2},\ \cdots,X_{i_k}}\left(x_{i_1},x_{i_2},\ \cdots,x_{i_k}\right)=\undercbrace{\int_{-\infty}^{\infty}\int_{-\infty}^{\infty}\cdots\int_{-\infty}^{\infty}}_{n-k}{f_{\mX}(\mx)\dif x_i}
$$
where $i\neq i_1,i_2,\cdots i_k$.\\
We call $$f_{X_{i_1},X_{i_2},\ \cdots,X_{i_k}}\left(x_{i_1},x_{i_2},\ \cdots,x_{i_k}\right)$$ the joint p.d.f. marginalized over $X_{i_1},X_{i_2},\ \cdots,X_{i_k}$. If $k=1$, we call $f_{X_i}(x_i)$ the marginal p.d.f. of $X_i$.
\end{theorem}

\begin{proof}
\\[4cm]\vspace{0.01cm}
\end{proof}

\begin{theorem}{Expectation of Measurable Function}{Expectation of Measurable Function}
Suppose $X_1,\ X_2,\ \cdots,X_n$ are jointly continuous r.v.'s of a probability space $(\Omega,\ma, P)$, and $g$ is a measurable function from $(\mathbb{R}^n,\ \mb_{\mathbb{R}^n})$ to $(\mathbb{R},\mb_\mathbb{R})$. Then
$$
E\left[g(\mx)\right]=\int_{-\infty}^{\infty}\int_{-\infty}^{\infty}{\cdots\int_{-\infty}^{\infty}{g(\mx)f_{\mX}(\mx)\dif x_n\cdots \dif x_2\dif x_1}}.
$$
\end{theorem}

\begin{proof}
\\[4cm]\vspace{0.01cm}
\end{proof}

\begin{remark}{Properties of Joint P.D.F.}{Properties of Joint P.D.F.}
(1)
\[f_{\mX}(\mx) > 0,\ \forall\mx \in \mathbb{R}^{n}.\]
(2)
\[\int_{- \infty}^{\infty}{\int_{- \infty}^{\infty}{\cdots\int_{- \infty}^{\infty}{f_{\mX}(\mx)\dif\mx}}} = P(\mX \in \mathbb{R}^{n}) = 1.\]
(3)
\[P( X_{1} \in B_{1},X_{2} \in B_{2},\ \cdots,X_{n} \in B_{n}) = \int_{B_{1}}^{}{\int_{B_{2}}^{}{\cdots\int_{B_{n}}^{}{f_{\mX}(\mx)\dif x_{n}\cdots \dif x_{2}}}}\dif x_{1},\]
$\forall B_{i} \in \mb_{\mathbb{R}^{n}},\ i = 1,2,\ \cdots,n$.\\
(4)
\[P(\mX = \mathbf{a}) = \int_{a_{1}}^{a_{1}}{\int_{a_{2}}^{a_{2}}{\cdots\int_{a_{n}}^{a_{n}}{f_{\mX}(\mx)\dif x_{n}\cdots \dif x_{2}}}}\dif x_{1} = 0.\]
(5)
\[\begin{aligned}
&\quad\  P(a_{i} \ls X_{i} \ls a_{i} + \delta_{i},\ i = 1,2,\ \cdots,n)\\
&= \int_{a_{1}}^{a_{1} + \delta_{1}}{\int_{a_{2}}^{a_{2} + \delta_{2}}{\cdots\int_{a_{n}}^{a_{n} + \delta_{n}}{f_{\mX}(\mx)\dif x_{n}\cdots \dif x_{2}}}}\dif x_{1}\\
&= f_{\mX}( {\mathbf{a}}_{\bm{\delta}} ) \cdot \delta_{1} \cdot \delta_{2} \cdots\delta_{n} \text{ for some }{\mathbf{a}}_{\bm{\delta}} \in \prod_{i = 1}^{n}{\lbrack a_{i},a_{i} + \delta_{i}\rbrack} \text{ if } f_{\mX}(\mx) \text{ is continuous}.\\
&\Rightarrow \lim_{\bm{\delta}\rightarrow\mathbf{0}}\frac{P(a_{i} \ls X_{i} \ls a_{i} + \delta_{i},\ i = 1,2,\ \cdots,n)}{\delta_{1} \cdot \delta_{2} \cdots \delta_{n}} = \lim_{\bm{\delta}\rightarrow\mathbf{0}}{f_{\mX}( {\mathbf{a}}_{\bm{\delta}} )} = f_{\mX}(\mathbf{a})\\
&\ \text{and } P(a_{i} \ls X_{i} \ls a_{i} + \delta_{i},\ i = 1,2,\ \cdots,n) \approx f_{\mX}(\mathbf{a}) \cdot \delta_{1} \cdot \delta_{2} \cdots\delta_{n}.\\
\end{aligned}\]
\end{remark}

\begin{corollary}{Expectation of Linear Combined Measurable Function}{Expectation of Linear Combined Measurable Function}
Suppose \(X_{1},\ X_{2},\ \cdots,X_{n}\) are \textbf{jointly continuous}
r.v.'s of a probability space $(\Omega,\ma, P)$ and \(g_{1},\ g_{2},\ \cdots, g_{m}\) are \textbf{measurable functions} from \((\mathbb{R}^{n}, \mb_{\mathbb{R}^{n}})\) to \(\mathbb{(R,}\mb_{\mathbb{R}})\), and \(\alpha_{1},\alpha_{2},\ \cdots,\alpha_{m} \in \mathbb{R}\), then
\[\sum_{k = 1}^{m}\alpha_{k}\cdot g_{k}(\mx)
\]
is a continuous r.v. of $(\Omega,\ma, P)$ and 
\[E\left\lbrack \sum_{k = 1}^{m}\alpha_{k}\cdot g_{k}(\mx) \right\rbrack = \sum_{k = 1}^{m}\alpha_{k}\cdot E\left\lbrack g_{k}(\mx) \right\rbrack.\]
\end{corollary}

\begin{proof}
\\[4cm]\vspace{0.01cm}
\end{proof}

\begin{definition}{Joint C.D.F.}{Joint C.D.F.}
Let \(X_{1},\ X_{2},\ \cdots,X_{n}\) be r.v.'s of a probability space $(\Omega,\ma, P)$. The \textbf{joint c.d.f.} of \(X_{1},\ X_{2},\ \cdots,X_{n}\) is given by
\[F_{\mX}(\mx) = P( X_{1} \ls x_{1},X_{2} \ls x_{2},\ \cdots,X_{n} \ls x_{n}),\ \forall\mx \in \mathbb{R}^{n}.\]
\end{definition}

\begin{theorem}{Joint Marginal C.D.F.}{Joint Marginal C.D.F.}
Suppose \(X_{1},\ X_{2},\ \cdots,X_{n}\) are r.v.'s of a probability space $(\Omega,\ma, P)$. Then
\[\begin{aligned}
&F_{X_{i_{1}},X_{i_{2}},\cdots\ ,X_{i_{k}}}\left( x_{i_{1}},x_{i_{2}},\ \cdots,x_{i_{k}} \right) \\
=& F_{\mX}\left( \infty,\, \cdots,\infty,x_{i_{1}},\infty,\ \cdots,\infty,x_{i_{2}},\infty,\ \cdots,\infty,x_{i_{k}},\infty,\ \cdots,\infty \right)
\end{aligned}\]
We call \[F_{X_{i_{1}},X_{i_{2}},\ \cdots,X_{i_{k}}}\left( x_{i_{1}},x_{i_{2}},\ \cdots,x_{i_{k}} \right)\] the \textbf{joint c.d.f.} marginalized over \(X_{1},\ X_{2},\ \cdots,X_{n}\). If \(k = 1\), we call \(F_{X_{i}}(x_{i})\) the \textbf{marginal c.d.f.} of \(X_{i}\).
\end{theorem}

\begin{proof}
\\[4cm]\vspace{0.01cm}
\end{proof}

\begin{theorem}{Properties of Joint C.D.F.}{Properties of Joint C.D.F.}
Suppose \(X_{1},\ X_{2},\ \cdots,X_{n}\) are r.v.'s of a probability
space $(\Omega,\ma, P)$.\\
(1) \(F_{\mX}(\mx)\) is \textbf{increasing} and \textbf{right
continuous} in each argument \(x_{i},i = 1,2,\ \cdots,n.\)\\
(2) \(F_{\mX}(\mx) = 0\) if there exists at least one \(i\) such
that \(x_{i} = - \infty\).\\
(3) \(F_{\mX}\left( \infty,\infty,\ \cdots,\infty \right) = 1\).\\
(4) If \(X_{1},\ X_{2},\ \cdots,X_{n}\) are \textbf{jointly continuous}
r.v.'s, then
\[f_{\mX}(\mx) = \frac{\partial F_{\mX}(\mx)}{\partial x_{1}\partial x_{2}\cdots\partial x_{n}},\ \forall\mx \in \mathbb{R}^{n}.\]
\end{theorem}

\begin{proof}
\\[4cm]\vspace{0.01cm}
\end{proof}

\section{Independent R.V.'s}

\begin{definition}{Independent Set}{Independent Set}
Let \(\{X_{i},\ i \in I\}\) be r.v.'s of a probability space $(\Omega,\ma, P)$. We say that the r.v.'s \(\{X_{i},\ i \in I\}\) are \textbf{independent} if for any finite subset
\(\left\{ X_{i_{1}},X_{i_{2}},\ \cdots,X_{i_{k}} \right\}\ (k \gs 2)\) of \(\{ X_{i},\ i \in I\}\), the events \(X_{i_{1}} \in B_{i_{1}},X_{i_{2}} \in B_{i_{2}},\ \cdots,X_{i_{k}} \in B_{i_{k}}\) are independent \(\forall B_{i_{1}},B_{i_{2}},\ \cdots,B_{i_{k}} \in \mb_{\mathbb{R}}.\) Otherwise, the r.v.'s \(\{X_{i},\ i \in I\}\) are dependent.
\end{definition}

\begin{theorem}{Equivalent Statements of Independence}{Equivalent Statements of Independence}
Suppose \(X_{1},\ X_{2},\ \cdots,X_{n}\) are r.v.'s of a probability space $(\Omega,\ma, P)$. The following three statements are \textbf{equivalent}:\\
(1) \(X_{1},\ X_{2},\ \cdots,X_{n}\) are independent.\\
(2)
\[P( X_{1} \in B_{1},X_{2} \in B_{2},\ \cdots,X_{n} \in B_{n}) = \prod_{i = 1}^{n}{P(X_{i} \in B_{i})},\forall B_{1},B_{2},\ \cdots,B_{n} \in \mb_{\mathbb{R}}\]
(3)
\[F_{\mX}(\mx) = \prod_{i = 1}^{n}{F_{X_{i}}(x_{i})},\ \forall \mx\in \mathbb{R}^{n}\]
\end{theorem}

\begin{proof}
\\[4cm]\vspace{0.01cm}
\end{proof}

\begin{theorem}{Necessary and Suf\/f\/icient Condition of Independence}{Necessary and Sufficient Condition of Independence}
Suppose \(X_{1},\ X_{2},\ \cdots,X_{n}\) are r.v.'s of a probability
space $(\Omega,\ma, P)$.\\
(1) If \(X_{1},\ X_{2},\ \cdots,X_{n}\) are \textbf{discrete} r.v.'s, then \(X_{1},\ X_{2},\ \cdots,X_{n}\) are independent
\[\Leftrightarrow P_{\mX}(\mx) = \prod_{i = 1}^{n}{P_{X_{i}}(x_{i})},\ \forall\mx \in \mathbb{R}^{n}\]
(2) If \(X_{1},\ X_{2},\ \cdots,X_{n}\) are \textbf{jointly continuous}
r.v.'s, then \(X_{1},\ X_{2},\ \cdots,X_{n}\) are independent
\[\Leftrightarrow f_{\mX}(\mx) = \prod_{i = 1}^{n}{f_{X_{i}}(x_{i})},\ \forall \mx\in \mathbb{R}^{n}\]
\end{theorem}

\begin{proof}
\\[4cm]\vspace{0.01cm}
\end{proof}

\begin{definition}{Indicator Function}{Indicator Function}
Let $(\Omega,\ma, P)$ be a probability space, and \(A \in \ma\). The \textbf{indicator function} \(I_{A}\) of the event \(A\) is given by
\[I_{A}(w) = \left\{ \begin{aligned}
&1,\ \mathrm{\text{\ if}}\ w \in A \\
&0,\quad \mathrm{\text{o.w.}} \\
\end{aligned} \right.\quad \mathrm{\text{i.e.}}\quad I_{A} = \left\{ \begin{aligned}
&1,\ \mathrm{\text{ if }} A \ \mathrm{\text{occurs}} \\
&0,\quad \mathrm{\text{o.w.}} \\
\end{aligned} \right.\]
\end{definition}

\begin{theorem}{Indicator Function is a Discrete Measurable Function}{Indicator Function is a Discrete Measurable Function}
Suppose $(\Omega,\ma, P)$ is a probability space. \(I_{A}\) is a \textbf{discrete r.v.} of $(\Omega,\ma, P)$ for all \(A \in \ma\).
\end{theorem}

\begin{proof}
\\[4cm]\vspace{0.01cm}
\end{proof}

\begin{theorem}{Indicator R.V.'s Indicates Independence}{Indicator R.V.'s Indicates Independence}
Suppose $(\Omega,\ma, P)$ is a probability space, and \(A_{1},A_{2},\ \cdots,A_{n} \in \ma\). The events \(A_{1},A_{2},\ \cdots,A_{n}\) are \textbf{independent} \(\Leftrightarrow\) the \textbf{indicator r.v.'s}
\(I_{A_{1}},I_{A_{2}},\ \cdots,I_{A_{n}}\) are \textbf{independent}.
\end{theorem}

\begin{proof}
\\[4cm]\vspace{0.01cm}
\end{proof}

\begin{theorem}{Expectation of Measurable Functions of Independent R.V.}{Expectation of Measurable Functions of Independent R.V.}
Suppose \(X_{1},\ X_{2},\ \cdots,X_{n}\) are independent r.v.'s of a
probability space $(\Omega,\ma, P)$, and \(g_{1},g_{2},\ \cdots,g_{n}\) are measurable functions from \(\mathbb{(R,}\mb_{\mathbb{R}})\) to
\(\mathbb{(R,}\mb_{\mathbb{R}})\). Then \(g_{1}(x_{1}),g_{2}(x_{2}),\ \cdots,g_{n}(X_{n})\) are independent and
\[E\left\lbrack \prod_{i = 1}^{n}{g_{i}(x_{i})} \right\rbrack = \prod_{i = 1}^{n}{E\lbrack g_{i}(x_{i})\rbrack}.\]
\end{theorem}

\begin{proof}
\\[4cm]\vspace{0.01cm}
\end{proof}

\begin{remark}{Independent Expectations Can't Imply Independence of R.V.'s}{Independent Expectations Can't Imply Independence of R.V.'s}
The converse is \textbf{not true}, i.e.,
\[E\left\lbrack \prod_{i = 1}^{n}{g_{i}(x_{i})} \right\rbrack = \prod_{i = 1}^{n}{E\lbrack g_{i}(x_{i})\rbrack} \nRightarrow g_{1}(x_{1}),\ g_{2}(x_{2}),\ \cdots, g_{n}(x_{n})\mathrm{\text{\ are\ independent.}}\]
\end{remark}

\section{Conditional Distributions}

\begin{lemma}{Properties of Conditional Probability}{Properties of Conditional Probability}
Suppose $(\Omega,\ma, P)$ is a probability space, and \(A,B,A_{1},A_{2},\ \cdots,A_{n},B_{1},B_{2},\ \cdots,B_{n} \in \ma\).
\[P( A | B ) = \left\{ \begin{aligned}
&\frac{P(A  \cap B)}{P(B)},\mathrm{\text{\ if\ }}P( B ) \neq 0\\
&\qquad0\qquad ,\mathrm{\text{\ if\ }}P( B ) = 0 \\
\end{aligned} \right.\]
(1) If \(P(B) \neq 0\), then \(P(\cdot | B)\) regarded as a function on \(\ma\) is a \textbf{probability} \textbf{measure.}\\
(2) \textbf{Multiplication theorem:}
\[P( A_{1}\cap{A_{2}\cap{\cdots\cap A_{n}}} ) = P( A_{1} )P( A_{2} | A_{1} ) \cdots P(A_{n}|A_{1}\cap{A_{2}\cap{\  \cdots\cap A_{n - 1}}}).\]
(3) \textbf{Total probability theorem:}\\
If \({\{ B_{n}\}}_{n = 1}^{\infty}\) is a partition of $\Omega$, then
\[P(A) = \sum_{n = 1}^{\infty}{P(B_{n}) \cdot}P(A|B_{n}),\forall A \in \ma.\]
(4) \textbf{Bayes' theorem:}\\
If \(P(A) \neq 0\) and \({\{ B_{n}\}}_{n = 1}^{\infty}\) is
a partition of $\Omega$, then
\[P(B_{k} | A) = \frac{P(B_{k}) \cdot P(A|B_{k})}{\sum\limits_{n = 1}^{\infty}{P(B_{n}) \cdot}P( A | B_{n} )},\ \forall A \in \ma\mathrm{,\ }P(A) > 0,\ k = 1,2,\cdots\]
\end{lemma}

\begin{proof}
\\[4cm]\vspace{0.01cm}
\end{proof}

\(\bigstar\ P_{X|Y}(x|y):\ X\) and \(Y\) are discrete r.v.'s

\begin{definition}{P.M.F. and C.D.F. of D-D}{P.M.F. and C.D.F. of D-D}
Let \(X\) and \(Y\) be discrete r.v.'s of a probability space
$(\Omega,\ma, P)$ and \(y \in \mathbb{R}\). The conditional p.m.f. \(P_{X|Y}(x|y)\) of \(X\) given that \(Y = y\) is given by
\[P_{X|Y}(x|y) = \left\{ \begin{aligned}
P( X = x | Y = y ) &= \frac{P(X = x,Y = y)}{P(Y = y)} \\
&= \frac{P_{X,Y}(x,y)}{P_{Y}(y)},P_{Y}(y) \neq 0,\forall x \in \mathbb{R} \\
&0, \qquad\qquad\mathrm{\text{o.w.}}\\
\end{aligned} \right.\ \]
The conditional c.d.f. \(F_{X|Y}\left( \cdot | y \right)\) of \(X\) given that \(Y = y\) is given by
\[\begin{aligned}
F_{X|Y}(x|y) &= P( X \ls x | Y = y )\\
& = \sum_{t \ls X ,\ t \in X(\Omega)}^{}{P( X = t | Y = y )} \\
&= \sum_{t \ls X ,\ t \in X(\Omega)}^{}{P_{X|Y}\left( t | y \right)},\forall x \in \mathbb{R}.
\end{aligned}\]
\end{definition}

\begin{remark}{Joint P.M.F.}{Joint P.M.F.}
(1) \(P_{X,Y}(x,y) = P_{Y}(y) \cdot P_{X|Y}(x|y) = P_{X}(x) \cdot P_{Y|X}(y|x)\).\\
(2) A similar def\/inition can be made for discrete \textbf{random} \textbf{vectors}.
\end{remark}

\begin{theorem}{Properties of D-D Conditional Probability}{Properties of D-D Conditional Probability}
Suppose \(X,Y,X_{1},X_{2},\ \cdots,X_{n}\) are discrete r.v.'s of a
probability space $(\Omega,\ma, P)$.\\
(1) If \(y \in \mathbb{R}\) and \(P_{Y}(y) \neq 0\), then \(P_{X|Y}\left( \cdot | y \right)\) is a p.m.f.\\
(2) $\forall x \in \mathbb{R}^{n},$
 \[P_{X}(x) = P_{X_{1}}(x_{1}) \cdot P_{X_{2}|X_{1}}\left( x_{2}|x_{1} \right) \cdots P_{X_{n}|X_{1},X_{2},\ \cdots,X_{n - 1}}\left( x_{n}|x_{1},x_{2},\ \cdots,x_{n - 1} \right).\]
(3) $\forall x \in \mathbb{R},$
\[P_{X}(x) = \sum_{y \in Y(\Omega)}^{}{P_{Y}(y)} \cdot P_{X|Y}(x|y).\]
(4) If \(x \in \mathbb{R}\) and \(P_{X}(x) \neq 0\), then
\[P_{Y|X}(y|x) = \frac{P_{Y}(y) \cdot P_{X|Y}(x|y)}{\sum\limits_{y \in Y(\Omega)}^{}{P_{Y}(y)} \cdot P_{X|Y}(x|y)},\forall y \in \mathbb{R}.\]
\end{theorem}

\begin{proof}
\\[4cm]\vspace{0.01cm}
\end{proof}

\(\bigstar\ f_{X|Y}(x|y):\ X\) and \(Y\) are jointly continuous r.v.'s

\begin{definition}{C.D.F. and P.D.F. of C-C}{C.D.F. and P.D.F. of C-C}
Let \(X\) and \(Y\) be jointly continuous r.v.'s of a probability
space $(\Omega,\ma, P)$ and \(y \in \mathbb{R}\). The conditional c.d.f. \(F_{X|Y}(x|y)\) of \(X\) given that \(Y = y\) is given by
\[F_{X|Y}(x|y) = \left\{ \begin{aligned}
\lim_{\delta \rightarrow 0}&\ \ {P( X = x | y \ls Y \ls y + \delta )}\\ 
&\quad= \lim_{\delta \rightarrow 0}\frac{P( X = x,y \ls Y \ls y + \delta)}{P( y \ls Y \ls y + \delta)} \\
&\quad= \lim_{\delta \rightarrow 0}\frac{\lbrack F_{X,Y}\left( x,y + \delta \right) - F_{X,Y}(x,y)\rbrack/\delta}{\lbrack F_{Y}\left( y + \delta \right) - F_{Y}(y)\rbrack/\delta} \\
&\quad= \dfrac{\dfrac{\partial F_{X,Y}(x,y)}{\partial y}}{f_{Y}(y)},f_{Y}(y) \neq 0,\forall x \in \mathbb{R} \\
0&, \qquad\qquad\mathrm{\text{o.w.}} \\
\end{aligned} \right.\ \]
The conditional p.d.f. \(f_{X|Y}\left( \cdot | y \right)\) of \(X\) given that \(Y = y\) is given by
\[f_{X|Y}(x|y) = \left\{ \begin{aligned}
\frac{\partial F_{X,Y}(x,y)}{\partial x} &= \frac{f_{X,Y}(x,y)}{f_{Y}(y)},f_{Y}(y) \neq 0,\forall x \in \mathbb{R} \\
&0,\qquad\mathrm{\text{o.w.}} \\
\end{aligned} \right.\]
\end{definition}

\begin{remark}{Joint P.D.F.}{Joint P.D.F.}
(1) \(f_{X,Y}(x,y) = f_{Y}(y) \cdot f_{X|Y}(x|y) = f_{X}(x) \cdot f_{Y|X}(y|x),\ \forall x,y \in \mathbb{R}\)\\
(2) A similar def\/inition can be made for jointly continuous \textbf{random vectors}.
\end{remark}

\begin{theorem}{Properties of C-C Conditional Probability}{Properties of C-C Conditional Probability}
Suppose \(X,Y,X_{1},\ X_{2},\ \cdots,X_{n}\) are jointly continuous r.v.'s of a probability space $(\Omega,\ma, P)$.\\
(1) If \(y \in \mathbb{R}\) and \(f_{Y}(y) \neq 0\), then \(f_{X|Y}\left( \cdot | y \right)\) is a p.d.f.\\
(2) $\forall x \in \mathbb{R}^{n}$,
\[f_{X}(x) = f_{X_{1}}(x_{1}) \cdot f_{X_{2}|X_{1}}\left( x_{2}|x_{1} \right) \cdot \cdot \cdot f_{X_{n}|X_{1},X_{2},\ \cdots\ ,X_{n - 1}}\left( x_{n}|x_{1},x_{2},\ \cdots,x_{n - 1} \right).\]
(3)
\[f_{X}(x) = \int_{- \infty}^{\infty}{f_{Y}(y) \cdot f_{X|Y}(x|y)}\dif y,\forall x \in \mathbb{R}.\]
(4) If \(x \in \mathbb{R}\) and \(f_{X}(x) \neq 0\), then
\[f_{Y|X}(y|x) = \frac{f_{Y}(y) \cdot f_{X|Y}(x|y)}{\int_{- \infty}^{\infty}{f_{Y}(y) \cdot f_{X|Y}(x|y)\dif y}},\forall y \in \mathbb{R}.\]
\end{theorem}

\begin{proof}
\\[4cm]\vspace{0.01cm}
\end{proof}

\(\bigstar\ f_{X|Y}(x|y)\ \mathrm{\text{and}}\ P_{X|Y}(x|y):\ X\) is a continuous r.v. and \(Y\) is a discrete r.v.

\begin{definition}{C.D.F., P.D.F. and P.M.F. of C-D and D-C}{C.D.F., P.D.F. and P.M.F. of C-D and D-C}
Let \(X\) be a continuous r.v. and \(Y\) be a discrete r.v. of a probability space $(\Omega,\ma, P)$.\\
The conditional \textbf{c.d.f.} \(F_{X|Y}\left( \cdot | y \right)\) of \(X\) given that \(Y = y,\ y \in \mathbb{R}\) is given by
\[F_{X|Y}(x|y) = \left\{ \begin{aligned}
&P(X \ls x | Y = y),P_{Y}(y) \neq 0,\forall x \in \mathbb{R} \\
&\qquad\qquad0,\qquad\mathrm{\text{o.w.}} \\
\end{aligned} \right.\ \]
The conditional \textbf{p.d.f.} \(f_{X|Y}\left( \cdot | y \right)\) of \(X\) given that \(Y = y,\ y \in \mathbb{R}\) is given by
\[f_{X|Y}(x|y) = \left\{ \begin{aligned}
\frac{\partial F_{X,Y}(x,y)}{\partial x} &= \lim_{\delta \rightarrow 0}\frac{F_{X|Y}\left( x + \delta | y \right) - F_{X|Y}(x|y)}{\delta} \\
&= \lim_{\delta \rightarrow 0}\frac{P( x \ls X \ls x + \delta | Y = y )}{\delta},P_{Y}(y) \neq 0,\forall x \in \mathbb{R} \\
&0,\qquad\mathrm{\text{o.w.}} \\
\end{aligned} \right.\ \]
The conditional \textbf{p.m.f.} \(P_{X|Y}\left( \cdot | y \right)\) of \(Y\) given that \(X = x,x \in \mathbb{R}\) is given by
\[P_{Y|X}(y|x) = \left\{ \begin{aligned}
\lim_{\delta \rightarrow 0}&\ \ P( Y = y | x \ls X \ls x + \delta ) \\
&\quad= \lim_{\delta \rightarrow 0}\frac{P(Y = y) \cdot P( x \ls X \ls x + \delta | Y = y )/\delta}{P(x \ls X \ls x + \delta)/\delta} \\
&\quad= \frac{P_{Y}(y) \cdot f_{X|Y}(x|y)}{f_{X}(x)},f_{X}(x) \neq 0,\forall y \in \mathbb{R} \\
&0, \qquad\mathrm{\text{o.w.}} \\
\end{aligned} \right.\ \]
The conditional \textbf{c.d.f.} \(F_{Y|X}\left( \cdot | x \right)\) of \(Y\) given that \(X = x,x \in \mathbb{R}\) is given by
\[F_{Y|X}(y|x) = \left\{ \begin{aligned}
\sum_{t \ls X,\ t \in X(\Omega) }^{}{P_{Y,X}\left( t|x \right)} &= \frac{\sum\limits_{t \ls X,\ t \in X(\Omega)}^{}{P_{Y}(t) \cdot f_{X|Y}\left( x | t \right)}}{f_{X}(x)},\\
&\quad f_{X}(x) \neq 0,\forall y \in \mathbb{R} \\
&0, \qquad\mathrm{\text{o.w.}} \\
\end{aligned} \right.\ \]
\end{definition}

\begin{remark}{Calculation of C-D P.D.F. and D-C P.M.F.}{Calculation of C-D P.D.F. and D-C P.M.F.}
(1)
\(P_{Y}(y) \cdot f_{X|Y}(x|y) = f_{X}(x) \cdot P_{Y|X}(y|x),\ \forall x,y \in \mathbb{R}.\)

(2) If \(y \in \mathbb{R}\) and \(P_{Y}(y) \neq 0\), then

\[f_{X|Y}(x|y) = \frac{f_{X}(x) \cdot P_{Y|X}(y|x)}{P_{Y}(y)},\forall x \in \mathbb{R}.\]

If \(x \in \mathbb{R}\) and \(f_{X}(x) \neq 0\), then

\[P_{Y|X}(y|x) = \frac{P_{Y}(y) \cdot f_{X|Y}(x|y)}{f_{X}(x)},\forall y \in \mathbb{R}.\]

\end{remark}

\begin{theorem}{Properties of C-D and D-C Conditional Probability}{Properties of C-D and D-C Conditional Probability}
Suppose \(X\) is a continuous r.v. and \(Y\) is a discrete r.v. of a probability space $(\Omega,\ma, P)$.\\
(1) If \(y \in \mathbb{R}\) and \(P_{Y}(y) \neq 0\), then
\(f_{X|Y}\left( \cdot | y \right)\) is a p.d.f. If \(x \in \mathbb{R}\) and \(f_{X}(x) \neq 0\), then \(P_{Y|X}(y|x)\) is a p.m.f.\\
(2)
\[f_{X}(x) = \sum_{y \in Y(\Omega)}^{}{P_{Y}(y) \cdot f_{X|Y}(x|y)},\ \forall x \in \mathbb{R}.\]

\[P_{Y}(y) = \int_{- \infty}^{\infty}{f_{X}(x) \cdot P_{Y|X}(y|x)\dif x,}\ \forall y \in \mathbb{R}.\]

(3) If \(x \in \mathbb{R}\) and \(f_{X}(x) \neq 0\), then

\[P_{Y|X}(y|x) = \frac{P_{Y}(y) \cdot f_{X|Y}(x|y)}{\sum\limits_{y \in Y(\Omega)}^{}{P_{Y}(y) \cdot f_{X|Y}(x|y)}},\ \forall y \in \mathbb{R}.\]

If \(y \in \mathbb{R}\) and \(P_{Y}(y) \neq 0\), then

\[f_{X|Y}(x|y) = \frac{f_{X}(x) \cdot P_{Y|X}(y|x)}{\int_{- \infty}^{\infty}{f_{X}(x) \cdot P_{Y|X}(y|x)\dif x}}\dif x,\ \forall x \in \mathbb{R}.\]
\end{theorem}

\begin{proof}
\\[4cm]\vspace{0.01cm}
\end{proof}

\begin{definition}{Expectation of Conditional R.V.}{Expectation of Conditional R.V.}
Let \(X\) and \(Y\) be r.v.'s of a probability space $(\Omega,\ma, P)$ and
\(y \in \mathbb{R}\). The conditional expectation \(E\lbrack X|Y = y\rbrack\) of \(X\) given that \(Y = y\) is given by
\[E\left\lbrack X | Y = y \right\rbrack = \left\{ \begin{aligned}
&\sum_{x \in X(\Omega)}^{}{x \cdot P_{X|Y}(x|y),\quad \mathrm{\text{if\ }}X\mathrm{\text{\ is\ a\ discrete\ r.v.}}} \\
&\int_{- \infty}^{\infty}{x \cdot}f_{X|Y}(x|y)\dif x,\quad\mathrm{\text{if\ }}X\mathrm{\text{\ is\ a\ continuous\ r.v.}} \\
\end{aligned} \right.\ \]
\end{definition}

\begin{theorem}{Expectation of Conditional Measurable Function}{Expectation of Conditional Measurable Function}
Suppose \(X\) and \(Y\) are r.v.'s of a probability space $(\Omega,\ma, P)$), and \(g\) is a measurable function from \(\mathbb{(R,}\mb_{\mathbb{R}})\) to \(\mathbb{(R,}\mb_{\mathbb{R}})\). Then
\[E\left\lbrack g(X) | Y = y \right\rbrack = \left\{ \begin{aligned}
&\sum_{x \in X(\Omega)}^{}{g(x) \cdot P_{X|Y}(x|y),\quad \mathrm{\text{if\ }}X\mathrm{\text{\ is\ a\ discrete\ r.v.}}} \\
&\int_{- \infty}^{\infty}g(x) \cdot f_{X|Y}(x|y)\dif x,\quad\mathrm{\text{if\ }}X\mathrm{\text{\ is\ a\ continuous\ r.v.}} \\
\end{aligned} \right.\ \]
\end{theorem}

\begin{proof}
\\[4cm]\vspace{0.01cm}
\end{proof}

\section{Transformations of Two R.V.'s}

\begin{theorem}{Transformations of Two R.V.'s}{Transformations of Two R.V.'s}
Suppose \(X\) and \(Y\) are r.v.'s of a probability space $(\Omega,\ma, P)$, \(g\) and \(h\) are measurable functions from \((\mathbb{R}^{2},\mb_{\mathbb{R}^{2}})\) to \(\mathbb{(R,}\mb_{\mathbb{R}})\), and \(U = g(X,Y)\) and \(V = h(X,Y)\).\\
(1) If \(X\) and \(Y\) are discrete r.v.'s, then \(U\) and \(V\) are discrete r.v.'s and
\[P_{U,V}(u,v) = \sum_{(x,y):g(x,y) = u,h(x,y) = v}^{}{P_{X,Y}(x,y)}.\]
(2) If \(X\) and \(Y\) are jointly continuous r.v.'s, \(U\) and \(V\) are discrete r.v.'s, then
\[P_{U,V}(u,v) = \iint\limits_{\{(x,y):g(x,y) = u,h(x,y) = v\}}^{}{f_{X,Y}(x,y)\dif x\dif y.}\]
(3) If \(X\)and \(Y\)are jointly continuous r.v.'s, \(U\)and \(V\)are jointly continuous r.v.'s, and
\[J(x,y) = \left| \begin{matrix}
\dfrac{\partial g(x,y)}{\partial x} & \dfrac{\partial g(x,y)}{\partial y} \\
\dfrac{\partial h(x,y)}{\partial x} & \dfrac{\partial h(x,y)}{\partial y} \\
\end{matrix} \right| \neq 0\]

\(\forall(x,y) \in \{(x,y):g(x,y) = u,\ h(x,y) = v\}\), where \(J(x,y)\) is the Jacobian determinant, \((u,v) \in g(X,Y)(\Omega) \times h(X,Y)(\Omega)\), then
\[f_{U,V}(u,v) = \sum_{(x,y):g(x,y) = u,h(x,y) = v}^{}\frac{f_{X,Y}(x,y)}{|J(x,y)|}\]
\end{theorem}

\begin{proof}
\\[4cm]\vspace{0.01cm}
\end{proof}

\begin{theorem}{Convolution Theorem}{Convolution Theorem}
Suppose \(X\) and \(Y\) are two independent r.v.'s of a probability space $(\Omega,\ma, P)$ and \(Z = X + Y\).\\
(1) If \(X\) and \(Y\) are discrete r.v.'s, then
\[P_{Z}(z) = \sum_{x \in X(\Omega)}^{}{P_{X}(x) \cdot}P_{Y}(z-x)\]
(2) If \(X\) and \(Y\) are jointly continuous r.v.'s, then
\[f_{Z}(z) = \int_{- \infty}^{\infty}{f_{X}(x) \cdot f_{Y}(z-x)\dif x}.\]
\end{theorem}

\begin{proof}
\\[4cm]\vspace{0.01cm}
\end{proof}

\section{Order Statistics}

\begin{definition}{Order Statistic}{Order Statistic}
Let \(X_{1},\ X_{2},\ \cdots,X_{n}\) be i.i.d. r.v.'s of a probability space $(\Omega,\ma, P)$. The \(i^{th}\) order statistic \(X_{(i)},\ \ i = 1,2,\ \cdots,n\)
of \(X_{1},\ X_{2},\ \cdots,X_{n}\) is defined as the \(i^{th}\)
\textbf{smallest} value in \(\left\{ X_{1},\ X_{2},\ \cdots,X_{n} \right\}\) so that
\(X_{(1)} \ls X_{(2)} \ls \cdots\ls X_{(n)}\), namely, \(X_{\left( i \right)}(w) =\) the \(i^{th}\ \)smallest value in \(\left\{ X_{1}(w),\ X_{2}(w),\ \cdots,X_{n}(w) \right\}\) for all \(w \in \Omega\). In particular,
\(X_{(1)} = \min\{ X_{1},\ X_{2},\ \cdots,X_{n}\}\) and \(X_{(n)} = \max\{ X_{1},\ X_{2},\ \cdots,X_{n}\}\).
\end{definition}

\begin{remark}{Without Equal $\&$ Not I.I.D.}{Without Equal Not I.I.D.}
(1) If \(X_{1},\ X_{2},\ \cdots,X_{n}\) are jointly continuous r.v.'s, then
\[P\left( X_{\left( i \right)} = X_{\left( j \right)} \right) = 0,\ \forall i \neq j\ \  \Rightarrow P\left( X_{\left( 1 \right)} < X_{\left( 2 \right)} < \cdots< X_{\left( n \right)} \right) = 1.\]
(2) \(X_{(i)},\ \ i = 1,2,\ \cdots,n\) is a function of \(X_{1},\ X_{2},\ \cdots,X_{n}\) \(\Rightarrow X_{(1)},X_{(2)},\ \cdots,X_{(n)}\) are \textbf{neither independent} \textbf{nor identically distributed} in general.
\end{remark}

\begin{definition}{Random Sample}{Random Sample}
A random sample of size $n$ of a probability space $(\Omega,\ma, P)$ is a sequence of $n$ i.i.d. r.v.'s $X_1,\ X_2,\ \cdots,X_n$ of $(\Omega,\ma, P)$.
\end{definition}

\begin{definition}{Range, Midrange, Median and Mean of Random Sample}{Range, Midrange, Median and Mean of Random Sample}
Let \(X_{1},\ X_{2},\ \cdots,X_{n}\) be a random sample of size \(n\) of
a probability space $(\Omega,\ma, P)$.

The \textbf{sample range} is given by
\(X_{\left( 1 \right)} + X_{\left( n \right)}\).\vspace{0.5cm}

The \textbf{sample midrange} is given by
\(\dfrac{X_{\left( 1 \right)} + X_{\left( n \right)}}{2}\).

The \textbf{sample median} is given by \(\left\{ \begin{aligned}
&\quad\ X_{\left( i - 1 \right)},\qquad\mathrm{\text{\ if}}\ n = 2i + 1 \\
&\frac{X_{\left( i \right)} + X_{\left( i + 1 \right)}}{2},\ \mathrm{\text{if}}\ n = 2i\\
\end{aligned} \right.\)\vspace{0.5cm}

\(\dis\mathrm{\text{The\ }}\mathrm{\textbf{sample\ mean}}\ \overline{X}\ \mathrm{\text{is\ given\ by}}\text{\ \ }\overline{X} = \dfrac{1}{n}\sum\limits_{i = 1}^{n}X_{i}.\)
\end{definition}

\begin{remark}{Forced Decline}{Forced Decline}
If \(\exists\ i_{j} < i_{l} \Rightarrow x_{i_{j}} \gs x_{i_{l}}\), then
\[\begin{aligned}
&F_{X_{(i_{1})},X_{(i_{2})},\cdots\ ,X_{(i_{k})}}\left( x_{i_{1}},\ \cdots,x_{i_{j}},\ \cdots,x_{i_{l}},\ \cdots,\ x_{i_{k}} \right) \\
=& F_{X_{(i_{1})},X_{(i_{2})},\cdots\ ,X_{(i_{k})}}\left( x_{i_{1}},\ \cdots,x_{i_{l}},\ \cdots,x_{i_{l}},\ \cdots,\ x_{i_{k}} \right)\ \end{aligned}\]
and
\(f_{X_{(i_{1})},X_{(i_{2})},\cdots\ ,X_{(i_{k})}}\left( x_{i_{1}},x_{i_{2}},\ \cdots,\ x_{i_{k}} \right) = 0\).
\end{remark}

\begin{theorem}{C.D.F. and P.D.F. of Jointly Order R.V.'s}{C.D.F. and P.D.F. of Jointly Order R.V.'s}
Suppose \(X_{1},\ X_{2},\ \cdots,X_{n}\) are i.i.d. jointly continuous r.v.'s of a probability space $(\Omega,\ma, P)$ with common c.d.f. \(F(x)\) and common p.d.f. \(f(x)\). If
\(1 \ls i_{1} \ls i_{2} \ls \cdots \ls i_{k} \ls n,\  - \infty < x_{i_{1}} < x_{i_{2}} < \cdots < x_{i_{k}} < \infty\), then
\[\begin{aligned}
&F_{X_{\left(i_{1}\right)},X_{\left(i_{2}\right)},\ \cdots,X_{\left(i_{k}\right)}}\left( x_{i_{1}},x_{i_{2}},\ \cdots,\ x_{i_{k}} \right)\\
=& \sum_{j_{k} = i_{k}}^{n}{\sum_{j_{k - 1} = i_{k - 1}}^{j_{k}}\cdots}\sum_{j_{1} = i_{1}}^{j_{2}} \binom{n}{j_{k}}\binom{j_{k}}{j_{k - 1}} \cdots \binom{j_{2}}{j_{1}}\left\lbrack F\left( x_{i_{1}} \right) \right\rbrack^{j_{1}}\ \left\lbrack F\left( x_{i_{2}} \right) - F\left( x_{i_{1}} \right) \right\rbrack^{j_{2} - j_{1}}\\
&\qquad\cdots\ \left\lbrack F\left( x_{i_{k}} \right) - F\left( x_{i_{k - 1}} \right) \right\rbrack^{j_{k} - j_{k - 1}}\ \left\lbrack 1 - F\left( x_{i_{k}} \right) \right\rbrack^{n - j_{k}}\\
\end{aligned}\]
and
\[\begin{aligned}
&\qquad f_{X_{\left(i_{1}\right)},X_{\left(i_{2}\right)},\ \cdots,X_{\left(i_{k}\right)}}\left( x_{i_{1}},x_{i_{2}},\ \cdots,\ x_{i_{k}} \right)\\
&= \frac{n!}{\left( i_{1} - 1 \right)!\left( i_{2} - i_{1} - 1 \right)!\cdots\left( i_{k} - i_{k - 1} - 1 \right)!\left( n - i_{k} \right)!}\\
&\cdot f\left( x_{i_{1}} \right)f\left( x_{i_{2}} \right)\cdots f\left( x_{i_{k}} \right)\cdot \left\lbrack F\left( x_{i_{1}} \right) \right\rbrack^{i_{1} - 1}\left\lbrack F\left( x_{i_{2}} \right) - F\left( x_{i_{1}} \right) \right\rbrack^{i_{2} - i_{1} - 1}\\
&\cdots\ \left\lbrack F\left( x_{i_{k}} \right) - F\left( x_{i_{k - 1}} \right) \right\rbrack^{i_{k} - i_{k - 1} - 1}\ \left\lbrack 1 - F\left( x_{i_{k}} \right) \right\rbrack^{n - i_{k}}\\
\end{aligned}\]
\vspace{0.01cm}
\end{theorem}

\begin{proof}
\\[4cm]\vspace{0.01cm}
\end{proof}

\begin{corollary}{Beta R.V. vs Binomial R.V.}{Beta R.V. vs Binomial R.V.}
Suppose \(X_{1},\ X_{2},\ \cdots,X_{n}\) are i.i.d. r.v.'s \(\sim U(0,1)\), then
\[X_{(i)}\sim\mb\left( i,n + 1 - i \right),\ i = 1,2,\ \cdots,n.\]
\end{corollary}

\begin{proof}
\[\begin{aligned}
f_{X_{\left( i \right)}}(x) &= \frac{n!}{\left( i - 1 \right)!\left( n - i \right)!}f(x)\left\lbrack F(x) \right\rbrack^{i - 1}\left\lbrack 1 - F(x) \right\rbrack^{n - i}\\
&= \frac{n!}{\left( i - 1 \right)!\left( n - i \right)!}1 \cdot x^{i - 1}{(1 - x)}^{n - i}\\
&= \frac{\Gamma(n + 1)}{\Gamma(i)\Gamma(n + 1 - i)}x^{i - 1}{(1 - x)}^{(n + 1 - i) - 1}\\
&= \frac{x^{i - 1}{(1 - x)}^{(n + 1 - i) - 1}}{B(i,n + 1 - i)},\ 0 < x < 1\\
&\Rightarrow X_{(i)}\sim\mb\left( i,n + 1 - i \right)
\end{aligned}\]
\end{proof}

\begin{corollary}{Cases One, Two and $n$ Order R.V.'s }{Cases One, Two and}
(1)
\[\begin{aligned}
\ F_{X_{\left( i \right)}}(x) &= \sum_{j = i}^{n}{\binom{n}{j}\left\lbrack F(x) \right\rbrack^{j}\left\lbrack 1 - F(x) \right\rbrack^{n - j}},\ -\infty < x < \infty,\\
f_{X_{\left( i \right)}}(x) &= \frac{n!}{\left( i - 1 \right)!\left( n - i \right)!}f(x)\left\lbrack F(x) \right\rbrack^{i - 1}\left\lbrack 1 - F(x) \right\rbrack^{n - i},\ -\infty < x < \infty.
\end{aligned}\]
In particular,
\[\begin{aligned}
F_{X_{\left( 1 \right)}}(x) &= 1 - \left\lbrack 1 - F(x) \right\rbrack^{n},\ -\infty < x < \infty,\\
f_{X_{\left( 1 \right)}}(x) &= n\cdot f(x)\left\lbrack 1 - F(x) \right\rbrack^{n - 1},\ -\infty < x < \infty,
\end{aligned}\]
and
\[F_{X_{\left( n \right)}}(x) = \left\lbrack F(x) \right\rbrack^{n},\ f_{X_{\left( 1 \right)}}(x) = nf(x)\left\lbrack F(x) \right\rbrack^{n - 1},\ -\infty < x < \infty.\]
(2)
\[\begin{aligned}
&F_{X_{\left( i_{1} \right)},X_{\left( i_{2} \right)}}(x,y)\\
=& \sum_{j_{2} = i_{2}}^{n}{\sum_{j_{1} = i_{1}}^{j_{2}}{\binom{n}{j_{2}}  \binom{j_{2}}{j_{1}} \left\lbrack F(x) \right\rbrack^{j_{1}}\left\lbrack F(y) - F(x) \right\rbrack^{j_{2} - j_{1}}\left\lbrack 1 - F(y) \right\rbrack^{n - j_{2}}}},\\
& - \infty < x < y < \infty
\end{aligned}\]
\[\begin{aligned}
f_{X_{\left( i_{1} \right)},X_{\left( i_{2} \right)}}(x,y) =& \frac{n!}{\left( i_{1} - 1 \right)!\left( i_{2} - i_{1} - 1 \right)!\left( n - i_{2} \right)!}f(x)f(y)\left\lbrack F(x) \right\rbrack^{j_{1}}
\\
\cdot &\left\lbrack F(y) - F(x) \right\rbrack^{j_{2} - j_{1}}\left\lbrack 1 - F(y) \right\rbrack^{n - j_{2}}, - \infty < x < y < \infty
\end{aligned}\]
(3)
\[\begin{aligned}
&\quad F_{X_{(1)},X_{(2)},\ \cdots,X_{(n)}}\left( x_{1},x_{2},\ \cdots,\ x_{n} \right)\\
&= \sum_{j_{n - 1} = i_{n - 1}}^{n}{\sum_{j_{n - 2} = i_{n - 2}}^{j_{n - 1}}\cdots}\sum_{j_{1} = i_{1}}^{j_{2}} \binom{n}{j_{n - 1}}  \binom{j_{n - 1}}{j_{n - 2}} \cdots\binom{j_{2}}{j_{1}}\left\lbrack F( x_{1} ) \right\rbrack^{j_{1}}\\
&\cdot\left\lbrack F( x_{2} ) - F( x_{1} ) \right\rbrack^{j_{2} - j_{1}}\cdots\ \left\lbrack F( x_{n - 1} ) - F( x_{n - 2} ) \right\rbrack^{j_{n - 1} - j_{n - 2}}\ \left\lbrack F( x_{n} ) - F( x_{n - 1} ) \right\rbrack^{n - j_{n - 1}}\\
\end{aligned}\]
and
\[\begin{aligned}
&f_{X_{\left( 1 \right)},X_{\left( 2 \right)},\ \cdots,X_{\left( n \right)}}\left( x_{1},x_{2},\ \cdots,\ x_{n} \right)\\
=& n!f(x_{1})f(x_{2})\cdots f(x_{n}),\ \ \  - \infty < x_{1} < x_{2} < \cdots < x_{n} < \infty
\end{aligned}\]
\end{corollary}

\begin{proof}
\\[4cm]\vspace{0.01cm}
\end{proof}

\section{Multinomial Distributions}

$\bigstar$ Consider an experiment with \(k\) possible outcomes \(w_{1},\ w_{2},\ \cdots,w_{k}\). Let \(A_{\left( i \right)} = \left\{ w_{i} \right\}\) be the event that the outcome is \(w_{i}\) and let \(P_{i} = P(A_{i}),i = 1,2,\ \cdots,k.\) Suppose that such an experiment is independently and successively performed \(n\) times. Let \(X_{i},\ i = 1,2,\ \cdots,k\) be the number of times that event \(A_{i}\) occurs. Then
\[\begin{aligned}
&P_{X_{1},X_{2},\ \cdots,X_{k}}\left( x_{1},x_{2},\, \cdots,x_{k} \right) \\
=& P( X_{1} = x_{1},X_{2} = x_{2},\ \cdots,X_{k} = x_{k} )\\
=& \dfrac{n!}{x_{1}!x_{2}!\cdots x_{k}!}P_{1}^{x_{1}}P_{2}^{x_{2}}\cdots P_{k}^{x_{k}},\ x_{1},x_{2},\ \cdots,\ x_{k} \gs 0\ \mathrm{\text{and}}\ \sum_{i=1}^k x_i=n.
\end{aligned}\]

\begin{definition}{Multinomial Joint R.V.'s}{Multinomial Joint R.V.'s}
Let \(X_{1},\ X_{2},\ \cdots,X_{k}\) be discrete r.v.'s of a probability space $(\Omega,\ma, P)$. We call \(X_{1},\ X_{2},\ \cdots,X_{k}\) multinomial joint r.v.'s with parameters \(n,P_{1},P_{2},\ \cdots,P_{k}\), where \(n \gs 1,\ P_{1},P_{2},\ \cdots,P_{k} \gs 0,\ P_{1} + P_{2} + \cdots + P_{k} = 1\), if the joint p.m.f. is given by
\[P_{\mX}(\mx) = \left\{ \begin{aligned}
\frac{n!}{x_{1}!x_{2}!\cdots x_{k}!} & P_{1}^{x_{1}}P_{2}^{x_{2}}\cdots P_{k}^{x_{k}},\ \mathrm{\ }x_{1},x_{2},\ \cdots,x_{k} \gs 0\ \mathrm{\text{and}}\ \sum_{i=1}^k x_i=n \\
&0, \qquad\qquad\mathrm{\text{o.w.}}\\
\end{aligned} \right.\]
\end{definition}

\begin{remark}{Verification of P.M.F.}{Verification of P.M.F.}
\(P_{\mX}(\mx) \gs 0,\ \forall\mx \in \mathbb{R}^{n}\) and
\[\sum_{\mbox{\tiny$\begin{aligned}
x_{1},x_{2},\,\cdots,\ x_{k} \gs 0 \\
x_{1} + x_{2} + \cdots + x_{k} = n \\
\end{aligned}$}}^{}{\frac{n!}{x_{1}!x_{2}!\cdots x_{k}!}P_{1}^{x_{1}}P_{2}^{x_{2}}\cdots P_{k}^{x_{k}}} = \left( P_{1} + P_{2} + \cdots + P_{k} \right)^{n} = 1\]
\(\Rightarrow P_{\mX}(\mx)\) is a p.m.f.
\end{remark}

\begin{theorem}{Splitting of Multinomial Joint R.V.'s}{Splitting of Multinomial Joint R.V.'s}
Suppose \(X_{1},\ X_{2},\ \cdots,X_{l}\) are multinomial r.v.'s of a probability space $(\Omega,\ma, P)$, with parameters \(n,P_{1},P_{2},\ \cdots,P_{l}\),
where \(n \gs 1,\ P_{1},P_{2},\ \cdots,P_{k} \gs 0,\ P_{1} + P_{2} + \cdots + P_{k} = 1\). Then
\[X_{(i_{1})},X_{(i_{2})},\cdots\ ,X_{\left( i_{k} \right)},\ n - X_{\left( i_{1} \right)} - X_{\left( i_{2} \right)} - \cdots - X_{\left( i_{k} \right)}\]
are multinomial joint r.v.'s with parameters \[n,P_{i_{1}},P_{i_{2}},\ \cdots,P_{i_{k}},\ 1 - P_{i_{1}} - P_{i_{2}} - \cdots - P_{i_{k}}.\]
\end{theorem}

\begin{proof}
\\[4cm]\vspace{0.01cm}
\end{proof}

\chapter{More Expectations and Variance}

\section{Expected Values of Sums of R.V.'s}

\begin{theorem}{Expectations of Sum of Finite R.V.'s}{Expectations of Sum of Finite R.V.'s}
Suppose \(X_{1},\ X_{2},\ \cdots,X_{n}\) are r.v.'s of a probability space $(\Omega,\ma, P)$, then
\[E\left\lbrack \sum_{i = 1}^{n}X_{i} \right\rbrack = \sum_{i = 1}^{n}{E\lbrack X_{i}\rbrack}.\]
\end{theorem}

\begin{proof}
\\[4cm]\vspace{0.01cm}
\end{proof}

\begin{theorem}{Expectations of Sum of Infinite R.V.'s}{Expectations of Sum of Infinite R.V.'s}
Suppose \(X_{1},\ X_{2},\ \cdots\) are r.v.'s of a probability space $(\Omega,\ma, P)$. If \[\sum_{i = 1}^{\infty}{E\lbrack X_{i}\rbrack} < \infty\] or if
\(X_{i}\) is nonnegative for all \(i = 1,2,\ \cdots,\) then
\[E\left\lbrack \sum_{i = 1}^{\infty}X_{i} \right\rbrack = \sum_{i = 1}^{\infty}{E\lbrack X_{i}\rbrack}.\]
\end{theorem}

\begin{proof}
\\[4cm]\vspace{0.01cm}
\end{proof}

\begin{remark}{General Expectations of Sum of Infinite R.V.'s}{General Expectations of Sum of Infinite R.V.'s}
In general,
\[E\left\lbrack \sum_{i = 1}^{\infty}X_{i} \right\rbrack \neq \sum_{i = 1}^{\infty}{E\lbrack X_{i}\rbrack}.\]
\end{remark}

\begin{corollary}{Expectation of Integer-Valued R.V.}{Expectation of Integer-Valued R.V.}
Suppose \(X\) is an integer-valued r.v. of a probability space $(\Omega,\ma, P)$, then
\[E[X] = \sum_{i = 1}^{\infty}{P(x \gs i)} - \sum_{i = 1}^{\infty}{P( x \ls - i )}.\]
\end{corollary}

\begin{proof}
\\[4cm]\vspace{0.01cm}
\end{proof}

\section{Covariance and Correlation Coef\/f\/icients}

\begin{theorem}{Cauchy-Schwarz Inequality}{Cauchy-Schwarz Inequality}
Suppose \(X\) and \(Y\) are r.v.'s of a probability space $(\Omega,\ma, P)$, and suppose \(E\lbrack X^{2}\rbrack\) and \(E\lbrack Y^{2}\rbrack\)
exists. Then
\[\left| E[XY] \right| \ls \sqrt{E\left\lbrack X^{2} \right\rbrack \cdot E\left\lbrack Y^{2} \right\rbrack}.\]
\(\text{``=''}  \Leftrightarrow X = 0\) with probability 1 or \(Y = 0\) with probability 1 or \(Y = aX\) with probability 1, where
\[a = \frac{E[XY]}{E\left\lbrack X^{2} \right\rbrack}.\]
\end{theorem}

\begin{proof}
\\[4cm]\vspace{0.01cm}
\end{proof}

\begin{remark}{Cauchy-Schwarz Equalities}{Cauchy-Schwarz Equalities}
Suppose that \(E\lbrack X^{2}\rbrack \neq 0\) and \(E\lbrack Y^{2}\rbrack \neq 0\), then

\[\left. \ E[XY] \right.\  = \sqrt{E\left\lbrack X^{2} \right\rbrack \cdot E\left\lbrack Y^{2} \right\rbrack}\  \Leftrightarrow Y = aX\]
with probability 1, where

\[a = \frac{E[XY]}{E\left\lbrack X^{2} \right\rbrack} = \sqrt{\frac{E\left\lbrack Y^{2} \right\rbrack}{E\left\lbrack X^{2} \right\rbrack}} > 0.\]

\[\left. \ E[XY] \right.\  = - \sqrt{E\left\lbrack X^{2} \right\rbrack \cdot E\left\lbrack Y^{2} \right\rbrack}\  \Leftrightarrow Y = aX\]
with probability 1, where

\[a = \frac{E[XY]}{E\left\lbrack X^{2} \right\rbrack} = - \sqrt{\frac{E\left\lbrack Y^{2} \right\rbrack}{E\left\lbrack X^{2} \right\rbrack}} < 0.\]

\end{remark}

\begin{corollary}{Variance Larger Than or Equal to Zero}{Variance Larger Than or Equal to Zero}
Suppose \(X\) is a r.v. of a probability space $(\Omega,\ma, P)$ and suppose
\(E\lbrack X^{2}\rbrack\) exists, then
\[\left| E[X] \right|^{2} \ls E\left\lbrack X^{2} \right\rbrack.\]
\end{corollary}

\begin{proof}
\\[4cm]\vspace{0.01cm}
\end{proof}

\begin{definition}{Covariance}{Covariance}
Let \(X\) and \(Y\) be r.v.'s of a probability space $(\Omega,\ma, P)$ with means \(\mu_{X}\) and \(\mu_{Y}\), resp. The covariance \(Cov(X,Y)\) (or \(\sigma_{X,Y}\)) of \(X\) and \(Y\) is given by
\[Cov(x,y) = \sigma_{X,Y} = E\left\lbrack \left( X - \mu_{X} \right)\left( Y - \mu_{Y} \right) \right\rbrack.\]
We say that \(X\) and \(Y\) are positively correlated, negatively correlated and uncorrelated if
\(Cov(x,y) > 0,\ Cov(x,y) < 0\) and
\(Cov(x,y) = 0\), resp.
\end{definition}

\begin{remark}{Covariance of Linear Combination of Two R.V.'s}{Covariance of Linear Combination of Two R.V.'s}
(1)
\(Var(X) = E\left\lbrack \left( X - \mu_{X} \right)^{2} \right\rbrack\)
is a measure of the spread or dispersion of \(X\).

\( Var(Y) = E\left\lbrack \left( Y - \mu_{Y} \right)^{2} \right\rbrack\)
is a measure of the spread or dispersion of \(Y\).

\( Cov(x,y) = \sigma_{X,Y} = E\left\lbrack \left( X - \mu_{X} \right)\left( Y - \mu_{Y} \right) \right\rbrack\)
is a measure of the joint spread or dispersion of \(X\) and \(Y\).

(2)
\[\begin{aligned}
Var(aX + bY) &= E\left\lbrack \left[ \left( aX + bY \right) - \left( a\mu_{X} + b\mu_{Y} \right) \right]^{2} \right\rbrack\\
&= E\left\lbrack \left[ a\left( X - \mu_{X} \right) + b\left( Y - \mu_{Y} \right) \right]^{2} \right\rbrack\\
& = a^{2}Var(X) + b^{2}Var(Y) + 2abCov(x,y)
\end{aligned}\]
is a measure of the spread or dispersion along the \((ax + by)\)-direction.

\end{remark}

\begin{theorem}{Calculating Covariance}{Calculating Covariance}
Suppose \(X\) and \(Y\) are r.v.'s of a probability space $(\Omega,\ma, P)$.

(1) \(Var(X) = Cov(X,X)\).

(2)
\(Cov(x,y) = Cov(Y,X) = E[XY] - E[X] E[Y].\)

(3)
\(\left| Cov(x,y) \right| \ls \sigma_{X}\cdot \sigma_{Y},\ \text{``=''} \Leftrightarrow X = \mu_{X}\) with probability 1 or
\(Y = \mu_{Y}\) with probability 1 or $Y=aX+b$ with probability 1, where
\[a = \frac{\sigma_{X,Y}}{\sigma_{X}^{2}},\ b = \mu_{Y} - \mu_{X} \cdot \frac{\sigma_{X,Y}}{\sigma_{X}^{2}}.\]

If \(\sigma_{X} \neq 0\) and \(\sigma_{Y} \neq 0\), then
\[Cov( X,Y)= \sigma_{X}\cdot \sigma_{Y} \Leftrightarrow Y = aX + b\]
with probability 1, where
\[a = \frac{\sigma_{Y}}{\sigma_{X}} > 0,\ b = \mu_{Y} - \mu_{X} \cdot \frac{\sigma_{Y}}{\sigma_{X}}.\]
\[Cov( X,Y)=- \sigma_{X}\cdot \sigma_{Y} \Leftrightarrow Y = aX + b\]
with probability 1, where
\[a = - \frac{\sigma_{Y}}{\sigma_{X}} < 0,\ b = \mu_{Y} + \mu_{X} \cdot \frac{\sigma_{Y}}{\sigma_{X}}.\]
\end{theorem}

\begin{proof}
\\[4cm]\vspace{0.01cm}
\end{proof}

\begin{theorem}{Covariance of Two Linear Combined R.V.'s}{Covariance of Two Linear Combined R.V.'s}
Suppose \(X_{1},\ X_{2},\ \cdots,X_{n},\ Y_{1},\ Y_{2},\ \cdots,Y_{m}\)
are r.v.'s of a probability space $(\Omega,\ma, P)$.
(1)
\[Cov\left( \sum_{i = 1}^{n}{a_{i}X_{i}},\ \sum_{j = 1}^{m}{b_{j}Y_{j}} \right) = \sum_{i = 1}^{n}{\sum_{j = 1}^{m}a_{i}b_{j}}Cov\left( X_{i},Y_{j} \right).\]
(2)
\[Var\left( \sum_{i = 1}^{n}{a_{i}X_{i}} \right) = \sum_{i = 1}^{n}{a_{i}^{2}Var(x_{i})} + 2\sum_{1 \ls i < j \ls n}^{}a_{i}b_{j}Cov\left( X_{i},X_{j} \right).\]

In particular, if \(X_{1},\ X_{2},\ \cdots,X_{n}\) are \textbf{pairwise
uncorrelated}, then
\[Var\left( \sum_{i = 1}^{n}{a_{i}X_{i}} \right) = \sum_{i = 1}^{n}{a_{i}^{2}Var(x_{i})}.\]
\end{theorem}

\begin{proof}
\\[4cm]\vspace{0.01cm}
\end{proof}

\begin{theorem}{Independence Implies Uncorrelated}{Independence Implies Uncorrelated}
Suppose \(X\) and \(Y\) are r.v.'s of a probability space $(\Omega,\ma, P)$. If \(X\bot Y\), then \(X\) and \(Y\) are uncorrelated, i.e.,
\[Cov(x,y) = E[XY] - E[X] E[Y] = E[X] E[Y] - E[X] E[Y] = 0.\]
\vspace{0.01cm}
\end{theorem}

\begin{proof}
\\[4cm]\vspace{0.01cm}
\end{proof}

\begin{remark}{Uncorrelated Can't Imply Independence}{Uncorrelated Can't Imply Independence}
The inverse is not true, i.e.,
\[Cov(x,y) = 0 \nRightarrow X\bot Y.\]
\vspace{0.01cm}
\end{remark}

\begin{definition}{Correlation Coef\/f\/icient}{Correlation Coefficient}
Let \(X\) and \(Y\) be r.v.'s of a probability space $(\Omega,\ma, P)$ with
\(0 < \sigma_{X}^{2} < \infty,0 < \sigma_{Y}^{2} < \infty\). The correlation coef\/f\/icient between \(X\) and \(Y\) is given by
\[\rho_{X,Y} = Cov\left( X^{*},Y^{*} \right) = Cov\left( \frac{X - \mu_{X}}{\sigma_{X}},\frac{Y - \mu_{Y}}{\sigma_{Y}} \right) = \frac{\sigma_{X,Y}}{\sigma_{X}\sigma_{Y}}.\]
\end{definition}

\begin{remark}{Properties of Correlation Coef\/f\/icient}{Properties of Correlation Coefficient}
(1)
\(X^{*} = \dfrac{X - \mu_{X}}{\sigma_{X}}\ \mathrm{\text{is\ independent\ of\ the\ units\ in\ which\ }}X\mathrm{\text{\ is\ measured}}.\)
\vspace{0.1cm}

\(\Rightarrow \rho_{X,Y}\) is \textbf{independent of the units} in which
\(X\) and \(Y\) is measured.

(2) \(- 1 \ls \rho_{X,Y} \ls 1.\)

\(\rho_{X,Y} = 1 \Leftrightarrow Y = aX + b\) with probability 1, where
\[a = \frac{\sigma_{Y}}{\sigma_{X}} > 0,\ b = \mu_{Y} - \mu_{X} \cdot \frac{\sigma_{Y}}{\sigma_{X}}.\]

\(\rho_{X,Y} =- 1 \Leftrightarrow Y = aX + b\) with probability 1, where
\[a = - \frac{\sigma_{Y}}{\sigma_{X}} < 0,\ b = \mu_{Y} + \mu_{X} \cdot \frac{\sigma_{Y}}{\sigma_{X}}.\]
\end{remark}

\section{Conditioning on R.V.'s}

\begin{definition}{Conditional Expectation on R.V.'s}{Conditional Expectation on R.V.'s}
Let \(X\) and \(Y\) be r.v.'s of a probability space $(\Omega,\ma, P)$.\\
 Let \(g(Y) = E[X | Y = y],\ \forall y\mathbb{\in R}\). We denote \(E\lbrack X|Y\rbrack\) as the r.v. \(g(Y)\). Note that \(E\lbrack X|Y\rbrack\) is a function of \(Y\).
\end{definition}

\begin{theorem}{Marginal Expectation}{Marginal Expectation}
Suppose \(X\) and \(Y\) are r.v.'s of a probability space $(\Omega,\ma, P)$. Then
\[E\left\lbrack E\left\lbrack X | Y \right\rbrack \right\rbrack = E[X].\]
\vspace{0.01cm}
\end{theorem}

\begin{proof}
\\[4cm]\vspace{0.01cm}
\end{proof}

\begin{theorem}{Marginal Expectation of Measurable Function}{Marginal Expectation of Measurable Function}
Suppose \(X\) and \(Y\) are r.v.'s of a probability space $(\Omega,\ma, P)$. Then
\[E\left\lbrack E\left\lbrack X \cdot g(Y) | Y \right\rbrack \right\rbrack = g( Y )E\left\lbrack X|Y \right\rbrack.\]
\vspace{0.01cm}
\end{theorem}

\begin{proof}
\\[4cm]\vspace{0.01cm}
\end{proof}

\begin{theorem}{Wald's Equations}{Wald's Equations}
Suppose \(X_{1},\ X_{2},\ \cdots\) are i.i.d. r.v.'s \(\sim X\) and \(N\)
is a positive integer-valued r.v. of a probability space $(\Omega,\ma, P)$, and \(N\bot\{ X_{1},\ X_{2},\ \cdots\}\).

(1) If \(E[X] < \infty\) and
\(E\left\lbrack N \right\rbrack < \infty\), then
\[E\left\lbrack \sum_{i = 1}^{N}X_{i} \right\rbrack = E\left\lbrack N \right\rbrack \cdot E[X].\]

(2) If \(Var(X) < \infty\) and \(Var(N) < \infty\), then
\[Var\left( \sum_{i = 1}^{N}X_{i} \right) = E[N] \cdot Var(X) + \left( E[X] \right)^{2} \cdot Var(N).\]
\end{theorem}

\begin{proof}
\\[4cm]\vspace{0.01cm}
\end{proof}

\begin{theorem}{Law of Total Probability}{Law of Total Probability}
Suppose \(A\) is an event and \(X\) is a r.v. of a probability space $(\Omega,\ma, P)$, then
\[P(A) = \left\{ \begin{aligned}
&\sum_{x \in X(\Omega)}^{}{P(A|X = x) \cdot P_{X}(x)},\qquad\ \mathrm{\text{if\ }}X\mathrm{\text{\ is\ a\ discrete\ r.v.}}\\
&\int_{- \infty}^{\infty} P( A | X = x ) \cdot f_{X}( x )\dif x,\qquad\text{if\ }X\text{\ is\ a\ continuous\ r.v.} \\
\end{aligned} \right.\]
\end{theorem}

\begin{proof}
\\[4cm]\vspace{0.01cm}
\end{proof}

\begin{theorem}{Conditional Variance on R.V.'s}{Conditional Variance on R.V.'s}
Suppose \(X\) and \(Y\) are r.v.'s of a probability space $(\Omega,\ma, P)$, then
\[Var(X) = E[Var(x|y)] + Var(E[X | Y]).\]
\vspace{0.01cm}
\end{theorem}

\begin{proof}
\\[4cm]\vspace{0.01cm}
\end{proof}

\section{Bivariate Normal (Gaussian) Distribution}

\begin{definition}{Bivariate Normal (Gaussian) R.V.'s}{Bivariate Normal (Gaussian) R.V.'s}
Let \(X_{1}\) and \(X_{2}\) be r.v.'s of a probability space $(\Omega,\ma, P)$. We call \(X_{1}\) and \(X_{2}\) jointly normal (Gaussian) r.v.'s with parameters
\[\bm{\mu}=\binom{\mu_{1}}{\mu_{2}}\]
and
\[\sum_{}^{} = \begin{pmatrix}
\sigma_{11} & \sigma_{12} \\
\sigma_{21} & \sigma_{22} \\
\end{pmatrix} > 0,\]
where ``>0'' means positive def\/inite, denoted \[\mX\sim N\left(\bm{\mu} ,\sum \right),\] if their joint p.d.f. is given by
\[\begin{aligned}
&f_{X}(X) = \frac{1}{\sqrt{{(2\pi)}^{2}\left|\sum\right|}}\exp\left\lbrack - \frac{1}{2}\left(\mx -\bm{\mu} \right)^{T}\sum\nolimits_{}^{-1}\left( \mx-\bm{\mu} \right) \right\rbrack\\
&= \frac{1}{\sqrt{{(2\pi)}^{2}\left|\sum\right|}}\exp\left\lbrack - \frac{1}{2}\left( x_{1} - \mu_{1}, x_{2} - \mu_{2} \right)\frac{1}{\left| \sum \right|}\begin{pmatrix}
\sigma_{22} & {- \sigma}_{12} \\
 - \sigma_{12} & \sigma_{11} \\
\end{pmatrix} \binom{x_{1} - \mu_{1}}{x_{2} - \mu_{2}} \right\rbrack\\
&= \frac{1}{\sqrt{{(2\pi)}^{2}\left|\sum\right|}}\exp\left(\sum\nolimits_{}^* \right)
\end{aligned}\]
where
\[\left|\sum\right| = \det\left( \sum \right) = \sigma_{11} \cdot \sigma_{22} - \sigma_{12}^{2} > 0,\]
\[\sum\nolimits_{}^*=- \frac{1}{2\left| \sum \right|}\left[ \sigma_{22}\left( x_{1} - \mu_{1} \right)^{2} - 2\sigma_{12}\left( x_{1} - \mu_{1} \right)\left( x_{2} - \mu_{2} \right) + \sigma_{11}\left( x_{2} - \mu_{2} \right)^{2} \right].\]
Such a joint p.d.f. is called a bivariate normal p.d.f. with parameters
\(\bm{\mu}\) and \(\sum\).
\end{definition}

\begin{theorem}{Explicitly Normal (Gaussian) R.V.}{Explicitly Normal (Gaussian) R.V.}
Suppose \(X_{1}\) and \(X_{2}\) are r.v.'s of a probability space $(\Omega,\ma, P)$, and suppose \(\mX\sim N\left(\bm{\mu} ,\sum \right)\).

(1) \(X_{1}\sim N\left( \mu_{1},\sigma_{11} \right)\) and
\(X_{2}\sim N\left( \mu_{2},\sigma_{22} \right)\). Therefore
\[\mu_{1} = \mu_{X_{1}},\ \sigma_{11} = \sigma_{X_{1}}^{2} \triangleq \sigma_{1}^{2},\ \mu_{2} = \mu_{X_{2}},\ \sigma_{22} = \sigma_{X_{2}}^{2} \triangleq \sigma_{2}^{2}.\]
(2)
\[X_{2}|_{X_{1} = x_{1}}\sim N\left( \mu_{2} + \frac{\sigma_{12}}{\sigma_{11}}\left( x_{1} - \mu_{1} \right),\ \frac{\left| \sum \right|}{\sigma_{11}} \right)\]
and
\[X_{1}|_{X_{2} = x_{2}}\sim N\left( \mu_{1} + \frac{\sigma_{12}}{\sigma_{22}}\left( x_{2} - \mu_{2} \right),\ \frac{\left|\sum\right|}{\sigma_{22}} \right).\]
(3)
\(\sigma_{12} = \sigma_{X_{1},X_{2}} = \rho_{X_{1},X_{2}} \cdot \sigma_{X_{1}}\sigma_{X_{2}} \triangleq \rho \cdot \sigma_{1}\sigma_{2}\). Therefore
\[X_{2}|_{X_{1} = x_{1}}\sim N\left( \mu_{2} + \rho\cdot\frac{\sigma_{2}}{\sigma_{1}}\left( x_{1} - \mu_{1} \right),\left(1 - \rho^{2}\right)\sigma_{2}^{2} \right)\]
and
\[X_{1}|_{X_{2} = x_{2}}\sim N\left( \mu_{1} + \rho\cdot\frac{\sigma_{1}}{\sigma_{2}}\left( x_{2} - \mu_{2} \right),\left(1 - \rho^{2}\right)\sigma_{1}^{2} \right).\]
\end{theorem}

\begin{proof}
\\[4cm]\vspace{0.01cm}
\end{proof}

\begin{remark}{Mean Vector and Covariance Matrix}{Mean Vector and Covariance Matrix}
\(\dis\bm{\mu}= \binom{\mu_{1}}{\mu_{2}}\) is called the mean vector of $\mX$, and \(\dis\sum_{}^{} = \begin{pmatrix}
\sigma_{11} & \sigma_{12} \\
\sigma_{21} & \sigma_{22} \\
\end{pmatrix}\) is called the covariance matrix of $\mX$.
\end{remark}

\begin{lemma}{Linear Conditional Expectation and Constant Variance}{Linear Conditional Expectation and Constant Variance}
Suppose \(X_{1}\) and \(X_{2}\) are jointly continuous r.v.'s of a probability
space $(\Omega,\ma, P)$ with \(\mu_{X_{1}} = \mu_{1},\ \ \mu_{X_{2}} = \mu_{2},\ \sigma_{X_{1}}^{2} = \sigma_{1}^{2},\ \sigma_{X_{2}}^{2} = \sigma_{2}^{2},\ \rho_{X_{1},X_{2}} = \rho\).

(1) If
\(E[ X_{2} | X_{1} = x_{1} ] = ax_{1} + b\)
is a linear function in \(x_{1}\), then
\[E[ X_{2} | X_{1} = x_{1} ] = \mu_{2} + \rho\cdot\frac{\sigma_{2}}{\sigma_{1}}\left( x_{1} - \mu_{1} \right).\]
(2) If
\(E[ X_{2} | X_{1} = x_{1} ]= ax_{1} + b\)
is a linear function in \(x_{1}\), and
\(Var( X_{2}| X_{1} = x_{1} ) = \sigma^{2}\)
is a constant, then
\[Var( X_{2} | X_{1} = x_{1} ) = \left( 1 - \rho^{2} \right)\sigma_{2}^{2}.\]
\end{lemma}

\begin{proof}
\\[4cm]\vspace{0.01cm}
\end{proof}

\begin{theorem}{Derivation of Jointly Normal R.V.'s}{Derivation of Jointly Normal R.V.'s}
Suppose \(X_{1}\) and \(X_{2}\) are r.v.'s of a probability space $(\Omega,\ma, P)$. Suppose

(1) \(X_{1}\) is a normal r.v.

(2) \(X_{2}|X_{1} = x_{1}\) is a normal r.v. for all
\(x_{1}\mathbb{\in R}\).

(3) \(E[ X_{2} | X_{1} = x_{1} ]\) is a
linear function in \(X_{1}\), and
\(Var( X_{2}| X_{1} = x_{1}) = \sigma^{2}\)
is a constant.

Then \(X_{1}\) and \(X_{2}\) are \textbf{jointly normal} r.v.'s.
\end{theorem}

\begin{proof}
\\[4cm]\vspace{0.01cm}
\end{proof}

\begin{theorem}{Independence mutually Implies Uncorrelated}{Independence mutually Implies Uncorrelated}
Suppose \(X_{1}\) and \(X_{2}\) are jointly normal r.v.'s of a probability space $(\Omega,\ma, P)$. Then \(X_{1}\) and \(X_{2}\) are independent 
\(\Leftrightarrow\) \(X_{1}\) and \(X_{2}\) are uncorrelated.
\end{theorem}

\begin{proof}
\\[4cm]\vspace{0.01cm}
\end{proof}

\begin{theorem}{Linearly Generated Normal R.V.}{Linearly Generated Normal R.V.}
Suppose \(\mX\sim N\left( {\bm{\mu}}_{\mX},\sum\nolimits_{\mX} \right)\) and \(\mY= A\mX + b\), where \(A\) is \textbf{nonsingular}, i.e., \(|A| \neq 0\).
Then
\[\mY\sim N\left( A{\bm{\mu}}_{\mX} + b,A\sum\nolimits_{\mX}^{}A^{T} \right).\]
\end{theorem}

\begin{proof}
\\[4cm]\vspace{0.01cm}
\end{proof}

\chapter{Sums of Independent R.V.'s and Limit Theorems}

\section{Moment Generating Functions}

\begin{definition}{Moment Generating Function}{Moment Generating Function}
The moment generating function (m.g.f.) \(M_{X}(t)\) of a r.v. \(X\) is
given by
\(M_{X}(t) = E[ e^{tx} ]\)
if \(\exists\delta > 0 \Rightarrow M_{X}(t)\) is def\/ined
for all \(t \in ( - \delta,\delta)\).
\end{definition}

\begin{theorem}{Moment Generation}{Moment Generation}
(1)
\(E[X^{n}] = M_{X}^{\left( n \right)}(0),\ \forall n \gs 0.\)

(2) Maclaurin's series for \(M_{X}(t)\):
\[M_{X}(t) = \sum_{n = 0}^{\infty}\frac{M_{X}^{\left( n \right)}(0)}{n!}t^{n} = \sum_{n = 0}^{\infty}\frac{E[X^{n}]}{n!}t^{n}.\]
\end{theorem}

\begin{proof}
\\[4cm]\vspace{0.01cm}
\end{proof}

\begin{remark}{Sufficient Condition for $n^{th}$ Moment to Converge}{Sufficient Condition for }
If \(\left| M_{X}(t) \right| < \infty\) for some \(t > 0\),
then \(\left| E[X^{n}] \right| < \infty\) for
all \(n \gs 1\). But the converse is not true.
\end{remark}

\begin{theorem}{Same M.G.F. Implies Same C.D.F.}{Same M.G.F. Implies Same C.D.F.}
If \(M_{X}(t) = M_{Y}(t)\) for all
\(t \in ( - \delta,\delta)\) for some \(\delta > 0\), then the c.d.f. of \(X\) and \(Y\) are the same.
\end{theorem}

\begin{proof}
\\[4cm]\vspace{0.01cm}
\end{proof}

\section{Sums of Independent R.V.'s}

\begin{theorem}{M.G.F. of Sums of Independent R.V.'s}{M.G.F. of Sums of Independent R.V.'s}
Suppose \(X_{1},\ X_{2},\ \cdots,X_{n}\) are \textbf{independent} r.v.'s
with m.g.f.'s
\[M_{X_{1}}(t),M_{X_{2}}(t),\ \cdots,M_{X_{n}}(t)\] respectively. Then the m.g.f. of their \textbf{sum}
\(X = X_{1} + \ X_{2} + \cdots + \ X_{n}\) is
\[M_{X}(t) = \prod_{i = 1}^{n}{M_{X_{i}}(t)}.\]
\end{theorem}

\begin{proof}
\\[4cm]\vspace{0.01cm}
\end{proof}

\begin{theorem}{M.G.F. of Sums of Normal R.V.'s}{M.G.F. of Sums of Normal R.V.'s}
Suppose \(X_{1},\ X_{2},\ \cdots,X_{n}\) are \textbf{independent} r.v.'s
and \(X_{i}\sim N\left( \mu_{i},\ \sigma_{i}^{2} \right),\ \forall i = 1,2,\ \cdots,n\)
and suppose \(a_{1},a_{2},\ \cdots,a_n\in\mathbb{R}\).
If 
\[ X = \sum_{i = 1}^{n}{a_{i}X}_{i},\]
then
\[ X\sim N\left( \sum_{i = 1}^{n}{a_{i}\mu_{i}},\sum_{i = 1}^{n}{a_{i}^{2}\sigma_{i}^{2}} \right).\]
\end{theorem}

\begin{proof}
\\[4cm]\vspace{0.01cm}
\end{proof}

\begin{corollary}{M.G.F. of Sums of I.I.D. Normal R.V.'s}{M.G.F. of Sums of I.I.D. Normal R.V.'s}
Suppose \(X_{1},\ X_{2},\ \cdots,X_{n}\) are \textbf{i.i.d.} 
\(\sim N\left( \mu,\sigma^{2} \right)\), then
\[S_{n} = \sum_{i = 1}^{n}X_{i}\sim N\left( n\mu,n\sigma^{2} \right),\ \mathrm{\text{and}}\mathrm{\ }\ \overline{X} = \frac{S_{n}}{n}\sim N\left( \mu,\frac{\sigma^{2}}{n} \right).\]
\end{corollary}

\begin{proof}
\\[4cm]\vspace{0.01cm}
\end{proof}

\section{Markov and Chebyshev Inequalities}

\begin{theorem}{Markov's Inequality}{Markov's Inequality}
Suppose \(X\) is a nonnegative r.v., then
\[P(X \gs t) \ls \frac{E[X]}{t},\ \forall t > 0.\]
\end{theorem}

\begin{proof}
\\[4cm]\vspace{0.01cm}
\end{proof}

\begin{theorem}{Chebyshev's Inequality}{Chebyshev's Inequality}
\[P( \left| X - \mu_{X} \right| \gs t ) \ls \frac{\sigma_{X}^{2}}{t^{2}},\ \forall t > 0.\]
In particular,
\[P(\left| X - \mu_{X} \right| \gs k \cdot \sigma_{X}) \ls \frac{1}{k^{2}},\ \forall k > 0.\]
\end{theorem}

\begin{proof}
\\[4cm]\vspace{0.01cm}
\end{proof}

\begin{remark}{Not Tight Bounds}{Not Tight Bounds}
The bounds obtained by Markov and Chebyshev inequalities are usually
\textbf{not very tight}.\\
\vspace{0.01cm}
\end{remark}

\begin{theorem}{Zero Absolute Moment}{Zero Absolute Moment}
\[E[\left| X \right|] = 0\  \Leftrightarrow X = 0\ \mathrm{\text{with\ probability}}\ 1.\]
\end{theorem}

\begin{proof}
\\[4cm]\vspace{0.01cm}
\end{proof}

\begin{corollary}{Zero Variance}{Zero Variance}
\[Var(X) = 0\  \Leftrightarrow X = 0\ \mathrm{\text{with\ probability}}\ 1.\]
\end{corollary}

\begin{proof}
\\[4cm]\vspace{0.01cm}
\end{proof}

\begin{theorem}{Chebyshev's Inequality for I.I.D R.V.'s}{Chebyshev's Inequality for I.I.D R.V.'s}
Suppose \(X_{1},\ X_{2},\ \cdots,X_{n}\) are \textbf{i.i.d.} r.v.'s with
mean \(\mu\) and variance \(\sigma^{2} < \infty\). Let 
\[\overline{X} = \frac{1}{n}\sum_{i = 1}^{n}X_{i}\]
be the sample mean of \( X_{1},\ X_{2},\ \cdots,X_{n}.\)
Then
\[P\left( \left| \overline{X} - \mu \right| \gs \varepsilon \right) \ls \frac{\sigma^{2}}{n\varepsilon^{2}}.\]
\end{theorem}

\begin{proof}
\\[4cm]\vspace{0.01cm}
\end{proof}

\begin{theorem}{Chebyshev's Inequality for I.I.D. Bernoulli R.V.'s}{Chebyshev's Inequality for I.I.D. Bernoulli R.V.'s}
Suppose \(X_{1},\ X_{2},\ \cdots,X_{n}\) are i.i.d. \(\sim\) Bernoulli\((p)\). Let 
\[\overline{X} = \frac{1}{n}\sum_{i = 1}^{n}X_{i}\]
be the sample mean of \( X_{1},\ X_{2},\ \cdots,X_{n}.\)
Then
\[P\left( \left| \overline{X} - p \right| \gs \varepsilon \right) \ls \frac{p(1 - p)}{n\varepsilon^{2}} \ls \frac{1}{4n\varepsilon^{2}}.\]
\end{theorem}

\begin{proof}
\\[4cm]\vspace{0.01cm}
\end{proof}

\section{Laws of Large Numbers (LLN's)}

\begin{definition}{Converge in Probability}{Converge in Probability}
Let \(X,X_{1},\ X_{2},\cdots\) be r.v.'s of a probability space $(\Omega,\ma, P)$. We say that \(X_{n}\) converges to \(X\) \textbf{in probability}, denoted
\[X_{n}\xrightarrow{\ P\ }X,\]
if
\[\lim_{n \rightarrow \infty}{P(\left| X_{n} - X \right| < \varepsilon)} = 1,\ \forall\varepsilon > 0,\]
or
\[\lim_{n \rightarrow \infty}{P(\left| X_{n} - X \right| > \varepsilon)} = 0,\ \forall\varepsilon > 0.\]
\end{definition}

\begin{theorem}{Weak Law of Large Numbers (WLLN)}{Weak Law of Large Numbers (WLLN)}
Suppose \(X_{1},\ X_{2},\ \cdots\) are i.i.d. r.v.'s with mean \(\mu\) and variance \(\sigma^{2} < \infty\). Then
\[\overline{X_{n}} = \frac{1}{n}\sum_{i = 1}^{n}X_{i}\xrightarrow{\ P\ }\mu,\]
i.e.,
\[ \lim_{n \rightarrow \infty}{P\left( \left| \overline{X_{n}} - \mu \right| > \varepsilon \right)} = 0,\ \forall\varepsilon > 0.\]
\end{theorem}

\begin{proof}
\\[4cm]\vspace{0.01cm}
\end{proof}

\begin{remark}{Relative Frequency Converges to Probability in Probability}{Relative Frequency Converges to Probability in Probability}
Let an experiment be repeated independently and let \(n(A)\) be the
number of times an event \(A\) occurs in the first \(n\) repetitions of
the experiment. Let

\[X_{i} = \left\{ \begin{aligned}
&1,\mathrm{\text{if}}\ A \mathrm{\text{\ occurs\ on\ the}}\ i^{th}\ \mathrm{\text{repetition}} \\
&0, \mathrm{\text{o.w.}} \\
\end{aligned} \right.\]
Then
\[\begin{aligned}
 &n(A) = \sum_{i = 1}^{n}X_{i}\text{\ \ }\mathrm{\text{and}}\text{\ \ }E[X_{i} ]= 1 \cdot P(A) + 0 \cdot P(A^{c}) = P(A).\\
\Rightarrow& \lim_{n \rightarrow \infty}{P\left( \left| \frac{n(A)}{n} - P(A) \right| > \varepsilon \right)} = \lim_{n \rightarrow \infty}{P\left( \left| \frac{1}{n}\sum_{i = 1}^{n}X_{i} - P(A) \right| > \varepsilon \right)} = 0.
\end{aligned}\]
\(\therefore\) The relative frequency $\dfrac{n(A)}{n}$ of occurrence of $A$ is very likely close to $P(A)$ if $n$ is suf\/f\/iciently large.
\end{remark}

\begin{definition}{Converge Almost Surely}{Converge Almost Surely}
Let \(X,X_{1},\ X_{2},\cdots\) be r.v.'s of a probability space $(\Omega,\ma, P)$. We say that \(X_{n}\) converges to \(X\) \textbf{almost surely} (a.s.),
denoted
\[X_{n}\xrightarrow{\ \text{a.s.}\ }X,\]
if \[P\left( \lim_{n \rightarrow \infty}X_{n} = X \right) = 1.\]
\end{definition}

\begin{theorem}{Strong Law of Large Numbers (SLLN)}{Strong Law of Large Numbers (SLLN)}
Suppose \(X_{1},\ X_{2},\ \cdots\) are i.i.d. r.v.'s with mean \(\mu\). Then

\[\overline{X_{n}} = \frac{1}{n}\sum_{i = 1}^{n}X_{i}\xrightarrow{\ \text{a.s.}\ }\mu\]
i.e.,
\[P\left( \lim_{n \rightarrow \infty}\overline{X_{n}} = \mu \right) = 1.\]
\end{theorem}

\begin{proof}
\\[4cm]\vspace{0.01cm}
\end{proof}

\begin{remark}{Relative Frequency Converges Almost Surely}{Relative Frequency Converges Almost Surely}
\[P\left( \lim_{n \rightarrow \infty}\frac{n( A )}{n} = P(A) \right) = 1\  \Rightarrow \ \lim_{n \rightarrow \infty}\frac{n( A )}{n} = P( A )\ \mathrm{\text{with\ probability}}\ 1.\]
\end{remark}

\begin{theorem}{Converge Almost Surely Implies Convergence in Probability}{Converge Almost Surely Implies Convergence in Probability}
\[\mathrm{\text{If}}\mathrm{\ }\ X_{n}\xrightarrow{\ \text{a.s.}\ }X,\mathrm{\ }\mathrm{\ }\mathrm{\text{then}}\mathrm{\ }\ X_{n}\xrightarrow{\ P\ }X.\]
\end{theorem}

\begin{proof}
\\[4cm]\vspace{0.01cm}
\end{proof}

\section{Central Limit Theorem (CLT)}

\begin{theorem}{Levy Continuity Theorem}{Levy Continuity Theorem}
Suppose \(X,X_{1},\ X_{2},\ \cdots\) are r.v.'s of a probability space $(\Omega,\ma, P)$. 

\(\mathrm{\text{If}}\ \ \exists\delta > 0\  \Rightarrow \ \lim\limits_{n \rightarrow \infty}{M_{X_{n}}(t)} = M_{X}(t),\ \forall t \in \left( - \delta,\delta \right),\) \( \mathrm{\text{then}}\)\[ \lim_{n \rightarrow \infty}F_{n}(x) = F(x)\]
\(\mathrm{\text{if }}F(x)\text{\ }\mathrm{\text{is\ continuous\ at}}\ X.\)
\end{theorem}

\begin{proof}
\\[4cm]\vspace{0.01cm}
\end{proof}

\begin{theorem}{Central Limit Theorem (CLT)}{Central Limit Theorem (CLT)}
Suppose \(X_{1},\ X_{2},\ \cdots,X_{n}\) are i.i.d. r.v.'s with mean \(\mu\) and variance \(\sigma^{2}\). Let

\[S_{n}^{*} = \frac{X_{1} + X_{2} + \cdots + X_{n} - E[S_{n}]}{\sigma_{S_{n}}} = \frac{X_{1} + X_{2} + \cdots + X_{n} - n\mu}{\sigma\sqrt{n}}.\]
Then
\[\lim_{n \rightarrow \infty}{F_{S_{n}^{*}}(X) = \Phi(x)},\]
i.e.,
\[{\lim_{n \rightarrow \infty}P\left( \frac{X_{1} + X_{2} + \cdots + X_{n} - n\mu}{\sigma\sqrt{n}} \ls x \right)}{= \Phi(x)} = \int_{- \infty}^{x}\frac{1}{\sqrt{2\pi}}e^{- \tfrac{y^{2}}{2}}\dif y.\]
Equivalently,
\[\begin{aligned}
{\lim_{n \rightarrow \infty}P\left( \frac{\overline{X} - \mu}{\frac{\sigma}{\sqrt{n}}} \ls x \right)}&= \lim_{n \rightarrow \infty}P\left( \frac{\overline{X} - \mu}{\sqrt{\frac{Var(X)}{n}}} \ls x \right)\\
& = \lim_{n \rightarrow \infty}P\left( \frac{\overline{X} - E\left\lbrack\, \overline{X}\, \right\rbrack}{\sigma_{\overline{X}}} \ls x \right) \\
&= \Phi(x).\\
\end{aligned}\]
\end{theorem}

\begin{proof}
\\[4cm]\vspace{0.01cm}
\end{proof}

%%=====模板========================================
%%================================================

%\chapter{Templates}
%
%$X_1,\ X_2,\ \cdots,X_n$
%$(\Omega,\ma, P)$
%
%\begin{definition}{}{}
%
%\end{definition}
%
%\begin{remark}{}{}
%
%\end{remark}
%
%\begin{theorem}{}{}
%
%\end{theorem}
%
%\begin{corollary}{}{}
%
%\end{corollary}
%
%\begin{lemma}{}{}
%
%\end{lemma}
%
%\begin{proof}
%
%\end{proof}

\end{document}
